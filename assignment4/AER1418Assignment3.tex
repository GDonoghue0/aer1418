\documentclass{article}


\usepackage{amsmath}
\usepackage{amssymb}
\usepackage{cancel}
\usepackage{graphicx}
\usepackage{float}
\usepackage{matlab-prettifier}
\usepackage[margin=1.0in]{geometry}




% bold letters
\newcommand{\bola}{\mathbf{a}}
\newcommand{\bolc}{\mathbf{c}}
\newcommand{\bolf}{\mathbf{f}}
\newcommand{\bolg}{\mathbf{g}}
\newcommand{\boll}{\mathbf{l}}
\newcommand{\bolq}{\mathbf{q}}
\newcommand{\bolp}{\mathbf{p}}
\newcommand{\bolu}{\mathbf{u}}
\newcommand{\bolv}{\mathbf{v}}
\newcommand{\bolw}{\mathbf{w}}
\newcommand{\bolz}{\mathbf{z}}

\newcommand{\bolA}{\mathbf{A}}
\newcommand{\bolB}{\mathbf{B}}
\newcommand{\bolC}{\mathbf{C}}
\newcommand{\bolD}{\mathbf{D}}
\newcommand{\bolE}{\mathbf{E}}
\newcommand{\bolF}{\mathbf{F}}
\newcommand{\bolG}{\mathbf{G}}
\newcommand{\bolH}{\mathbf{H}}
\newcommand{\bolI}{\mathbf{I}}
\newcommand{\bolJ}{\mathbf{J}}
\newcommand{\bolK}{\mathbf{K}}
\newcommand{\bolL}{\mathbf{L}}
\newcommand{\bolM}{\mathbf{M}}
\newcommand{\bolN}{\mathbf{N}}
\newcommand{\bolO}{\mathbf{O}}
\newcommand{\bolP}{\mathbf{P}}
\newcommand{\bolQ}{\mathbf{Q}}
\newcommand{\bolR}{\mathbf{R}}
\newcommand{\bolS}{\mathbf{S}}
\newcommand{\bolT}{\mathbf{T}}
\newcommand{\bolU}{\mathbf{U}}
\newcommand{\bolV}{\mathbf{V}}
\newcommand{\bolW}{\mathbf{W}}
\newcommand{\bolX}{\mathbf{X}}
\newcommand{\bolY}{\mathbf{Y}}
\newcommand{\bolZ}{\mathbf{Z}}

% bold symbols
\newcommand{\bolalpha}{\boldsymbol{\alpha}}
\newcommand{\bolbeta}{\boldsymbol{\beta}}
\newcommand{\boleta}{\boldsymbol{\eta}}
\newcommand{\bolpsi}{\boldsymbol{\psi}}

% shadowed letters
\newcommand{\PP}{\mathbb{P}}
\newcommand{\RR}{\mathbb{R}}
\newcommand{\CC}{\mathbb{C}}
\newcommand{\ZZ}{\mathbb{Z}}

% mathcal letters
\newcommand{\calA}{\mathcal{A}}
\newcommand{\calB}{\mathcal{B}}
\newcommand{\calC}{\mathcal{C}}
\newcommand{\calD}{\mathcal{D}}
\newcommand{\calE}{\mathcal{E}}
\newcommand{\calF}{\mathcal{F}}
\newcommand{\calG}{\mathcal{G}}
\newcommand{\calH}{\mathcal{H}}
\newcommand{\calI}{\mathcal{I}}
\newcommand{\calJ}{\mathcal{J}}
\newcommand{\calK}{\mathcal{K}}
\newcommand{\calL}{\mathcal{L}}
\newcommand{\calM}{\mathcal{M}}
\newcommand{\calN}{\mathcal{N}}
\newcommand{\calO}{\mathcal{O}}
\newcommand{\calP}{\mathcal{P}}
\newcommand{\calQ}{\mathcal{Q}}
\newcommand{\calR}{\mathcal{R}}
\newcommand{\calS}{\mathcal{S}}
\newcommand{\calT}{\mathcal{T}}
\newcommand{\calU}{\mathcal{U}}
\newcommand{\calV}{\mathcal{V}}
\newcommand{\calW}{\mathcal{W}}
\newcommand{\calX}{\mathcal{X}}
\newcommand{\calY}{\mathcal{Y}}
\newcommand{\calZ}{\mathcal{Z}}


% derivatives
\newcommand{\pp}[2]{\frac{\partial #1}{\partial #2}}
\newcommand{\dd}[2]{\frac{d #1}{d #2}}


% fraction shortcut
\newcommand{\f}[2]{\frac{#1}{#2}}
\newcommand{\slfrac}[2]{\left.#1\middle/#2\right.}

% common operators
\newcommand{\vvvert}{|\kern-1pt|\kern-1pt|}
\newcommand{\enorm}[1]{\vvvert #1 \vvvert}


% matrices
\newcommand{\bmat}[1]{\left(\begin{array}{#1}}
\newcommand{\emat}{\end{array}\right)} 



\newcommand{\blist}{\begin{list}{\ballrefb}{\leftmargin=2.0em}
  \setlength{\itemsep}{2pt}
  \setlength{\parskip}{0pt}}
\newcommand{\elist}{\end{list}}



% common format strings
\def\etal{{\it et al.}}
\def\ie{{\it i.e.}}
\def\eg{{\it e.g.}}


\DeclareMathOperator*{\arginf}{arg\,inf}
\DeclareMathOperator*{\argsup}{arg\,sup}
\DeclareMathOperator*{\argmax}{arg\,max}
\DeclareMathOperator*{\argmin}{arg\,min}


\begin{document}
\Large\centering AER1418: Assignment 4\\
\normalsize\raggedright Geoff Donoghue \hfill April 23, 2019\\

\section*{Part 1. Advection-diffusion equation}
\begin{itemize}
	\item[(a)] We wish to show that \(\exists\gamma < \infty \) such that
	\begin{equation*}
		a_h(w,v) \equiv a(w,v) + \left(\tau\calL w, \calL v \right)_{L^2(\Omega)} \leq \gamma\|w\|_{\calV_h}\|v\|_{\calV_h}, \quad \forall w,v\in \calV_h.
	\end{equation*}
	Since we already know that
	\begin{equation*}
		a(w,v) \leq \left(\|\kappa\|_{L^\infty(\Omega)} + \|b\|_{L^\infty(\Omega)} + C^2_\text{tr}\|b\|_{L^\infty(\Gamma_N)} \right)\|w\|_{\calV_h}\|v\|_{\calV_h},
	\end{equation*}
	we seek some \(\gamma' < \infty \) such that
	\begin{equation*}
		\left(\tau\calL w, \calL v \right)_{L^2(\Omega)} \leq \gamma'\|w\|_{\calV_h}\|v\|_{\calV_h}, \quad \forall w,v\in \calV_h.
	\end{equation*}
	We first apply the Cauchy-Schwarz inequality to the leas-squares term, obtaining
	\begin{equation*}
		\left(\tau\calL w, \calL v \right)_{L^2(\Omega)} \leq \|\tau\calL w\|_{L^2(\Omega)}\|\calL v\|_{L^2(\Omega)}.
	\end{equation*}
	Now using the definition of \(\calL \), we can say that
	\begin{equation*}
		\left(\tau\calL w, \calL v \right)_{L^2(\Omega)} \leq \|\tau(-\nabla\cdot(\kappa\nabla w) + \nabla\cdot(bw))\|_{L^2(\Omega)}\|-\nabla\cdot(\kappa\nabla v) + \nabla\cdot(bv)\|_{L^2(\Omega)}.
	\end{equation*}
	By applying the triangle inequality to the norms containing sums we arrive at
	\begin{equation*}
		\left(\tau\calL w, \calL v \right)_{L^2(\Omega)} \leq (\|-\tau\nabla\cdot(\kappa\nabla w)\|_{L^2(\Omega)} + \|\tau\nabla\cdot(bw)\|_{L^2(\Omega)})(\|-\nabla\cdot(\kappa\nabla v)\|_{L^2(\Omega)} + \|\nabla\cdot(bv)\|_{L^2(\Omega)}).
	\end{equation*}
	If we assume that our advection and diffusion fields are constant we may reexpress the above as
	\begin{equation*}
		\left(\tau\calL w, \calL v \right)_{L^2(\Omega)} \leq \tau(\kappa\|\nabla^2 w\|_{L^2(\Omega)} + b\cdot\|\nabla w\|_{L^2(\Omega)})(\kappa\|\nabla^2 v\|_{L^2(\Omega)} + b\cdot\|\nabla v\|_{L^2(\Omega)}).
	\end{equation*}
	Using the definitions of the \(H^2(\Omega) \) and \(H^1(\Omega) \) semi-norms we can say that
	\begin{equation*}
		\left(\tau\calL w, \calL v \right)_{L^2(\Omega)} \leq \tau(\kappa|w|_{H^2(\Omega)} + b\cdot| w|_{H^1(\Omega)})(\kappa|v|_{H^2(\Omega)} + b\cdot|v|_{H^1(\Omega)}).
	\end{equation*}
	Since we have \(\calV_h \subset H^1(\Omega) \), \(\overline{\Omega} = \bigcup_{K\in\calT_h}\overline{K}\), and we assume the inverse esimtate estmate \(|v|_{H^2(K)} \leq c_\text{inv}h^{-1}\|v\|_{H^1(K)} \forall v \in \calV_h \) our inequality now becomes
	\begin{equation*}
		\left(\tau\calL w, \calL v \right)_{L^2(\Omega)} \leq \tau(\kappa c_\text{inv}h^{-1}\|w\|_{H^1(K)} + b\cdot| w|_{H^1(\Omega)})(\kappa c_\text{inv}h^{-1}\|v\|_{H^1(K)} + b\cdot|v|_{H^1(\Omega)}).
	\end{equation*}
	From the definition of the \(H^1(\Omega) \) seminorm, \(\|v\|_{H^1(\Omega)} \equiv \|v\|_{L^2(\Omega)} + |v|_{H^1(\Omega)}, \; \forall v \in H^1(\Omega) \), we note that by the non-negativity of \(\|v\|_{L^2(\Omega)} \) implies that \(\|v\|_{H^1(\Omega)} \geq |v|_{H^1(\Omega)} \), applying this to our inequality we arrive at
	\begin{equation*}
		\left(\tau\calL w, \calL v \right)_{L^2(\Omega)} \leq \tau(\kappa c_\text{inv}h^{-1}\|w\|_{H^1(K)} + b\cdot\| w\|_{H^1(\Omega)})(\kappa c_\text{inv}h^{-1}\|v\|_{H^1(K)} + b\cdot\|v\|_{H^1(\Omega)}).
	\end{equation*}
	\begin{equation*}
		\left(\tau\calL w, \calL v \right)_{L^2(\Omega)} \leq \tau(\kappa c_\text{inv}h^{-1}+b)\|w\|_{H^1(K)} (\kappa c_\text{inv}h^{-1} + b)\|v\|_{H^1(K)}.
	\end{equation*}
	Since when we invoked the inverse estimate inequality, we can now express \(\tau\) in the limit as \(h \to 0 \), which is \(\tau = \dfrac{h^2}{12\kappa} \), this updates our inequality to become
	\begin{equation*}
		\left(\tau\calL w, \calL v \right)_{L^2(\Omega)} \leq \frac{h^2}{12\kappa}(\kappa c_\text{inv}h^{-1}+b)^2 \|w\|_{H^1(K)} \|v\|_{H^1(K)}.
	\end{equation*}
	Since \(h \to 0\) the only term that survives is
	\begin{equation*}
		\left(\tau\calL w, \calL v \right)_{L^2(\Omega)} \leq \frac{\kappa}{12}c_\text{inv}^2 \|w\|_{H^1(K)} \|v\|_{H^1(K)}.
	\end{equation*}
	Therefore we say that the bilinear form of the advection-diffusion equation arising from the GLS discretization is continuous with a continuity constant
	\begin{equation*}
		\gamma = \|\kappa\|_{L^\infty(\Omega)} + \|b\|_{L^\infty(\Omega)} + C^2_\text{tr}\|b\|_{L^\infty(\Gamma_N)} + \frac{\kappa}{12}c_\text{inv}^2.
	\end{equation*}
	
	\item[(b)] Assuming a constant diffusion field, our problem is given by
	\begin{equation*}
		-\kappa\left(\dfrac{\partial^2u}{\partial x_1^2} + \dfrac{\partial^2u}{\partial x_2^2} \right) + \pp{u}{x_1} = 0.
	\end{equation*}
	We note that this problem is separable and express our solution as the product of monomial functions \(u = X_1X_2 \) and obtain
	\begin{equation*}
		-\kappa\dfrac{1}{X_1}\dfrac{\partial^2X_1}{\partial x_1^2} + \dfrac{1}{X_1}\pp{X_1}{x_1} = \kappa \dfrac{1}{X_2}\dfrac{\partial^2X_2}{\partial x_2^2} = m,
	\end{equation*}
	where \(m\) is some seperation constant. The ordinary differential equation in \(x_2 \),
	\begin{equation*}
		\dfrac{\partial^2X_2}{\partial x_2^2} = \dfrac{m}{\kappa}X_2,
	\end{equation*}
	is trivial to solve. The solution is given by
	\begin{equation*}
		X_2(x_2) = c_1\exp\left(\sqrt{\frac{m}{\kappa}}x_2\right) + c_2\exp\left(-\sqrt{\frac{m}{\kappa}}x_2\right).
	\end{equation*}
	Our boundary conditions state that
	\begin{equation*}
		\left.\dd{X_2}{x_2}\right|_{x_2 = 0} = 0 = c_1\sqrt{\dfrac{m}{\kappa}} - c_2\sqrt{\dfrac{m}{\kappa}} \implies c_1 = c_2,
	\end{equation*}
	\begin{equation*}
		\left.\dd{X_2}{x_2}\right|_{x_2 = 1} = 0 = c_1\sqrt{\dfrac{m}{\kappa}}\exp\left(\sqrt{\frac{m}{\kappa}}\right) - c_1\sqrt{\dfrac{m}{\kappa}}\exp\left(-\sqrt{\frac{m}{\kappa}}\right) \implies c_1 \vee m = 0.
	\end{equation*}
\end{itemize}

\section*{Part 2. Error estimation and adaptive mesh refinement}

\section*{Part 3. Adaptive eigensolver}


\section*{Appendix}
\end{document}
