\documentclass{article}


\usepackage{amsmath}
\usepackage{amssymb}
\usepackage{cancel}
\usepackage{graphicx}
\usepackage{float}
\usepackage{matlab-prettifier}
\usepackage[margin=1.0in]{geometry}


\include{commands}

\begin{document}
\Large\centering AER1418: Assignment 4\\
\normalsize\raggedright Geoff Donoghue \hfill April 23, 2019\\

\section*{Part 1. Advection-diffusion equation}
\begin{itemize}
	\item[(a)] We wish to show that \(\exists\gamma < \infty \) such that
	\begin{equation*}
		a_h(w,v) \equiv a(w,v) + \left(\tau\calL w, \calL v \right)_{L^2(\Omega)} \leq \gamma\|w\|_{\calV_h}\|v\|_{\calV_h}, \quad \forall w,v\in \calV_h.
	\end{equation*}
	Since we already know that
	\begin{equation*}
		a(w,v) \leq \left(\|\kappa\|_{L^\infty(\Omega)} + \|b\|_{L^\infty(\Omega)} + C^2_\text{tr}\|b\|_{L^\infty(\Gamma_N)} \right)\|w\|_{\calV_h}\|v\|_{\calV_h},
	\end{equation*}
	we seek some \(\gamma' < \infty \) such that
	\begin{equation*}
		\left(\tau\calL w, \calL v \right)_{L^2(\Omega)} \leq \gamma'\|w\|_{\calV_h}\|v\|_{\calV_h}, \quad \forall w,v\in \calV_h.
	\end{equation*}
	We first apply the Cauchy-Schwarz inequality to the leas-squares term, obtaining
	\begin{equation*}
		\left(\tau\calL w, \calL v \right)_{L^2(\Omega)} \leq \|\tau\calL w\|_{L^2(\Omega)}\|\calL v\|_{L^2(\Omega)}.
	\end{equation*}
	Now using the definition of \(\calL \), we can say that
	\begin{equation*}
		\left(\tau\calL w, \calL v \right)_{L^2(\Omega)} \leq \|\tau(-\nabla\cdot(\kappa\nabla w) + \nabla\cdot(bw))\|_{L^2(\Omega)}\|-\nabla\cdot(\kappa\nabla v) + \nabla\cdot(bv)\|_{L^2(\Omega)}.
	\end{equation*}
	By applying the triangle inequality to the norms containing sums we arrive at
	\begin{equation*}
		\left(\tau\calL w, \calL v \right)_{L^2(\Omega)} \leq (\|-\tau\nabla\cdot(\kappa\nabla w)\|_{L^2(\Omega)} + \|\tau\nabla\cdot(bw)\|_{L^2(\Omega)})(\|-\nabla\cdot(\kappa\nabla v)\|_{L^2(\Omega)} + \|\nabla\cdot(bv)\|_{L^2(\Omega)}).
	\end{equation*}
	If we assume that our advection and diffusion fields are constant we may reexpress the above as
	\begin{equation*}
		\left(\tau\calL w, \calL v \right)_{L^2(\Omega)} \leq \tau(\kappa\|\nabla^2 w\|_{L^2(\Omega)} + b\cdot\|\nabla w\|_{L^2(\Omega)})(\kappa\|\nabla^2 v\|_{L^2(\Omega)} + b\cdot\|\nabla v\|_{L^2(\Omega)}).
	\end{equation*}
	Using the definitions of the \(H^2(\Omega) \) and \(H^1(\Omega) \) semi-norms we can say that
	\begin{equation*}
		\left(\tau\calL w, \calL v \right)_{L^2(\Omega)} \leq \tau(\kappa|w|_{H^2(\Omega)} + b\cdot| w|_{H^1(\Omega)})(\kappa|v|_{H^2(\Omega)} + b\cdot|v|_{H^1(\Omega)}).
	\end{equation*}
	Since we have \(\calV_h \subset H^1(\Omega) \), \(\overline{\Omega} = \bigcup_{K\in\calT_h}\overline{K}\), and we assume the inverse esimtate estmate \(|v|_{H^2(K)} \leq c_\text{inv}h^{-1}\|v\|_{H^1(K)} \forall v \in \calV_h \) our inequality now becomes
	\begin{equation*}
		\left(\tau\calL w, \calL v \right)_{L^2(\Omega)} \leq \tau(\kappa c_\text{inv}h^{-1}\|w\|_{H^1(K)} + b\cdot| w|_{H^1(\Omega)})(\kappa c_\text{inv}h^{-1}\|v\|_{H^1(K)} + b\cdot|v|_{H^1(\Omega)}).
	\end{equation*}
	From the definition of the \(H^1(\Omega) \) seminorm, \(\|v\|_{H^1(\Omega)} \equiv \|v\|_{L^2(\Omega)} + |v|_{H^1(\Omega)}, \; \forall v \in H^1(\Omega) \), we note that by the non-negativity of \(\|v\|_{L^2(\Omega)} \) implies that \(\|v\|_{H^1(\Omega)} \geq |v|_{H^1(\Omega)} \), applying this to our inequality we arrive at
	\begin{equation*}
		\left(\tau\calL w, \calL v \right)_{L^2(\Omega)} \leq \tau(\kappa c_\text{inv}h^{-1}\|w\|_{H^1(K)} + b\cdot\| w\|_{H^1(\Omega)})(\kappa c_\text{inv}h^{-1}\|v\|_{H^1(K)} + b\cdot\|v\|_{H^1(\Omega)}).
	\end{equation*}
	\begin{equation*}
		\left(\tau\calL w, \calL v \right)_{L^2(\Omega)} \leq \tau(\kappa c_\text{inv}h^{-1}+b)\|w\|_{H^1(K)} (\kappa c_\text{inv}h^{-1} + b)\|v\|_{H^1(K)}.
	\end{equation*}
	Since when we invoked the inverse estimate inequality, we can now express \(\tau\) in the limit as \(h \to 0 \), which is \(\tau = \dfrac{h^2}{12\kappa} \), this updates our inequality to become
	\begin{equation*}
		\left(\tau\calL w, \calL v \right)_{L^2(\Omega)} \leq \frac{h^2}{12\kappa}(\kappa c_\text{inv}h^{-1}+b)^2 \|w\|_{H^1(K)} \|v\|_{H^1(K)}.
	\end{equation*}
	Since \(h \to 0\) the only term that survives is
	\begin{equation*}
		\left(\tau\calL w, \calL v \right)_{L^2(\Omega)} \leq \frac{\kappa}{12}c_\text{inv}^2 \|w\|_{H^1(K)} \|v\|_{H^1(K)}.
	\end{equation*}
	Therefore we say that the bilinear form of the advection-diffusion equation arising from the GLS discretization is continuous with a continuity constant
	\begin{equation*}
		\gamma = \|\kappa\|_{L^\infty(\Omega)} + \|b\|_{L^\infty(\Omega)} + C^2_\text{tr}\|b\|_{L^\infty(\Gamma_N)} + \frac{\kappa}{12}c_\text{inv}^2.
	\end{equation*}
	
	\item[(b)] Assuming a constant diffusion field, our problem is given by
	\begin{equation*}
		-\kappa\left(\dfrac{\partial^2u}{\partial x_1^2} + \dfrac{\partial^2u}{\partial x_2^2} \right) + \pp{u}{x_1} = 0.
	\end{equation*}
	We note that this problem is separable and express our solution as the product of monomial functions \(u = X_1X_2 \) and obtain
	\begin{equation*}
		-\kappa\dfrac{1}{X_1}\dfrac{\partial^2X_1}{\partial x_1^2} + \dfrac{1}{X_1}\pp{X_1}{x_1} = \kappa \dfrac{1}{X_2}\dfrac{\partial^2X_2}{\partial x_2^2} = m,
	\end{equation*}
	where \(m\) is some seperation constant. The ordinary differential equation in \(x_2 \),
	\begin{equation*}
		\dfrac{\partial^2X_2}{\partial x_2^2} = \dfrac{m}{\kappa}X_2,
	\end{equation*}
	is trivial to solve. The solution is given by
	\begin{equation*}
		X_2(x_2) = c_1\exp\left(\sqrt{\frac{m}{\kappa}}x_2\right) + c_2\exp\left(-\sqrt{\frac{m}{\kappa}}x_2\right).
	\end{equation*}
	Our boundary conditions state that
	\begin{equation*}
		\left.\dd{X_2}{x_2}\right|_{x_2 = 0} = 0 = c_1\sqrt{\dfrac{m}{\kappa}} - c_2\sqrt{\dfrac{m}{\kappa}} \implies c_1 = c_2,
	\end{equation*}
	\begin{equation*}
		\left.\dd{X_2}{x_2}\right|_{x_2 = 1} = 0 = c_1\sqrt{\dfrac{m}{\kappa}}\exp\left(\sqrt{\frac{m}{\kappa}}\right) - c_1\sqrt{\dfrac{m}{\kappa}}\exp\left(-\sqrt{\frac{m}{\kappa}}\right) \implies c_1 \vee m = 0.
	\end{equation*}
\end{itemize}

\section*{Part 2. Error estimation and adaptive mesh refinement}

\section*{Part 3. Adaptive eigensolver}


\section*{Appendix}
\end{document}
