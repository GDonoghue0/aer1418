\chapter{Linear elasticity}

\section{Motivation}
In this lecture we consider approximation of linear-elasticity problems by the finite element method.

\section{Vector-valued Sobolev spaces}
The Navier-Cauchy equation in $\RR^d$ is a system of equations with $d$ components. By way of preliminaries, we introduce vector-valued Sobolev spaces.
\begin{definition}[$H^1(\Omega)^d$ space]
  Given $\Omega \subset \RR^d$, the space of vector-valued functions $H^1(\Omega)^d$ is endowed with an inner product
  \begin{equation*}
    (w,v)_{H^1(\Omega)} \equiv \sum_{i=1}^d (w_i,v_i)_{H^1(\Omega)},
  \end{equation*}
  equipped with the associated induced norm $\| w \|_{H^1(\Omega)} \equiv \sqrt{(w,w)_{H^1(\Omega)}}$, and comprises function
  \begin{equation*}
    H^1(\Omega)^d \equiv \{ v \ | \ \| v \|_{H^1(\Omega)} < \infty \}.
  \end{equation*}
  Here, $v_i$ denotes the $i$-th component of the vector-valued field for $i = 1,\dots,d$. In other words, for $v \in H^1(\Omega)^d$, we have $v: \Omega \to \RR^d$ and $v_i \in H^1(\Omega)$.
\end{definition}
\begin{definition}[gradient of $H^1(\Omega)^d$ functions]
  For $v \in H^1(\Omega)^d$, the gradient $\nabla v L^2(\Omega)^{d \times d}$ is a tensor-valued field in such that
\begin{equation*}
  (\nabla v)_{ij} = \pp{v_i}{x_j}, \quad i,j = 1,\dots,d;
\end{equation*}
\end{definition}
\begin{corollary}
  For $v \in H^1(\Omega)^2$, the gradient $\nabla v \in L^2(\Omega)^{2 \times 2}$ is given by
\begin{equation*}
  \nabla v = \bmat{cc}
  \pp{v_1}{x_1} & \pp{v_1}{x_2} \\
  \pp{v_2}{x_1} & \pp{v_2}{x_2}
  \emat.
\end{equation*}
\end{corollary}
\begin{definition}[divergence of $H^1(\Omega)^d$ functions]
For $v \in H^1(\Omega)^d$, the divergence of $v \in L^2(\Omega)$ is a scalar-valued field such that
\begin{equation*}
  \nabla \cdot v = \sum_{i = 1}^d \dd{v_i}{x_i};
\end{equation*}
\end{definition}
\begin{corollary}
  For $v \in H^1(\Omega)^2$, the divergence $\nabla \cdot v \in L^2(\Omega)$ is given by
\begin{equation*}
  \nabla \cdot v = \pp{v_1}{x_1} + \pp{v_2}{x_2}.
\end{equation*}
\end{corollary}
\begin{definition}[divergence of $H^1(\Omega)^{d \times d}$ functions]
For $\sigma \in H^1(\Omega)^{d \times d}$, its divergence $\nabla \cdot \sigma \in L^2(\Omega)^d$ is a vectors-valued field such that
\begin{equation*}
  (\nabla \cdot \sigma)_i = \sum_{i=1}^d \pp{\sigma_{ij}}{x_j}.
\end{equation*}
\end{definition}
\begin{corollary}
  For $\sigma \in H^1(\Omega)^{2 \times 2}$, the divergence $\nabla \cdot \sigma \in L^2(\Omega)^2$ is given by
  \begin{equation*}
    \nabla \cdot \sigma =
    \bmat{c}
    \pp{\sigma_{11}}{x_1} + \pp{\sigma_{12}}{x_2} \\
    \pp{\sigma_{21}}{x_1} + \pp{\sigma_{22}}{x_2}
    \emat
  \end{equation*}
\end{corollary}
\section{Formulation}
We now formulate the linear elasticity problem.  Given a displacement field $v \in H^1(\Omega)^d$, we first introduce the strain tensor (field) $\epsilon(v) \in L^2(\Omega)^{d \times d}$ such that
\begin{equation*}
  \epsilon(v) = \frac{1}{2} (\nabla v + \nabla v^T).
\end{equation*}
We next introduce the associated stress tensor (field) $\sigma(v) \in L^2(\Omega)^{d \times d}$ such that
\begin{equation*}
  \sigma(v) = 2 \mu \epsilon(v) + \lambda \text{tr}(\epsilon(v)) I,
\end{equation*}
where $\lambda \in \RR_{>0}$ and $\mu \in \RR_{>0}$ are the Lam\'e constants.

Let $\Omega \subset \RR^d$ be a Lipschitz domain. We partition the boundary of $\Omega$ to the Dirichlet boundary $\Gamma_D$ and the Neumann boundary $\Gamma_N$ such that $\partial \Omega = \bar{\Gamma}_D \cup \bar{\Gamma}_N$. We then introduce the strong form of the linear elasticity problem: find $u$ such that
\begin{align*}
  - \nabla \cdot \sigma(u) &= f \quad \text{in } f \\
  u &= u^B \quad \text{on } \Gamma_{D}, \\
  n \cdot \sigma(u) &= g \quad \text{on } \Gamma_{N},
\end{align*}
where $f: \Omega \to \RR^d$ is the body force field and $g: \Gamma_N \to \RR^d$ is the traction force field.

We now derive a weak formulation for the linear elasticity problem.  To this end, we first introduce a Hilbert space
\begin{equation}
  \calV \equiv \{ v \in H^1(\Omega)^d \ | \ v|_{\Gamma_D} = 0 \}
  \label{eq:le_calV}
\end{equation}
and an affine space
\begin{equation*}
  \calV^E \equiv u^E + \calV,
\end{equation*}
where $u^E$ is any function in $H^1(\Omega)^d$ such that $u^E|_{\Gamma_D} = u^B$ on $\Gamma_D$; we recall that Dirichlet boundary conditions are essential boundary conditions that must be explicitly through the construction of the space.  We now take an arbitrary test function $v \in \calV$, multiply the equation by $v$, integrate by parts, and make appropriate substitutions for the natural boundary conditions: 
\begin{align*}
  0 &= 
  \int_{\Omega} v \cdot (-\nabla \cdot \sigma(u) ) dx - \int_{\Omega} v \cdot f dx \\
  &=
  \int_{\Omega} \nabla v : \sigma (u) dx - \underbrace{ \int_{\Gamma_D} v n \cdot \sigma(u) ds}_{= 0 \ : \ \text{Dirichlet BC}} - \int_{\Gamma_{N}} v \underbrace{ n \cdot \sigma(u)}_{g \ : \ \text{Neumann BC}} ds - \int_{\Omega} v \cdot f dx \\
  &=
  \int_{\Omega} \nabla v : \sigma (u) dx - \int_{\Gamma_N} v \cdot g ds - \int_{\Omega} v \cdot f dx 
\end{align*}
We can further simplify the term involving integration over $\Omega$.  We first note that, because $\epsilon(u)$ is symmetric,
\begin{equation*}
  \nabla v : \epsilon(u) = \epsilon(v) : \epsilon(u) .
\end{equation*}
We next note that $\nabla v: I = \text{tr}(\nabla v)$ and $\text{tr}(\nabla v) = \text{tr}(\epsilon(v))$ because $\epsilon(\cdot)$ preserves the diagonal terms:
\begin{equation*}
  = \nabla v : \text{tr}(\epsilon(u)) I 
  = (\nabla v : I) \text{tr}(\epsilon(u)) 
  = \text{tr}(\epsilon(v)) \text{tr}(\epsilon(u)).
\end{equation*}
It hence follows that
\begin{equation*}
  \nabla v : \sigma(u) = \nabla v : (2 \mu \epsilon(u) + \lambda \text{tr}(\epsilon(u)) I )
  = 2 \mu \epsilon(v) : \epsilon(u) + \lambda \text{tr}(\epsilon(v)) \text{tr}(\epsilon(u))
\end{equation*}
Our weak formulation is as follows: find $u \in \calV^E$ such that
\begin{equation}
  a(u,v) = \ell(v) \quad \forall v \in \calV,
  \label{eq:le_weak}
\end{equation}
where
\begin{align}
  a(w,v) &= \int_{\Omega} ( 2 \mu \epsilon(v) : \epsilon(w) + \lambda \text{tr}(\epsilon(v)) \text{tr}(\epsilon(w)) ) dx \quad \forall w, v \in \calV \label{eq:le_a} \\
  \ell(v) &=   \int_{\Omega} v \cdot f dx + \int_{\Gamma_N} v \cdot g ds \quad \forall v \in \calV \label{eq:le_ell}.
\end{align}

\section{Well-posedness}
We now wish to understand if a solution to the variational problem~\eqref{eq:le_weak} exists and, if so, is unique.  To this end, we introduce the \emph{Korn's inequality}.
\begin{theorem}[Korn's inequality]
  Let $\calV \subset H^1(\Omega)^d$ be given by~\eqref{eq:le_calV}.  There exists $\alpha > 0$ such that
  \begin{equation*}
    %    \| \epsilon(v) \|_{L^2(\Omega)} + \| v \|_{L^2(\Omega)} \geq \alpha \| v \|_{H^1(\Omega)} \quad \forall v \in H^1(\Omega)^d.
    \| \epsilon(v) \|_{L^2(\Omega)} \geq C_{\rm Korn} \| v \|_{H^1(\Omega)} \quad \forall v \in \calV.
  \end{equation*}
\end{theorem}
\begin{proposition}
  The bilinear form~\eqref{eq:le_a} associated with the linear elasticity problem is symmetric, coercive, and continuous in $\calV$ given by~\eqref{eq:le_calV}.
  \begin{proof}
    The symmetry of $a(\cdot,\cdot)$ is obvious from inspection. The coercivity of $a(\cdot,\cdot)$ is a consequence of the Korn's inequality: for any $v \in \calV$
    \begin{equation*}
      a(v,v) = 2 \mu \| \epsilon(v) \|_{L^2(\Omega)} + \lambda \| \text{tr}(\epsilon(v)) \|_{L^2(\Omega)}
      \geq 2 \mu \| \epsilon(v) \|_{L^2(\Omega)}
      \geq 2 \mu C_{\rm Korn} \| v \|_{H^1(\Omega)}.
    \end{equation*}
    It follows that $a(\cdot,\cdot)$ is coercive with the coercivity constant $\alpha \equiv 2 \mu C_{\rm Korn}$. The continuity
    \begin{equation*}
      a(w,v) = 2\mu (\epsilon(w), \epsilon(v))
    \end{equation*}
  \end{proof}
\end{proposition}

