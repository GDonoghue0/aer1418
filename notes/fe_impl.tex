\chapter{Finite element method: implementation}

\section{Introduction}
In this lecture, we introduce technical ingredients required to implement the finite element method. This lecture is organized as follows:
\begin{itemize}
\item Section~\ref{sec:fe_ref_elem} introduces the concept of reference finite element and a few common elements.  Techniques to generate finite elements will be discussed in the section.
\item Section~\ref{sec:fe_phy_elem} introduces physical elements which comprise the actual approximation space $\calV_h \subset \calV$.  Techniques to map the reference elements to physical elements will be discussed in the section.
\item Section~\ref{sec:fe_quad} introduces the concept of numerical quadrature which we use to perform integrals that appear in bilinear forms and linear forms.
\item Section~\ref{sec:fe_assembly} introduces how to put together ingredients from the previous sections to assemble the stiffness matrix and load vector.
\end{itemize}


%% Specifically, we introduce ingredients required to compute the stiffness matrix $\hat A_h \in \RR^{n \times n}$ such that
%% \begin{equation*}
%%   \hat A_{h,ij} = a(\phi_j,\phi_i) \quad \forall i,j = 1,\dots,n,
%% \end{equation*}
%% and the load vector $\hat f_h \in \RR^n$ such that
%% \begin{equation*}
%%   \hat f_{h,i} = \ell(\phi_i) \quad \forall i = 1,\dots,n,
%% \end{equation*}
%% where $\{ \phi_i \}_{i=1}^n$ is a basis for the approximation space $\calV_h$ of a dimension $n$.

%% We compute the 
%% The computation of the stiffness matrix and the load vector requires the following ingredients:
%% \begin{enumerate}
%% \item the ability to evaluate function values and gradients for a reference element $\tilde K$.
%% \item the ability to evaluate basis functions values and gradients for a physical element $K \in \calT_h$.
%% \item 
%% \end{enumerate}




%% We need three key ingredients
%% \begin{itemize}
%% \item[1.] a means to evaluate basis function values and gradients at any points in a reference element $\tilde K$.
%% \item[2.] a means to transform the integral over the element $K$ to an integral over the reference element $\tilde K$.
%% \item[3.] a means to approximate the integral over the reference element $\tilde K$ using a quadrature rule.
%% \end{itemize}

%% As we have seen in Lecture~\ref{ch:pos1d}, the computation of a finite element solution requires a few common tasks on the functions in $\calV_h$, such as the evaluation of the function values, the valuation of the gradient values, and the integration of the functions.  Our approach to construct the piecewise polynomial space is to first define a polynomial space on a \emph{reference element} (or a \emph{canonical element}) and then to map the polynomial functions to the actual elements that comprise the triangulation.  

\section{Reference elements}
\label{sec:fe_ref_elem}
\subsection{Reference domains}
We first introduce reference domains on which a reference finite element can be defined. The first reference domain we introduce is the \emph{reference line segment} $\tilde I \subset \RR^1$. (Note that all quantities associated with the reference space bear tilde ($\tilde \cdot$).)  While the definition of a reference line segment is not universal, our reference line segment, as shown in Figure~\ref{fig:fe_ref_line}, is a unit line segment delineated by two vertices
\begin{equation*}
  \tilde v_1 \equiv 0 \quad \text{and} \quad \tilde v_2 \equiv 1.
\end{equation*}
(In literature, it is just as common to see a reference line segment defined as $(-1,1)$.)  We consider the line segment oriented in the sense that it points from $\tilde v_1$ to $\tilde v_2$.

\begin{figure}
  \centering
  \subfigure[reference line segment $\tilde I$]{
    \includegraphics[width=0.3\textwidth]{ref_line}
    \label{fig:fe_ref_line}
  }
  \subfigure[reference triangle $\tilde T$]{
    \includegraphics[width=0.3\textwidth]{ref_tri}
    \label{fig:fe_ref_tri}
  }
  \caption{Reference line segment and triangle.\label{fig:fe_ref_elem}}
\end{figure}

  
We next introduce the \emph{reference triangle} $\tilde T \subset \RR^2$. While the definition of a reference triangle is again not universal, our reference triangle, as shown in Figure~\ref{fig:fe_ref_tri}, is a right triangle delineated by three vertices
\begin{equation*}
  \tilde v_1 \equiv (0,0), \quad \tilde v_2 \equiv (1,0), \quad \text{and} \quad \tilde v_3 \equiv (0,1).
\end{equation*}
The vertices are ordered in the counter-clockwise manner, starting with the first vertex at the origin. We also denote the three \emph{facets} of the triangles by
\begin{equation*}
  \tilde F_1 \equiv ( \tilde v_2, \tilde v_3), \quad  \tilde F_2 \equiv ( \tilde v_3, \tilde v_1), \quad \text{and} \quad  \tilde F_3 \equiv ( \tilde v_1, \tilde v_2).
\end{equation*}
(A facet is a $d-1$ entity associated with a canonical shape; for a triangle, a facet is an edge.)  We choose the convention that the facet number is the same as the vertex number of the vertex on the other side of the triangle. Each facet is oriented such that the collection of the three edges defines the triangle in the counter-clockwise orientation.  

We could introduce other reference domains including, for instance, a square in $\RR^2$, a tetrahedron in $\RR^3$, or a cube in $\RR^3$; however, in this lecture, we will only consider the reference line segment $\tilde I$ and the reference triangle $\tilde T$.  

\subsection{Linear Lagrange finite element on a line segment}
\label{sec:fe_lin_line}
We introduce arguably the simplest finite element: linear Lagrange elements on the reference line segment $\tilde I \equiv (0,1) \subset \RR^1$.  To this end, we introduce \emph{Lagrange shape functions} (or \emph{Lagrange basis functions}) for the space of linear functions on $\tilde I$, $\PP^1(\tilde I)$.
We choose for our interpolation nodes $\{\tilde z_1, \tilde z_2\}$ the endpoints of the line segment:
\begin{equation*}
  \tilde z_1 \equiv 0 \quad \text{and} \quad \tilde z_2 \equiv 1,
\end{equation*}
as shown in Figure~\ref{fig:fe_ref_line_p1}.
  \begin{figure}
    \centering
    \includegraphics[width=0.3\textwidth]{ref_line_p1}
    \caption{Linear Lagrange finite element on the reference line segment.}
    \label{fig:fe_ref_line_p1}
\end{figure}
Our shape functions are linear functions $\{\tilde \phi_1, \tilde \phi_2\}$ that satisfy the interpolation condition
\begin{equation}
  \tilde \phi_i(z_j) = \delta_{ij}, \quad i,j = 1,2;  \label{eq:fe_line_interp}
\end{equation}
here $\delta_{ij}$ is the \emph{Kronecker delta} such that $\delta_{ij} = 1$ for $i = j$ and $\delta_{ij} = 0$ for $i \neq j$. We readily confirm that the set of two linear function $\{\tilde \phi_1,\tilde \phi_2\}$ that satisfies the interpolation condition~\eqref{eq:fe_line_interp} is a basis for $\PP^1(\tilde I)$.

While the linear Lagrange shape functions can be found by inspection, we here follow a more systematic procedure that generalizes to higher dimensions and higher-order polynomials. To find the basis, we first express the shape functions in terms of the monomial basis $\{1,\tilde x\}$:
\begin{equation}
  \tilde \phi_j(x) = c^{(j)}_1 + c^{(j)}_2 \tilde x,  \quad j = 1, 2.
  \label{eq:fe_lin_line_rep}
\end{equation}
We now apply the interpolation condition~\eqref{eq:fe_line_interp} to find the coefficients.  For instance, $\tilde \phi_1$ must satisfy
\begin{equation*}
  \bmat{cc}
  1 & \tilde z_1 \\
  1 & \tilde z_2 \\
  \emat
  \bmat{c}
  c^{(1)}_1 \\ c^{(1)}_2
  \emat
  =
  \bmat{c}
  1 \\ 0 
  \emat
\end{equation*}
We can also pose a single matrix equation for the monomial coefficients of all three shape functions: 
\begin{equation*}
  \underbrace{ \bmat{cc}
  1 & \tilde z_1 \\
  1 & \tilde z_2 \\
  \emat }_{V}
  \underbrace{ 
  \bmat{cc}
  c_1^{(1)} & c_1^{(2)} \\
  c_2^{(1)} & c_2^{(2)} \\
  \emat }_{C}
  =
  \bmat{cc}
  1 & 0 \\
  0 & 1 \\
  \emat.
\end{equation*}
We note that the matrix $V$ is the \emph{Vandermonde matrix} associated with our monomial basis $\{ 1,\tilde x\}$ evaluated at the Lagrange interpolation points $\{\tilde z_1, \tilde z_2\}$.  The matrix $V$ is non-singular as long as the interpolation points are distinct, which is the case for our line segment. The unique coefficients are given by
\begin{equation*}
  C = V^{-1} = \bmat{cc} 1 & 0 \\ -1 & 1 \emat
\end{equation*}
and the associated shape functions are
\begin{align*}
  \tilde \phi_1(\tilde x) &= 1 - \tilde x_1 , \\
  \tilde \phi_2(\tilde x) &= \tilde x_2.
\end{align*}
The shape functions are shown in Figure~\ref{fig:fe_shape_line_p1}.

\begin{figure}
  \centering
  \subfigure[$\tilde \phi_1$]{
    \includegraphics[width=0.3\textwidth]{shape_line_p1_1}
  }
  \subfigure[$\tilde \phi_2$]{
    \includegraphics[width=0.3\textwidth]{shape_line_p1_2}
  }
  \caption{Linear Lagrange shape functions on the line segment $\tilde I$. \label{fig:fe_shape_line_p1}}
\end{figure}

Once we find the coefficients of the shape functions, we can evaluate the value of the functions at any point over $\tilde I \subset \RR^1$ by evaluating~\eqref{eq:fe_lin_line_rep}. We can also differentiate~\eqref{eq:fe_lin_line_rep} to obtain the gradient of the shape functions:
\begin{align*}
  \left. \pp{\tilde \phi_i}{\tilde x} \right|_{\tilde x} = c_2^{(i)}.
\end{align*}
More explicitly,
\begin{align*}
  \left. \dd{\tilde \phi_1}{\tilde x} \right|_{\tilde x} = -1
  \quad \text{and} \quad 
  \left. \dd{\tilde \phi_2}{\tilde x} \right|_{\tilde x} = 1.
\end{align*}
The derivatives are constant over the element because the shape functions are linear.

Before we introduce other finite elements, we use the linear Lagrange element as an example to describe three properties that formally defines a \emph{finite element}:
\begin{enumerate}
\item the domain over which the element is defined; e.g., the reference line segment $\tilde I$.
\item the finite-dimensional linear space of functions; e.g., the linear polynomial space $\PP^1(\tilde I)$.
\item the degree of freedom used to describe functions; e.g., the values of the function at the nodes $\{ \tilde z_1, \tilde z_2 \}$.
\end{enumerate}



\subsection{Linear Lagrange finite element on a triangle}
\label{sec:fe_lin_tri}

We next introduce a linear Lagrange element on the reference triangle $\tilde T \subset \RR^2$.  Linear functions in $\RR^2$ takes on the form $a_1 + a_2 \tilde x_1 + a_3 \tilde x_2$ and has three degrees of freedom; we hence need to identify a linear independent set of three linear functions.  In our case, we wish to identify a set of three linear \emph{Lagrange basis functions} for the space.  We choose for our interpolation nodes the three vertices of the triangle
\begin{equation*}
  \tilde z_1 \equiv (0,0),
  \quad \tilde z_2 = (1,0),
  \quad \text{and} \quad \tilde z_3 = (0,1),
\end{equation*}
as shown in Figure~\ref{fig:fe_ref_tri_p1}. Our shape functions are linear functions $\{ \tilde \phi_1, \tilde \phi_2, \tilde \phi_3 \}$ that satisfy the interpolation condition
\begin{equation}
  \tilde \phi_i(\tilde z_j) = \delta_{ij} \quad i,j = 1,\dots,3,
   \label{eq:fe_interp_tri}
\end{equation}
where $\delta_{ij}$ is the Kronecker delta.

\begin{figure}
  \centering
  \includegraphics[width=0.3\textwidth]{ref_tri_p1}
  \caption{Linear Lagrange finite element on the reference triangle.}
  \label{fig:fe_ref_tri_p1}
\end{figure}

We identify the shape functions using the same procedure used to identify the linear Lagrange shape functions on the unit line segment in Section~\ref{sec:fe_lin_line}. We first express the shape functions in terms of the monomial basis $\{ 1, \tilde x_1, \tilde x_2 \}$:
\begin{equation}
  \tilde \phi_j(\tilde x) = c^{(j)}_1 + c^{(j)}_2 \tilde x_1 + c_3^{(j)} \tilde x_2 \quad j = 1, 2, 3.
  \label{eq:fe_lin_tri_rep}
\end{equation}
We then apply the interpolation condition~\eqref{eq:fe_interp_tri} to find the coefficients: 
\begin{equation*}
  \underbrace{
   \bmat{ccc}
  1 & \tilde z_{1,1} & \tilde z_{1,2} \\
  1 & \tilde z_{2,1} & \tilde z_{2,2} \\
  1 & \tilde z_{3,1} & \tilde z_{3,2} \\
  \emat }_{\equiv V}
  \underbrace{ 
  \bmat{ccc}
  c_1^{(1)} & c_1^{(2)} & c_1^{(3)} \\
  c_2^{(1)} & c_2^{(2)} & c_2^{(3)} \\
  c_3^{(1)} & c_3^{(2)} & c_3^{(3)} \\
  \emat
  }_{\equiv C}
  =
  \bmat{ccc}
  1 & 0 & 0 \\
  0 & 1 & 0 \\
  0 & 0 & 1
  \emat,
\end{equation*}
where $\tilde z_{i,j}$ is the $j$-th coordinate of the $i$-th interpolation node. The Vandermonde matrix $V$ is non-singular as long as the interpolation points are not collinear, which is equivalent to the condition that the triangle have a finite area; the condition is obviously satisfied for our reference triangle $\tilde T$. The coefficients are given by 
\begin{equation*}
  C = V^{-1} \bmat{ccc}
  1 & 0 & 0\\
  -1 & 1 & 0 \\
  -1 & 0 & 1
  \emat;
\end{equation*}
the associated shape functions are
\begin{align}
  \tilde \phi_1(\tilde x) &= 1 - \tilde x_1 - \tilde x_2 \notag \\
  \tilde \phi_2(\tilde x) &= \tilde x_1 \label{eq:fe_lin_tri_expl} \\
  \tilde \phi_3(\tilde x) &= \tilde x_2. \notag
\end{align}
Figure~\ref{fig:fe_shape_tri_p1} visualizes the three basis functions.

\begin{figure}
  \centering
  \subfigure[$\tilde \phi_1$]{
    \includegraphics[width=0.3\textwidth]{shape_tri_p1_1}
  }
  \subfigure[$\tilde \phi_2$]{
    \includegraphics[width=0.3\textwidth]{shape_tri_p1_2}
  }
  \subfigure[$\tilde \phi_3$]{
    \includegraphics[width=0.3\textwidth]{shape_tri_p1_3}
  }
  \caption{Linear Lagrange shape functions on the reference triangle $\tilde K$.}
  \label{fig:fe_shape_tri_p1}
\end{figure}
We can also differentiate~\eqref{eq:fe_lin_tri_rep} to obtain the gradient of the shape functions:
\begin{equation*}
  \left. \pp{\tilde \phi_j}{\tilde x_1} \right|_{\tilde x} = c_2^{(j)} \quad \text{and} \quad  \left. \pp{\tilde \phi_j}{\tilde x_2} \right|_{\tilde x} = c_3^{(j)}, \quad j = 1,2,3.
\end{equation*}
More explicitly,
\begin{align*}
  \left. \pp{\tilde \phi_1}{\tilde x_1} \right|_{\tilde x} = -1
  \quad &\text{and} \quad
  \left. \pp{\tilde \phi_1}{\tilde x_2} \right|_{\tilde x} = -1\\
  \left. \pp{\tilde \phi_2}{\tilde x_1} \right|_{\tilde x} = 1
  \quad &\text{and} \quad
  \left. \pp{\tilde \phi_2}{\tilde x_2} \right|_{\tilde x} = 0\\
  \left. \pp{\tilde \phi_3}{\tilde x_1} \right|_{\tilde x} = 0
  \quad &\text{and} \quad
  \left. \pp{\tilde \phi_3}{\tilde x_2} \right|_{\tilde x} = 1.
\end{align*}
For the linear Lagrange element, the derivatives are constant (and trivially spans $\PP^0(\tilde T)$). 

We now relate the trace of the shape functions on facets $F_i$, $i = 1,2,3$, to the shape functions on a line segment. To this end, we first note that any restriction of a two-dimensional linear function to a line segment is a linear function on the line segment.  We apply this principle to map the linear shape functions $\{\tilde \phi_j\}_{j=1}^3$ defined on a triangle $\tilde T \subset \RR^2$ to the linear shape functions $\{ \chi_k \}_{k=1}^2$ defined on a line segment $I \subset \RR^1$.

The mapping of the Lagrange nodes on the triangle $\tilde T$ to those on the facets $\{ \tilde F_i \}_{i=1}^3$ is shown in Figure~\ref{fig:fe_ref_tri_face_p1}. For instance, the one-dimensional linear shape functions on the facet $\tilde F_1$, $\{ \chi_k^{\tilde F_1} \}_{k=1}^2$, are identified by the Lagrange nodes
\begin{equation*}
  \tilde z_1^{\tilde F_1} \equiv \tilde z_2 \quad \text{and} \quad
  \tilde z_2^{\tilde F_1} \equiv \tilde z_3 ,
\end{equation*}
and the shape functions are related by
\begin{equation*}
  \tilde \phi_2 |_{\tilde F_1} = \chi_1^{\tilde F_1}
  \quad \text{and} \quad
  \tilde \phi_3 |_{\tilde F_1} = \chi_2^{\tilde F_1} .
\end{equation*}
Analogous relationships exist for the facets $\tilde F_2$ and $\tilde F_3$. 
%% Similarly, on the facet $\tilde F_2$, the Lagrange nodes are
%% \begin{equation*}
%%   \tilde z_1^{\tilde F_2} \equiv \tilde z_3 \quad \text{and} \quad
%%   \tilde z_2^{\tilde F_2} \equiv \tilde z_1 ,
%% \end{equation*}
%% and the shape functions are related by
%% \begin{equation*}
%%   \tilde \phi_3 |_{\tilde F_2} = \chi_1^{\tilde F_2}
%%   \quad \text{and} \quad
%%   \tilde \phi_1 |_{\tilde F_2} = \chi_2^{\tilde F_2} ;
%% \end{equation*}
More compactly, we can introduce a facet-to-element node map
\begin{equation*}
  \theta_{\rm f-e} : \{\tilde F_1, \tilde F_2, \tilde F_3 \} \times \{1,2\} \to \{1,2,3\},
\end{equation*}
such that $j = \theta_{\rm f-e}(\tilde F_i,k)$ is the Lagrange node on $\tilde T$ identified with the $k$-th Lagrange node of the facet $\tilde F_i$. In other words,
\begin{equation*}
  \tilde z_k^{\tilde F_i} \equiv \tilde z_{\theta_{\rm f-e}(\tilde F_i,k)}, \quad k = 1,2, \ i = 1,2,3.
\end{equation*}
Accordingly, the trace of the two-dimensional shape functions $\{\tilde \phi_j\}_{j=1}^3$ are given related to the one-dimensional shape functions by
\begin{equation*}
  \tilde \phi_{\theta_{\rm f-e}(\tilde F_i,k)}|_{\tilde F_i} = \chi^{\tilde F_i}_k, \quad k = 1,2, \ i = 1,2,3.
\end{equation*}
These mappings we prove useful when the bilinear form or the linear form requires integration on a boundary.
  
\begin{figure}
  \centering
  \includegraphics[width=0.35\textwidth]{ref_tri_face_p1}
  \caption{Linear Lagrange triangular element-facet relationship.}
  \label{fig:fe_ref_tri_face_p1}
\end{figure}

To conclude this section, we note that our linear Lagrange finite element on a triangle is formally defined by (i) the domain --- the reference triangle $\tilde T$ ---, (ii) the linear function space --- the polynomial space $\PP^1(\tilde T)$ ---, and (iii) the degrees of freedom --- the values at the nodes $\{\tilde z_1,\tilde z_2, \tilde z_3\}$.

\subsection{Quadratic Lagrange finite element on a line segment}
\label{sec:fe_p2_line}
We now introduce a quadratic Lagrange element on the reference line segment $\tilde I \subset \RR^1$. A quadratic function in $\PP^1(\tilde I)$ takes on the form $a_1 + a_2 \tilde x + a_3 \tilde x^2$ and has three degrees of freedom; we hence wish to identify a linearly independent set of three quadratic Lagrange shape functions. To this end, we choose for our Lagrange interpolation nodes the two endpoints and the midpoint,
\begin{equation*}
  \tilde z_1 = 0, \quad \tilde z_2 = 1, \quad \text{and} \quad \tilde z_3 = 1/2. 
\end{equation*}
To find the Lagrange shape functions, we first express the functions in terms of the monomial basis $\{ 1, \tilde x, \tilde x^2 \}$:
\begin{equation}
  \tilde \phi_j(\tilde x) = c_1^{(j)} + c_2^{(j)} \tilde x + c_3^{(j)} \tilde x^2, \quad j = 1,\dots, 3.
  \label{eq:fe_quad_line_rep}
\end{equation}
We then express the interpolation condition $\tilde \phi_j(\tilde z_i) = \delta_{ij}$ as a $3 \times 3$ system $VC = I$, where $C \in \RR^{3 \times 3}$ is the coefficient matrix so that $C_{ij} = c_i^{(j)}$ and the $i$-th row of the Vandermonde matrix $V \in \RR^{3 \times 3}$ is
\begin{equation*}
  V_{i:} = \bmat{ccc} 1 & \tilde z_i & \tilde z_i^2\emat.
\end{equation*}
The matrix $V$ is non-singular because the monomial basis functions are linearly independent and the three interpolation points are distinct. The shape fucntions are shown in Figure~\ref{fig:fe_shape_line_p2}. 
\begin{figure}
  \centering
  \subfigure[$\tilde \phi_1$]{
    \includegraphics[width=0.3\textwidth]{./shape_line_p2_1}
  }
  \subfigure[$\tilde \phi_2$]{
    \includegraphics[width=0.3\textwidth]{./shape_line_p2_2}
  }
  \subfigure[$\tilde \phi_3$]{
    \includegraphics[width=0.3\textwidth]{./shape_line_p2_3}
  }
  \caption{Quadratic Lagrange shape functions on line segment $\tilde I$. \label{fig:fe_shape_line_p2}}
\end{figure}
The differentiation of \ref{eq:fe_quad_line_rep} yields the derivatives of the shape functions:
\begin{equation*}
  \left. \dd{\tilde \phi_j}{\tilde x} \right|_{\tilde x} = c_2^{(j)} + 2 c_3^{(j)} \tilde x , \quad j = 1,\dots,3.
\end{equation*}
Because the shape functions are quadratic, the derivatives vary linear over the domain $\tilde I$, and they span $\PP^1(\tilde I)$.

\subsection{Quadratic Lagrange finite element on a triangle}
\label{sec:fe_p2_tri}
We now introduce a quadratic Lagrange finite element on the reference triangle $\tilde T \subset \RR^2$. A quadratic function in $\PP^2(\tilde K)$ takes on the form $a_1 + a_2 \tilde x_1 + a_3 \tilde x_2 + a_4 \tilde x_1^2 + a_5 \tilde x_1 \tilde x_2 + a_6 \tilde x_2^2$ and has six degrees of freedom; we hence wish to identify a linearly independently set of six quadratic Lagrange shape functions.  To this end, we choose for our Lagrange interpolation nodes the three vertices of the triangle and three points at the middle of the edges,
\begin{equation*}
  \tilde z_1 = (0,0), \quad \tilde z_2 = (1,0), \quad \tilde z_3 = (0,1), \quad \tilde z_4 = (1/2,1/2), \quad \tilde z_5 = (0,1/2), \quad \tilde z_6 = (1/2,0),
\end{equation*}
as shown in Figure~\ref{fig:fe_ref_tri_p2}.
\begin{figure}
  \centering
  \includegraphics[width=0.3\textwidth]{ref_tri_p2}
  \caption{Quadratic Lagrange finite element on the reference triangle.}
  \label{fig:fe_ref_tri_p2}
\end{figure}
The ordering of the quadratic nodes is not universal in the finite element literature; we here adhere the convention that, for $i \in \{4,5,6\}$, the $i$-th node is on the midpoint of the $(i-3)$-th edge of the reference triangle. To find the Lagrange shape functions, we first express the basis functions in terms of the monomial basis $\{1,\tilde x_1, \tilde x_2, \tilde x_1^2 , \tilde x_1 \tilde x_2, \tilde x_2^2\}$:
\begin{equation}
  \tilde \phi_j(\tilde x) = c_1^{(j)} + c_2^{(j)} \tilde x_1 + c_3^{(j)} \tilde x_2 + c_4^{(j)} \tilde x_1^2 + c_5^{(j)} \tilde x_1 \tilde x_2 + c_6^{(j)} \tilde x_2^2 \quad j = 1,\dots,6.
  \label{eq:fe_quad_tri_rep}
\end{equation}
We then express the interpolation condition $\tilde \phi_j(\tilde z_i) = \delta_{ij}$ as a $6 \times 6$ matrix system $VC = I$, where $C \in \RR^{6 \times 6}$ is the coefficient matrix so that $C_{ij} = c^{(j)}_i$ and the $i$-th row of the Vandermonde matrix $V \in \RR^{6 \times 6}$ is
\begin{equation*}
  V_{i:} = \bmat{cccccc} 1 & \tilde z_{i,1} & \tilde z_{i,2} &  (z_{i,1})^2 & \tilde z_{i,1} \tilde z_{i,2} & (\tilde z_{i,2})^2 \emat .
\end{equation*}
The matrix $V$ is non-singular, and the linear system has a unique solution: $C = V^{-1}$.  Figure~\ref{fig:fe_shape_tri_p2} shows the six basis functions.
\begin{figure}
  \centering
  \subfigure[$\tilde \phi_1$]{
    \includegraphics[width=0.3\textwidth]{shape_tri_p2_1}
  }
  \subfigure[$\tilde \phi_2$]{
    \includegraphics[width=0.3\textwidth]{shape_tri_p2_2}
  }
  \subfigure[$\tilde \phi_3$]{
    \includegraphics[width=0.3\textwidth]{shape_tri_p2_3}
  }
  \subfigure[$\tilde \phi_4$]{
    \includegraphics[width=0.3\textwidth]{shape_tri_p2_4}
  }
  \subfigure[$\tilde \phi_5$]{
    \includegraphics[width=0.3\textwidth]{shape_tri_p2_5}
  }
  \subfigure[$\tilde \phi_6$]{
    \includegraphics[width=0.3\textwidth]{shape_tri_p2_6}
  }
  \caption{Quadratic Lagrange shape functions on the reference triangle.}
  \label{fig:fe_shape_tri_p2}
\end{figure}
 The differentiation of~\eqref{eq:fe_quad_tri_rep} yields the gradient of the shape functions,
\begin{align*}
  \left. \pp{\tilde \phi_j}{\tilde x_1} \right|_{\tilde x} &= c^{(j)}_2 + 2 c^{(j)}_4 \tilde x^2_1 + c^{(j)}_5 \tilde x_2
  \\
  \left. \pp{\tilde \phi_j}{\tilde x_2} \right|_{\tilde x}&= c^{(j)}_3 + c^{(j)}_5 \tilde x_1 + 2 c^{(j)}_6 \tilde x_2.
\end{align*}
For the quadratic Lagrange element, the derivatives are linear functions, and they span $\PP^1(\tilde K)$.


As we have done for the linear Lagrange element in Section~\ref{sec:fe_p1_tri}, we now relate the trace of the shape functions on facets $F_i$, $i = 1,2,3$, to the shape functions on a line segment. We first note that any restriction of a two-dimensional quadratic function to a line segment is a quadratic function on the line segment.  We apply this principle to map the quadratic shape functions $\{\tilde \phi_j\}_{j=1}^6$ defined on a triangle $\tilde T \subset \RR^2$ to the quadratic shape functions $\{ \chi_k \}_{k=1}^3$ defined on a line segment $I \subset \RR^1$.

  
\begin{figure}
  \centering
  \includegraphics[width=0.35\textwidth]{ref_tri_face_p2}
  \caption{Quadratic Lagrange triangular element-facet relationship.}
  \label{fig:fe_ref_tri_face_p2}
\end{figure}


The mapping of the Lagrange nodes on the triangle $\tilde T$ to those on the facets $\{ \tilde F_i \}_{i=1}^3$ is shown in Figure~\ref{fig:fe_ref_tri_face_p2}. We note that the one-dimensional quadratic shape functions on the facet $\tilde F_1$, $\{ \chi_k^{\tilde F_1} \}_{k=1}^3$, are identified by the Lagrange nodes
\begin{equation*}
  \tilde z_1^{\tilde F_1} \equiv \tilde z_2 ,
  \tilde z_2^{\tilde F_1} \equiv \tilde z_3 ,
  \quad \text{and} \quad
  \tilde z_3^{\tilde F_1} \equiv \tilde z_4 ,
\end{equation*}
and the shape functions are related by
\begin{equation*}
  \tilde \phi_2 |_{\tilde F_1} = \chi_1^{\tilde F_1} ,
  \tilde \phi_3 |_{\tilde F_1} = \chi_2^{\tilde F_1} ,
  \quad \text{and} \quad
  \tilde \phi_4 |_{\tilde F_1} = \chi_3^{\tilde F_1} .
\end{equation*}
Analogous relationships exist for the facets $\tilde F_2$ and $\tilde F_3$. 
%% Similarly, on the facet $\tilde F_2$, the Lagrange nodes are
%% \begin{equation*}
%%   \tilde z_1^{\tilde F_2} \equiv \tilde z_3 \quad \text{and} \quad
%%   \tilde z_2^{\tilde F_2} \equiv \tilde z_1 ,
%% \end{equation*}
%% and the shape functions are related by
%% \begin{equation*}
%%   \tilde \phi_3 |_{\tilde F_2} = \chi_1^{\tilde F_2}
%%   \quad \text{and} \quad
%%   \tilde \phi_1 |_{\tilde F_2} = \chi_2^{\tilde F_2} ;
%% \end{equation*}
More compactly, we can introduce a facet-to-element node map
\begin{equation*}
  \theta_{\rm f-e} : \{\tilde F_1, \tilde F_2, \tilde F_3 \} \times \{1,2,3\} \to \{1,\dots,6\},
\end{equation*}
such that $j = \theta_{\rm f-e}(\tilde F_i,k)$ is the Lagrange node on $\tilde T$ identified with the $k$-th Lagrange node of the facet $\tilde F_i$. In other words,
\begin{equation*}
  \tilde z_k^{\tilde F_i} \equiv \tilde z_{\theta_{\rm f-e}(\tilde F_i,k)}, \quad k = 1,2, \ i = 1,2,3.
\end{equation*}
Accordingly, the trace of the two-dimensional shape functions $\{\tilde \phi_j\}_{j=1}^3$ are given related to the one-dimensional shape functions by
\begin{equation*}
  \tilde \phi_{\theta_{\rm f-e}(\tilde F_i,k)}|_{\tilde F_i} = \chi^{\tilde F_i}_k, \quad k = 1,2, \ i = 1,2,3.
\end{equation*}


\subsection{Generalization: an advanced method for arbitrary domain and polynomial degree using Legendre polynomials}
\label{sec:fe_gen_shape}
We can generalize the procedure to generate Lagrange shape functions of an arbitrary degree on an arbitrary domain $\tilde K$.  Say we wish to generate Lagrange shape functions for a polynomial space of degree $p$ with a dimension $n_s$ and the interpolation nodes $\{ \tilde z_i \}_{i=1}^{n_s}$.  We first identify \emph{any} basis $\{ \tilde \psi_k \}_{k=1}^{n_s}$.  We then express the Lagrange shape functions as
\begin{equation}
  \tilde \phi_j(\tilde x) = \sum_{j=1}^{n_s} c^{(j)}_k \psi_k(\tilde x), \quad \tilde x \in \tilde K, \quad j = 1,\dots,n_s.
  \label{eq:fe_gen_val_rep}
\end{equation}
The coefficients that satisfy the interpolation condition $\tilde \phi_j(\tilde z_i) = \delta_{ij}$ must satisfy the $n_s \times n_s$ system
\begin{align}
  \bmat{ccc}
  \psi_1(\tilde z_1) & \cdots & \psi_{n_s}(\tilde z_1) \\
  \vdots & \ddots & \vdots \\
  \psi_1(\tilde z_{n_s}) & \cdots & \psi_{n_s}(\tilde z_{n_s}) 
  \emat
  \bmat{ccc}
  c^{(1)}_1 & \cdots & c^{(n_s)}_1 \\
  \vdots & \ddots & \vdots \\
  c^{(1)}_{n_s} & \cdots & c^{(n_s)}_{n_s}
  \emat
  =
  I_{n_s},
  \label{eq:fe_gen_coeff_sys}
\end{align}
where $I_{n_s}$ is the $n_s \times n_s$ identity matrix. The derivative of the shape functions are then given by
\begin{equation}
  \left. \pp{\tilde \phi_j}{\tilde x_i} \right|_{\tilde x}
  = \sum_{k=1}^{n_s} c^{(j)}_k \left. \pp{\tilde \psi_k}{\tilde x_i} \right|_{\tilde x}, \quad \tilde x \in \tilde K.
  \label{eq:fe_gen_der_rep}
\end{equation}
Note that this is a generalization, or simply an abstraction, of the method we have discussed for linear and quadratic shape functions on a line segment and triangle discussed in the previous sections.

The efficient implementation of this procedure relies on the efficient evaluation of the values and gradients of the basis functions $\{\tilde \psi_k \}_{k=1}^{n_s}$. One convenient choice for $\{\tilde \psi_k \}_{k=1}^{n_s}$ are the \emph{Legendre polynomials}, which is a set of orthogonal polynomials in $L^2(\tilde I)$. We provide a formal definition.
\begin{definition}[Legendre polynomial]
  \label{def:fe_legendre_poly}
  The Legendre polynomials $\{ \psi_i \}_{i=0}^{n}$ are hierarchical polynomials defined on a unit line segment $\tilde I \equiv (0,1)$ such that
  \begin{itemize}
  \item[(i)] $\text{span}\{ \psi_i \}_{i=0}^n = \PP^{n}(\tilde I)$  
  \item[(ii)] $(\psi_i,\psi_j)_{L^2(\Omega)} \equiv \int_{\tilde I} \psi_i \psi_j dx =  \delta_{ij}$, $\forall i,j = 1,\dots,n$. 
  \end{itemize}
\end{definition}
The orthogonality condition (ii) implies that the set $\{ \psi_i \}_{i=0}^{n}$ is linearly independent and hence is a basis for $\PP^n(\tilde I)$. In addition the Legendre polynomials of arbitrary degree and their respective derivatives can be evaluated using recurrence relations. We can hence use~\eqref{eq:fe_gen_val_rep} and \eqref{eq:fe_gen_der_rep} with the Legendre polynomials as the underlying basis.  In higher dimensions, we may use the tensor product of Legendre polynomials as the underlying basis.  The use of the Legendre polynomial as the underlying polynomial also ensures the linear system~\eqref{eq:fe_gen_coeff_sys} remains well-posed even if the polynomial degree is very high; the Vandermonde matrix associated with the monomials become ill-conditioned for a high-degree polynomials as the monomials becomes nearly linearly dependent.

%% \subsection{Computer implementation}
%% The procedure described in this section can be easily and efficiently implemented in any programming language. We here describe the procedure using a MATLAB-like language. The ingredients of the code are
%% \begin{itemize}
%% \item \texttt{interp\_nodes}.  The function returns an array of interpolation nodes $Z \in \RR^{n_s \times d}$ so that $Z_{ij} = \tilde z_{i,j}$ for a given reference domain and polynomial degree.
%% \item \texttt{monomial}. The input to the function is an array of evaluation points $X \in \RR^{n \times d}$ so that $X_{ij}$ is the $j$-th coordinate of the $i$-th evaluation. The function returns a matrix of function values $\Psi(X) \in \RR^{n \times n_s}$ so that $\Psi_{ij}(X) = \psi_j(x_i)$ and a order-3 tensor of derivative values $\nabla\Psi(X) \in \RR^{n \times n_s \times d}$ so that $(\nabla \Psi(X))_{ijk} = \left. \pp{\psi_j}{x_k} \right|_{x_i}$.  Here, $\psi_j$ is the $j$-th monomial basis function.  More generally $\{ \psi_j \}_{j=1}^{n_s}$ is underlying basis from which shape functions $\{ \phi_j \}_{j=1}^{n_s}$ are constructed; they could be, for example, (a tensor-product of) Legendre polynomials.
%% \end{itemize}
%% From these two functions, we can readily evaluate the values and gradients of Lagrange shape functions.  Our Vandermonde matrix $V \in \RR^{n_s \times n_s}$ associated with interpolation nodes $Z \in \RR^{n_s \times d}$ is given by
%% \begin{equation*}
%%   V = \Psi(Z) \quad (\text{in } \RR^{n_s \times n_s}).
%% \end{equation*}m
%% The Lagrange shape functions evaluated at the $n$ points associated with $X \in \RR^{n \times d}$ is given by $\Phi(X) \in \RR^{n \times n_s}$ such that
%% \begin{equation*}
%%   \Phi(X) = \Psi(X) V^{-1} \quad (\text{in } \RR^{n \times n_s}).
%% \end{equation*}
%% Here, $(\Phi(X))_{ij} = \phi_j(x_i)$, and we invoke the fact that the Vandermonde matrix $V$ is the inverse of the coefficient matrix $C$. Similarly, the derivative of the Lagrange shape functions 

  \section{Physical elements}
  \label{sec:fe_phy_elem}
\subsection{Isoparametric mapping}
\label{sec:fe_iso_map}
We have so far introduced shape functions $\{\tilde \phi_i\}_{i=1}^{n_s}$ defined on a reference element $\tilde K$, where the reference element may be the reference line segment $\tilde I$ or the reference triangle $\tilde T$. We now wish to construct shape functions $\{\phi_i\}_{i=1}^n$ which span the approximation space $\calV_h \subset \calV$.  To clearly distinguish between quantities defined on the reference domain and those defined on the actual physical domain, we will qualify the latter quantities with the adjective \emph{physical}. (Note that the quantities associated with the physical space, unlike those associated with teh reference space, do not bear tilde ($\tilde \cdot$).)

To begin, we create a mapping from a point $\tilde x$ in the reference element $\tilde K \equiv \tilde T$ to a point $x$ in the physical element $K$.  The physical element is delineated by $n_s$ nodes, $\{ z^K_\alpha \}_{\alpha=1}^{n_s}$.  To map a point $\tilde x \in \tilde K$ in the reference domain to a point $x \in K$ in the physical domain, we employ an \emph{isoparametric mapping}, $\calG^K: \tilde K \to K \subset \RR^2$ given by 
\begin{equation}
  x = \calG^K(\tilde x) \equiv \sum_{\alpha=1}^{n_s} z^K_\alpha \tilde \phi_\alpha(\tilde x) ,
  \label{eq:fe_iso_map}
\end{equation}
where $z^K_\alpha \in \RR^d$ is the coordinate of the $\alpha$-th node of the physical element $K$, $\tilde \phi_\alpha \in \PP^p(\tilde K)$ is the Lagrange shape function associated with the $\alpha$-th node of the reference element, and $n_s$ is the number of the shape functions.  The isoparametric mapping $\calG^K: \tilde K \to K$ is a unique map that (i) is a polynomial map of degree $p$ and (ii) maps the Lagrange interpolation points $\{ \tilde z_\alpha \}_{\alpha}^{n_s}$ of the reference element $\tilde K$ to the respective Lagrange interpolation points $\{ z_\alpha \}_{\alpha}^{n_s}$ of the physical element $K$. The isoparametric mapping $\calG^K: \tilde K \to K$ is invertible under a reasonable condition; we will discuss the condition in detail shortly.

An example of $\PP^1$ isoparametric mapping of a triangle is shown in Figure~\ref{fig:fe_impl_isomap_p1}.  The physical domain $K_8$ is defined by the physical nodes $\{ z_1^{K_8}, z_2^{K_8}, z_3^{K_8} \}$ which are in turn identified by the three nodes of the mesh $\{ z_2, z_3, z_1 \}$.  Because the reference element is linear, the mapping $\tilde G^{K_8}: \tilde K \to K_8$ is affine.

\begin{figure}
  \centering
  \subfigure[triangulation]{
    \includegraphics[width=0.3\textwidth]{fe_mesh_p1}
    \label{fig:fe_impl_mesh_p1}
  }
  \subfigure[reference domain $\tilde K \equiv \tilde T$]{
    \includegraphics[width=0.3\textwidth]{isomap_p1_ref}
    %\label{fig:fe_impl_isomap_p1_ref}
  }
  \subfigure[physical domain $K_8$]{
    \includegraphics[width=0.33\textwidth]{isomap_p1_phy}
    %\label{fig:fe_impl_isomap_p1_phy}
  }
  \caption{$\PP^1$ isoparametric mapping. \label{fig:fe_impl_isomap_p1}}
\end{figure}

As an another example, we provide a $\PP^2$ isoparametric mapping of a triangle in Figure~\ref{fig:fe_impl_isomap_p2}. The physical domain $K_8$ is defined by the physical nodes $\{ z_1^{K_8}, z_2^{K_8}, z_3^{K_8}, z_4^{K_8}, z_5^{K_8}, z_6^{K_8} \}$ which are in turn identified by the six nodes of the mesh $\{ z_2, z_3, z_1, z_{11}, z_{10}, z_{13} \}$.  Because the reference element is quadratic, the mapping $\calG^{K_8}: \tilde K \to K_8$ is also quadratic (plus the constant shift).  Note that we can represent curved geometries using $\PP^{p>1}$ isoparametric mapping; as we will see later, the accurate representation of the curved geometry is important to realize higher-order approximation of boundary value problems. 

\begin{figure}
  \centering
  \subfigure[triangulation]{
    \includegraphics[width=0.3\textwidth]{fe_mesh_p2}
    \label{fig:fe_impl_mesh_p2}
  }
  \subfigure[reference domain $\tilde K \equiv \tilde T$]{
    \includegraphics[width=0.3\textwidth]{isomap_p2_ref}
    %\label{fig:fe_impl_isomap_p1_ref}
  }
  \subfigure[physical domain $K_8$]{
    \includegraphics[width=0.33\textwidth]{isomap_p2_phy}
    %\label{fig:fe_impl_isomap_p1_phy}
  }
  \caption{$\PP^2$ isoparametric mapping. \label{fig:fe_impl_isomap_p2}}
\end{figure}

We now introduce a few quantities derived from the isoparametric mapping.  We can differentiate the mapping~\eqref{eq:fe_iso_map} to evaluate the Jacobian of the mapping from $\tilde K$ to $K$, $J^K: \tilde K \to \RR^{d \times d}$, given by
\begin{equation*}
  J^K_{ij}(\tilde x)
  \equiv \left. \pp{x_i}{\tilde x_j} \right|_{\tilde x}
  = \left. \pp{\calG^K_i}{\tilde x_j} \right|_{\tilde x}
  = \sum_{\alpha= 1}^{n_s} z^K_\alpha \left. \pp{\tilde \phi_\alpha}{\tilde x_j} \right|_{\tilde x}.
\end{equation*}
As $\{\tilde \phi_\alpha\}_{\alpha=1}^{n_s}$ is a basis for $\PP^p(\tilde K)$, the Jacobian $J^K$ is a degree $p-1$ polynomial in $\tilde x$; consequently, for a $\PP^1(\tilde K)$ isoparametric mapping, the Jacobian is constant over $\tilde K$.

The determinant of the Jacobian $\text{det}(J^K(\tilde x))$ or, more compactly $|J^K(\tilde x)|$, relates the reference area $d \tilde x$ to the physical area $dx$ and is given by
\begin{equation*}
  dx = |J^K(\tilde x)| d\tilde x.
\end{equation*}
For a $\PP^1$ isoparametric mapping, $|J^K|$ is constant over $\tilde K$ because $J^K$ is constant. For a higher-order isoparametric mapping, $|J^K|$ is a polynomial function over $\tilde K$.

The isoparametric mapping~\eqref{eq:fe_iso_map} is invertible if and only if
\begin{equation}
  |J^K(\tilde x)| > 0 \quad \forall \tilde x \in \tilde K.
  \label{eq:fe_iso_map_invcond}
\end{equation}
For a $\PP^1$ isoparametric mapping, the condition \eqref{eq:fe_iso_map_invcond} is equivalent to the condition that (i) the area of the physical triangle is finite and (ii) the vertices are ordered in the counter-clock manner.  In fact, for a $\PP^1$ mapping, $|J^K|$ is constant and $|J^K|/2$ is the area of the physical triangle; the factor of $1/2$ is needed because the area of our reference triangle $\tilde T$ is $1/2$. For a higher-order isoparametric mapping that results in curved elements, the condition \eqref{eq:fe_iso_map_invcond} must be checked for all $\tilde x \in \tilde K$.

We can also compute the Jacobian associated with the inverse mapping, $(\tilde G^K)^{-1}: K \to \tilde K$, or \emph{inverse Jacobian}:
\begin{equation*}
  \left. \pp{\tilde x_i}{x_j} \right|_{\tilde x}
  = \left. \pp{((\tilde G^K)^{-1})_i}{x_j} \right|_{\tilde x}
  = ((J^K(\tilde x))^{-1})_{ij}.
\end{equation*}
The algebraic inverse of the Jacobian $\pp{x}{\tilde x}$ is the inverse Jacobian. The inverse Jacobian $\pp{x_i}{\tilde x_j}$ is well defined at $\tilde x \in \tilde K$ if and only if $|J^K(\tilde x)| > 0$.  For a $\PP^1$ isoparametric mapping, the inverse Jacobian is constant over $\tilde K$ because the Jacobian is constant.  For a higher-order isoparametric mapping, the inverse Jacobian is a non-polynomial function of $\tilde x$ because the inverse of a polynomial function is in general not a polynomial.

\subsection{Physical shape functions}
  \label{sec:fe_phy_shape}
Given an isoparametric mapping $\calG^K : \tilde K \ to K$, we now introduce physical shape functions $\{ \phi^K_\alpha \}_{\alpha=1}^{n_s}$ associated with the physical element $K \in \calT_h$.  We choose the basis functions that satisfy
\begin{equation}
  \phi^K_\alpha(x = \calG^K(\tilde x)) = \tilde \phi_\alpha(\tilde x) \quad \forall \tilde x \in \tilde K, \quad \alpha =  1, \dots, n_s,
  \label{eq:fe_impl_phiK}
\end{equation}
where $\tilde x \mapsto x$ is provided by the isoparametric mapping~\eqref{eq:fe_iso_map}. In words, the physical basis function $\phi^K_\alpha$ evaluated at the physical point $x(\tilde x) \in K$ takes on the same value as the associated reference basis function $\phi_\alpha$ evaluated at the associated reference point $\tilde x \in \tilde K$.

We can also differentiate~\eqref{eq:fe_impl_phiK} to obtain the derivative of the physical basis functions in the physical space: given any $\tilde x \in \tilde K$, 
\begin{equation}
  \left. \pp{\phi^K_\alpha}{x_i} \right|_{x = \calG^K(\tilde x)} =
  \sum_{j=1}^{d} \pp{\tilde x_j}{x_i} \left. \pp{\tilde \phi_\alpha}{\tilde x_j} \right|_{\tilde x} \left.  \right|_{\tilde x}, \quad i = 1,\dots,d, \ \alpha = 1,\dots,n_s.
  \label{eq:fe_impl_dphiK}
\end{equation}
The relationship may be expressed more explicitly using matrix-vector notation; for $d = 2$,
\begin{equation*}
  \bmat{c}
  \pp{\phi_\alpha^K}{x_1} \\ \pp{\phi_\alpha^K}{x_2}
  \emat_{\tilde x}
  =
  \bmat{cc}
  \pp{\tilde x_1}{x_1} & \pp{\tilde x_2}{x_1} \\
  \pp{\tilde x_1}{x_2} & \pp{\tilde x_2}{x_2}
  \emat_{\tilde x}
  \bmat{c}
  \pp{\tilde \phi_\alpha}{\tilde x_1} \\ \pp{\tilde \phi_\alpha}{\tilde x_2}
  \emat_{\tilde x}.
\end{equation*}
Or, noting that $\pp{\tilde x_j}{x_i} = ((J^K)^{-1})_{ji} = ((J^K)^{-T})_{ij}$, a more compact expression (for any $d$) is
\begin{equation}
  \nabla \phi_\alpha^K(x = \calG^K(\tilde x)) = (J^K)^{-T}(\tilde x) \tilde \nabla \tilde \phi(\tilde x).
  \label{eq:fe_impl_dphiK_2}
\end{equation}
Expressions~\eqref{eq:fe_impl_phiK} and \eqref{eq:fe_impl_dphiK} (or equivalently \eqref{eq:fe_impl_dphiK_2}) allows us to evaluate the value and gradient, respectively, of the physical basis function $\phi^K_\alpha$ at a physical point $x(\tilde x) \in K$ associated with a select reference point $\tilde x \in \tilde K$; we will soon see this is exactly the capability we need to evaluate stiffness matrices and load vectors.

As an example, consider the physical element $K_9$ in the triangulation shown in Figure~\ref{fig:fe_impl_mesh_p1} comprises linear elements.  The element $K_9$ is delineated by the nodes $\{ z^{K_9}_1 = z_4, z^{K_9}_2 = z_3, z^{K_9}_3 = z_2 \}$. The reference shape function $\tilde \phi_2 \in \PP^1(\tilde T)$ maps to the physical shape function $\phi_2^{K_9}$ as shown in Figure~\ref{fig:fe_impl_shape_map_p1}.  We in fact recognize that $\phi_2^{K_9}$ is the restriction of the physical (global) shape function $\phi_3$ associated with $z_3$; formally, $\phi_2^{K_9} \equiv \phi_3|_{K_9}$. %Because the shape function is linear and the geometry mapping is linear, the physical shape function is a linear function: $\phi_i^{K_9} \in \PP^1(K_9)$, $i = 1,\dots,3$.


%% As an example, consider the triangulation shown in Figure~\ref{fig:fe_impl_mesh_p1}.  The physical element $K_5 \in \calT_h$ is delineated by the nodes
%% \begin{equation*}
%%   z^{K_5}_1 = v_6 = (0.28,-0.07), \quad 
%%   z^{K_5}_2 = v_5 = (-0.21,0.98), \quad \text{and} \quad
%%   z^{K_5}_3 = v_3 = (-0.29,0.04);
%% \end{equation*}
%% we recall that the dot ($\bullet$) in Figure~\ref{fig:fe_mesh_p1} denotes the first vertex of the physical element, and vertices are in the counter-clockwise order. The third reference basis function $\tilde \phi_3$ over $\tilde K$ is shown in Figure~\ref{fig:fe_impl_loc_shape}.  The associated physical basis function $\phi_3^{K_5}$ over $K_5 \in \calT_h$ is shown in Figure~\ref{fig:fe_impl_glob_shape}.  Note that the $\phi_3^{K_5}$ is by definition defined over only $K_5 \subset \Omega$ and not the entire $\Omega$.

\begin{figure}
  \centering
  \subfigure[reference shape function $\tilde \phi_2$]{
    \includegraphics[width=0.3\textwidth]{shape_tri_p1_2}
    \label{fig:fe_impl_loc_shape_p1}
  }
  \subfigure[physical shape function $\phi^{K_9}_2$]{
    \includegraphics[width=0.33\textwidth]{shape_global_p1_part}
    \label{fig:fe_impl_glob_shape_p1}
  }
  \caption{Mapping of shape functions for a $\PP^1$ element.}
  \label{fig:fe_impl_shape_map_p1}
\end{figure}


As another example, consider the physical element $K_9$ in the triangulation shown in Figure~\ref{fig:fe_impl_mesh_p2} comprises quadratic elements. The element $K_9$ is delineated by the nodes $\{ z^{K_9}_1 = z_4, z^{K_9}_2 = z_3, z^{K_9}_3 = z_2,  z^{K_9}_4 = z_{13},  z^{K_9}_5 = z_{14},  z^{K_9}_6 = z_{15} \}$.  The reference shape function $\tilde \phi_2 \in \PP^2(\tilde T)$ maps to the physical shape function $\phi_2^{K_9}$ as shown in Figure~\ref{fig:fe_impl_shape_map_p2}.  We again recognize that $\phi_2^{K_9}$ is the restriction of the physical (global) shape function $\phi_3$ associated with $z_3$; formally, $\phi_2^{K_9} \equiv \phi_3|_{K_9}$.

\begin{figure}
  \centering
  \subfigure[reference shape function $\tilde \phi_2$]{
    \includegraphics[width=0.3\textwidth]{shape_tri_p2_2}
    \label{fig:fe_impl_loc_shape_p2}
  }
  \subfigure[physical shape function $\phi^{K_9}_2$]{
    \includegraphics[width=0.33\textwidth]{shape_global_p2_part}
    \label{fig:fe_impl_glob_shape_p2}
  }
  \caption{Mapping of shape functions for a $\PP^2$ element.}
  \label{fig:fe_impl_shape_map_p2}
\end{figure}

We make one remark about our physical basis functions defined by~\eqref{eq:fe_impl_phiK}.  Even though the reference basis function $\tilde \phi: \tilde K \to \RR$ is a polynomial in $\tilde K$, the physical basis function $\phi^K_\alpha: K \to \RR$ is in general a non-polynomial function in $K$. To see this, we observe that $\phi^K_\alpha(x) = \tilde \phi_\alpha((\calG^K)^{-1}(x))$, $\forall x \in K$; because the inverse map $(\calG^K)^{-1}: K \to \tilde K$ is not a polynomial in $K$ for a $\PP^{p>1}$ isoparametric mapping, the function $\phi^K_\alpha(\cdot) = \tilde \phi_\alpha((\calG^K)^{-1}(\cdot)) = \tilde \phi_\alpha \circ (\calG^K)^{-1}$ is not a polynomial in $K$.  Conversely, we note that $\phi^K_\alpha(\calG(\cdot)) = \phi^K_\alpha \circ \calG^K = \tilde \phi(\cdot)$ \emph{is} a polynomial in $\tilde K$. Hence, in the presence of curved elements, approximation space is not
\begin{equation*}
  \calV_h \equiv \{ v \in H^1(\Omega) \ | v|_K \in \PP^p(K), \ \forall K \in \calT_h \}
\end{equation*}
but rather
\begin{equation*}
  \calV_h \equiv \{ v \in H^1(\Omega) \ | \ v|_K \circ \calG^K \in \PP^p(\tilde K), \ \forall K \in \calT_h \}.
\end{equation*}
However, for a $\PP^1$ isoparametric mapping, the notation is simplified because the inverse mapping $(\calG^K)^{-1}: K \to \tilde K$ is affine and the shape function $\phi^K_\alpha(\cdot) = \tilde \phi_\alpha((\calG^K)^{-1}(\cdot))$ is a linear function in $K$.

%Using these basis functions, we can represent any function 
%given $\tilde x \in \tilde K$,
%\begin{equation*}
%  v(x(\tilde x))
  %= \tilde v(\tilde x)
%  = \sum_{\alpha = 1}^{n_s} \hat v_\alpha \phi^K_\alpha(x(\tilde x))
%\end{equation*}
%\begin{equation*}
%  \left. \pp{v}{x} \right|_{x(\tilde x)} 
  % = \left. \pp{\tilde v}{\tilde x} \right|_{\tilde x} \left. \pp{\tilde x}{x} \right|_{\tilde x}
%  = \sum_{\alpha = 1}^{n_s} \hat v_\alpha \left. \pp{\phi^K_\alpha}{x} \right|_{x(\tilde x)}
%\end{equation*}

\subsection{Isoparametric mapping for the facets}
In Section~\ref{sec:fe_iso_map}, we introduced an isoparametric mapping $\calG^K: \tilde K \to K$ from a reference element $\tilde K \subset \RR^d$ to $K \subset \RR^d$.  While this is the only mapping we need to evaluate bilinear and linear forms that only require integration over $\Omega$, forms that require integration over boundaries --- for example a Neumann boundary $\Gamma_N \subset \partial \Omega$ --- requires another mapping.  Specifically, in $d=2$ dimensions, we require an isoparametric mapping $\calG^F$ from a reference line segment $\tilde I \subset \RR^{1 \equiv d-1}$ to a physical facet $F \subset \RR^{2 \equiv d}$.  It is important to note that the mapping is from a $(d-1)$-dimensional reference element to a $d$-dimensional entity.

Figure~\ref{fig:fe_impl_isomap_face_p2} shows a concrete example of an isoparametric mapping from a reference line segment $\tilde I$ to a physical facet $F(K_8,2)$, the second facet (or edge) of the physical element $K_8$.  (We henceforth abbreviate the arguments of $F$ whenever they are unambiguous without an explicit notation.) We recall from Section~\ref{sec:fe_p2_tri} that the Lagrange nodes on the reference facet $\tilde F_2$ are related to those on the reference element $\tilde T$ by $\{ \tilde z_1^{\tilde F_2} \equiv \tilde z_3, \tilde z_2^{\tilde F_2} \equiv \tilde z_1, \tilde z_3^{\tilde F_2} \equiv \tilde z_5 \}$.  Accordingly, the Lagrange nodes on the physical facet $F(K_8,2)$ are related to those on the physical element $\{z^{K_8}_j\}_{j=1}^6$, and the (global) nodes $\{z_i\}_{i=1}^n$,  by $\{z_1^{F(K_8,2)} \equiv z^{K_8}_3 \equiv z_1, z_2^{F(K_8,2)} \equiv z^{K_8}_1 \equiv z_2, z_3^{F(K_8,2)} \equiv z^{K_8}_5 \equiv z_{10} \}$. 
\begin{figure}
  \centering
  \subfigure[triangulation]{
    \includegraphics[width=0.3\textwidth]{fe_mesh_p2}
  }
  \subfigure[reference line segment $\tilde K \equiv \tilde I$]{
    \includegraphics[width=0.3\textwidth]{isomap_p2_ref_edge}
    %\label{fig:fe_impl_isomap_p1_ref}
  }
  \subfigure[physical facet $F(K_8,2)$]{
    \includegraphics[width=0.33\textwidth]{isomap_p2_phy_edge}
    %\label{fig:fe_impl_isomap_p1_phy}
  }
  \caption{$\PP^2$ isoparametric mapping for a facet. \label{fig:fe_impl_isomap_face_p2}}
\end{figure}
Our isoparametric mapping for the facet, $\calG^F: \tilde I \to F$, is given by
\begin{equation}
  x = \calG^F(\tilde s) \equiv \sum_{\alpha=1}^{n_s^F} z^F_\alpha \tilde \chi_{\alpha}(\tilde s),
  \label{eq:fe_iso_map_face}
\end{equation}
where $z^F_\alpha \in \RR^d$ is the coordinate of the $\alpha$-th node of the physical facet $F \subset \RR^2$, $\chi_{\alpha} \in \PP^p(\tilde I)$ is the Lagrange shape function associated with the $\alpha$-th node of the reference line segment $\tilde I \subset \RR^1$.  Note that the input is $\tilde s \in \tilde I \subset \RR^{d-1}$ while the output is $x \in \Omega \subset \RR^d$.  Similar to the isoparametric mapping~\eqref{eq:fe_iso_map} for the element, the isoparametric mapping~\eqref{eq:fe_iso_map_face} is a polynomial map of degree $p$ that maps the Lagrange interpolation points of the reference entity to the physical entity.

We can differentiate~\eqref{eq:fe_iso_map_face} to obtain the Jacobian, $J^F: \tilde I \to F \subset \RR^{d \times (d-1)}$ such that
\begin{equation*}
  J^F_{ij}(\tilde s)
  \equiv
  \left. \pp{x_i}{\tilde s_j} \right|_{\tilde s}
  =
  \left. \pp{\calG^F}{\tilde s_j} \right|_{\tilde s}
  =
  \sum_{\alpha = 1}^{n_s^F} z^F_\alpha \left. \pp{\tilde \phi_\alpha}{\tilde s_j} \right|_{\tilde s},
  \quad i = 1,\dots,d, \ j = 1,\dots,d-1.
\end{equation*}
The Jacobian is rectangular because the mapping is from $\RR^{d-1}$ to $\RR^d$.

The physical length $ds$ is related to the reference length $d\tilde s$ by the relationship
\begin{equation*}
  ds = |J^F(\tilde s)|d\tilde s,
\end{equation*}
where, in two dimensions, 
\begin{equation*}
  |J^F(\tilde s)| \equiv \sqrt{J^F_{11}(\tilde s)^2 + J^F_{21}(\tilde s)^2}.
\end{equation*}
Analogously to the determinant condition $|J(\tilde x)| > 0$ $\forall \tilde x \in \tilde K$ for an element isoparametic mapping, a facet isoparametic mapping is valid if and only if
\begin{equation*}
  |J^F(\tilde s)| > 0 \quad \forall \tilde s \in \tilde I.
\end{equation*}
This condition is automatically satisfied if $|J^K(\tilde x)| >0$ $\forall \tilde x \in \overline{\tilde K}$ for the element $K$ associated with the facet $F$.

The evaluation of forms associated with a boundary value problem sometimes also require the evaluation of the \emph{unit outward-pointing normal vector}. In two dimension, the normal vector is given by
\begin{equation*}
  n(\tilde s) = \frac{1}{|J^F(\tilde s)|} \bmat{c} -J^F_{21}(\tilde s) \\ J^F_{11}(\tilde s) \emat .
\end{equation*}
This definition of the normal vector is for our counter-clockwise convention for the facet orientation; the sign needs to be reversed if the clockwise convection is used for the facet orientation.

\section{Numerical quadrature}
\label{sec:fe_quad}
\subsection{Motivation}
The entries of the element stiffness matrix $\hat A^K \in \RR^{n_s \times n_s}$ can be expressed as an integral over the reference element $\tilde K$:
\begin{align*}
  \hat A^K_{\alpha\beta}
  &\equiv \int_{K} \left. \pp{\phi_\alpha^K}{x_i} \right|_{x} a_{ij}(x) \left. \pp{\phi_\beta^K}{x_j} \right|_x dx
  \\
  &= \int_{\tilde K} \left. \pp{\phi_\alpha^K}{x_i} \right|_{x(\tilde x)} a_{ij}(x(\tilde x)) \left. \pp{\phi_\beta^K}{x_j} \right|_{x(\tilde x)} |J(\tilde x)| d\tilde x , \quad \alpha,\beta = 1,\dots,n_s.
\end{align*}
So far in this lecture we have introduced the means to evaluate all the terms that appear in the integrand --- $\left. \pp{\phi_\alpha^K}{x_i} \right|_{x(\tilde x)}$, $a_{ij}(x(\tilde x))$, and $|J(\tilde x)|$.  The last ingredient we need is a means to evaluate the integral over $\tilde K$ from a finite evaluation of the integrand.

We now introduce \emph{numerical quadrature} (or just \emph{quadrature}) to perform the integration.  Specifically, our quadrature problem is as follows: given $f: \tilde K \to \RR$, estimate the integral
\begin{equation*}
  I \equiv \int_{\tilde K} f(\tilde x) d \tilde x
\end{equation*}
by
\begin{equation*}
  Q \equiv \sum_{q=1}^{n_q} \tilde \rho_q f(\tilde \xi_q),
\end{equation*}
where $\{\tilde \xi_q \in \tilde K \}_{q=1}^{n_q}$ is a set of \emph{quadrature points} and $\{\tilde \rho_q \in \RR \}_{q=1}^{n_q}$ is the associated set of \emph{quadrature weights}.

\subsection{Gauss quadrature in $\RR^1$}
\label{sec:fe_quad_1d}
We first consider a one-dimensional quadrature for a unit line segment $\tilde I \equiv (0,1) \subset \RR^1$.  (Note: one-dimensional quadrature rules are often defined for the line segment $(-1,1)$; in this lecture we define them for $(0,1)$ to be consistent with our definition of a unit line segment.)  While there are many one-dimensional quadrature rule, we focus here on arguably the most efficient quadrature rule: the Gauss quadrature.

The $n_q$-point Gauss quadrature rule is defined by quadrature points $\{\tilde \xi_q \in \tilde I \}_{q=1}^{n_q}$ and quadrature weights $\{ \tilde \rho_q \in \tilde I \}_{q=1}^{n_q}$ such that the rule integrates exactly polynomials of degrees up to and including $2n_q - 1$: i.e.,
\begin{equation}
  \int_{\tilde I} f(\tilde x) d\tilde x = \sum_{q=1}^{n_q} \tilde \rho_q f(\tilde \xi_q) \quad \forall f \in \PP^{2n_q-1}(\tilde I).
  \label{eq:fe_impl_gauss_cond}
\end{equation}
Our intuition might suggest the existence of a $n_q$-point quadrature rule that integrates exactly polynomials of degree $2n_q-1$, as the polynomials have $2n_q$ degrees of freedom and the quadrature rule also has $2n_q$ degrees of freedom --- $n_q$ points and $n_q$ weighs.


We can show the existence of a $n_q$-point quadrature rule that integrates exactly polynomials of degree $2n_q-1$ in a constructive manner using the scaled Legendre polynomials $\{ \tilde \psi_i \}$. (Our Legendre polynomials are scaled such that they are orthogonal over with respect to $L^2(\tilde I \equiv (0,1))$ instead of the usual $L^2((-1,1))$.) The quadrature points $\{ \tilde \xi_q \}_{q=1}^{n_q}$ are the roots of the degree $n_q$ Legendre polynomial:
\begin{equation}
  \tilde \psi_{n_q}(\tilde \xi_q) = 0, \quad q = 1,\dots,n_q.
  \label{eq:fe_impl_gauss_points}
\end{equation}
The quadrature weights $\{\tilde \rho_q \}_{q=1}^{n_q}$ are then chosen to satisfy the following linear equation:
\begin{equation}
  \bmat{ccc}
  \tilde \psi_0(\tilde \xi_1) & \dots & \tilde \psi_0(\tilde \xi_{n_q}) \\
  \vdots & \ddots & \vdots \\
  \tilde \psi_{n_q-1}(\tilde \xi_1) & \dots & \tilde \psi_{n_q-1}(\tilde \xi_{n_q}) 
  \emat
  \bmat{c}
  \tilde \rho_1 \\ \vdots \\ \tilde \rho_{n_q}
  \emat
  =
  \bmat{c}
  \int_{\tilde I} \tilde \psi_0(\tilde x) d\tilde x \\
  \vdots \\
  \int_{\tilde I} \tilde \psi_{n_q-1}(\tilde x) d\tilde x
  \emat .
  \label{eq:fe_impl_gauss_weights}
\end{equation}
The linear system is well-posed because the Legendre polynomials $\{ \tilde \psi_i \}_{i=0}^{n_q-1}$ are linearly independent and the $n_q$ quadrature points are distinct.

We wish to show the conditions~\eqref{eq:fe_impl_gauss_points}~and~\eqref{eq:fe_impl_gauss_weights} yield a quadrature rule that integrates exactly polynomials of degree $2n_q-1$. To begin, we introduces a basis $\{p_i\}_{i=0}^{2n_q-1}$ for $\PP^{2n_q-1}(\tilde I)$ such that $\forall \tilde x \in \tilde I$
\begin{align*}
  p_0(\tilde x) &= \tilde \psi_0(\tilde x) = 1, &
  p_1(\tilde x) &= \tilde \psi_1(\tilde x), &
  &\dots, &
  p_{n_q-1}(\tilde x) &= \tilde \psi_{n_q-1}(\tilde x) \\
  p_{n_q}(\tilde x) &= \tilde \psi_{n_q}(\tilde x) \tilde \psi_0(\tilde x), & 
  p_{n_q+1}(\tilde x) &= \tilde \psi_{n_q}(\tilde x) \tilde \psi_1(\tilde x), &  
  &\dots, &
  p_{2n_q-1}(\tilde x) &= \tilde \psi_{n_q}(\tilde x) \tilde \psi_{n_q-1}(\tilde x) .
%  p_0(x) &= \tilde \psi_0(x) = 1 \\
%  p_1(x) &= \tilde \psi_1(x) \\
%  &\vdots \\
%  p_{n_q-1}(x) &= \tilde \psi_{n_q-1}(x) \\
%  p_{n_q}(x) &= \tilde \psi_{n_q}(x) \tilde \psi_0(x) \\
%  p_{n_q+1}(x) &= \tilde \psi_{n_q}(x) \tilde \psi_1(x) \\
%  &\vdots \\
%  p_{2n_q-1}(x) &= \tilde \psi_{n_q}(x) \tilde \psi_{n_q-1}(x) 
\end{align*}
We now wish to confirm that
\begin{equation}
  \int_{\tilde I} p_i(\tilde x) dx = \sum_{q=1}^{n_q} \tilde \rho_q p_i(\tilde \xi_q), \quad \forall i = 0,\dots,2 n_q - 1,
  \label{eq:fe_impl_gauss_cond_2}
\end{equation}
which is equivalent to the original condition~\eqref{eq:fe_impl_gauss_cond}. We first readily confirm that the first $n_q$ basis functions, $\{p_i\}_{i=0}^{n_q} \equiv \{ \tilde \psi_i \}_{i=0}^{n_q}$, are integrated exactly because our weights $\{ \tilde \rho_q \}_{q=1}^{n_q}$ are chosen to integrate the functions in condition~\eqref{eq:fe_impl_gauss_weights}. 

To see that the next $n_q$ basis functions, $\{ p_i \}_{i=n_q}^{2n_q-1}$ are integrated exactly, we see that the left-hand side of~\ref{eq:fe_impl_gauss_cond_2} for $i = n_q,\dots,2_{n_q}-1$ yields 
\begin{equation*}
  (\text{LHS})
  = \int_{\tilde I} p_{p+j}(\tilde x) d \tilde x
  = \int_{\tilde I} \tilde \psi_{n_q}(\tilde x) \tilde \psi_{j}(\tilde x) d \tilde x
  = 0, \quad j = 0,\dots,n_q,
\end{equation*}
since the Legendre polynomials are orthogonal in $L^2(\tilde I)$. On the other hand, the right-hand side of~\ref{eq:fe_impl_gauss_cond_2} for $i = n_q,\dots,2_{n_q}-1$ yields
\begin{equation*}
  (\text{RHS})
  = \sum_{q=1}^{n_q} \tilde \rho_q p_{p+j}(\tilde \xi_q)
  = \sum_{q=1}^{n_q} \tilde \rho_q \tilde \psi_{n_q}(\tilde \xi_q) \tilde \psi_{j}(\tilde \xi_q)
  = 0, \quad j = 0,\dots,n_q,
\end{equation*}
since $\{\tilde \xi_q\}_{q=1}^{n_q}$ are the roots of $\tilde \psi_{n_q}$ by~\eqref{eq:fe_impl_gauss_points}. In summary, the exact integration condition~\ref{eq:fe_impl_gauss_cond_2} are satisfied (i) for $i = 0,\dots,n_q-1$ because of the choice of the weights $\{\tilde \rho_q\}_{q=1}^{n_q}$ ~\eqref{eq:fe_impl_gauss_weights} and (ii) for $i = n_q, \dots, 2 n_q-1$ because of the choice of the points $\{\tilde \rho_q\}_{q=1}^{n_q}$ by~\eqref{eq:fe_impl_gauss_points}.

Table~\ref{tb:fe_impl_gauss} shows the Gauss quadrature rules for $n_q = 1,\dots,4$.  As visualized in Figure~\ref{fig:fe_impl_gauss_points}, the quadrature points are clustered towards the endpoints.
\begin{figure}
  \centering
  \subfigure[$p = 1$]{
    \includegraphics[width=0.45\textwidth]{quad_1d_p1}
  }
  \subfigure[$p = 3$]{
    \includegraphics[width=0.45\textwidth]{quad_1d_p3}
  }
  \subfigure[$p = 5$]{
    \includegraphics[width=0.45\textwidth]{quad_1d_p5}
  }
  \subfigure[$p = 7$]{
    \includegraphics[width=0.45\textwidth]{quad_1d_p7}
  }
  \caption{Gauss quadrature points for $(0,1)$. \label{fig:fe_impl_gauss_points}}
\end{figure}

\begin{table}
  \centering
  \begin{tabular}{cccc}
    $p$ & $n_q$ & $\tilde \xi$ & $\tilde \rho$ \\
    \hline
    $1$ & $1$ & $0.500000000000000$ & $1.000000000000000$ \\
    \hline
    $3$ & $2$ & $0.211324865405187$ & $0.500000000000000$ \\ 
    & & $0.788675134594813$ & $0.500000000000000$ \\
    \hline
    $5$ & $3$ & $0.112701665379258$ & $0.277777777777778$ \\ 
     & & $0.500000000000000$ & $0.444444444444444$ \\ 
     & & $0.887298334620742$ & $0.277777777777778$ \\
    \hline
    $7$ & $4$ & $0.069431844202974$ & $0.173927422568727$ \\ 
    & & $0.330009478207572$ & $0.326072577431273$ \\ 
    & & $0.669990521792428$ & $0.326072577431273$ \\ 
    & & $0.930568155797026$ & $0.173927422568727$ 
  \end{tabular}
  \caption{Gauss quadrature rules for $p = 1$ to $7$ polynomials. \label{tb:fe_impl_gauss}}
  \label{tb:integ_gauss}
\end{table}

\subsection{Numerical quadrature in $\RR^d$}
\label{sec:fe_quad_nd}
Similarly to the Gauss quadrature in $\RR^1$, there also exists efficient quadrature rules for integration of domains in $\RR^d$, $d > 1$.  If the domain is a square $[0,1]^2 \subset \RR^2$ or cube $[0,1]^3 \subset \RR^3$, then we can obtain the associated quadrature rule by the tensor-product of the one-dimensional Gauss rules; for example, for $[0,1]^2$, 
\begin{equation*}
  \int_{\tilde x_2=0}^1 \int_{\tilde x_1=0}^1 f(\tilde x_1,\tilde x_2) d\tilde x_1 d\tilde x_2
  \approx \sum_{i_2 = 1}^{n_q^{\rm 1d}} \sum_{i_1 = 1}^{n_q^{\rm 1d}} \tilde \rho_{i_2}^{\rm 1d} \tilde \rho_{i_1}^{\rm 1d} f(\tilde \xi_1^{\rm 1d}, \tilde \xi_2^{\rm 1d}).
\end{equation*}
These rules maximizes the degree of tensor-product polynomials integrated exactly for a given number of quadrature points.

For a domain in $\RR^d$, $d > 1$, that does not result from a tensor-product of a one-dimensional domain, the ``optimal'' quadrature rules are much more difficult to identify.  In fact, the optimal rules for (say) a triangle is not as universally standardized as that for a square. Table~\ref{tb:fe_impl_integ_gauss2} shows examples of efficient quadrature rule for our unit right triangle, and Figure~\ref{fig:fe_impl_integ_gauss2} visualizes the quadrature points.  Similarly to the one-dimensional Gauss rule, the quadrature points are clustered towards the edge of the triangular domain.


\begin{table}
  \centering
  \begin{tabular}{ccccc}
    $p$ & $n_q$ & $\tilde \xi_1$ & $\tilde \xi_2$ & $\tilde \rho$ \\
    \hline
    $1$ & $1$ & $0.333333333333333$ & $0.333333333333333$ & $0.500000000000000$ \\
    \hline
    $2$ & $3$ & $0.166666666666667$ & $0.166666666666667$ & $0.166666666666667$ \\ 
    & & $0.666666666666667$ & $0.166666666666667$ & $0.166666666666667$ \\ 
    & & $0.166666666666667$ & $0.666666666666667$ & $0.166666666666667$ \\ 
    \hline
    $4$ & $6$ & $0.091576213509771$ & $0.091576213509771$ & $0.054975871827661$ \\ 
    & & $0.816847572980459$ & $0.091576213509771$ & $0.054975871827661$ \\ 
    & & $0.091576213509771$ & $0.816847572980459$ & $0.054975871827661$ \\ 
    & & $0.445948490915965$ & $0.445948490915965$ & $0.111690794839006$ \\
    & & $0.108103018168070$ & $0.445948490915965$ & $0.111690794839006$ \\ 
    & & $0.445948490915965$ & $0.108103018168070$ & $0.111690794839006$ \\ 
  \end{tabular}
  \caption{Gauss quadrature rules for $p = 1$ to $4$ polynomials.}
  \label{tb:fe_impl_integ_gauss2}
\end{table}

\begin{figure}
  \centering
  \subfigure[$p = 1$]{
    \includegraphics[width=0.23\textwidth]{quad_2d_p1}
  }
  \subfigure[$p = 2$]{
    \includegraphics[width=0.23\textwidth]{quad_2d_p2}
  } 
  \subfigure[$p = 4$]{
    \includegraphics[width=0.23\textwidth]{quad_2d_p4}
  } 
  \subfigure[$p = 5$]{
    \includegraphics[width=0.23\textwidth]{quad_2d_p5}
  }
  \subfigure[$p = 6$]{
    \includegraphics[width=0.23\textwidth]{quad_2d_p6}
  }
  \subfigure[$p = 7$]{
    \includegraphics[width=0.23\textwidth]{quad_2d_p7}
  }
  \caption{Numerical quadrature points for the reference triangle.}
  \label{fig:fe_impl_integ_gauss2}
\end{figure}

\section{Assembly}
\label{sec:fe_assembly}
\subsection{Local stiffness matrix and vectors}
We now put together the techniques we have learned in this lecture to evaluate local stiffness matrices and load vectors.  To provide a concrete example, in this section we consider $\calV \equiv H^1(\Omega)$, a bilinear form $a: \calV \times \calV \to \RR$ such that
\begin{equation}
  a(w,v) \equiv \int_\Omega \nabla v \cdot a \nabla w dx , \quad \forall w,v \in \calV,
  \label{eq:fe_impl_model_a}
\end{equation}
for $a: \Omega \to \RR^{d \times d}$ the diffusion tensor field, and a linear form $\ell: \calV \to \RR$ such that
\begin{equation}
  \ell(v) \equiv \int_\Omega v f dx \quad \forall v \in \calV,
  \label{eq:fe_impl_model_ell}
\end{equation}
for $f: \Omega \to \RR$ the source function. Our approximation space is given by
\begin{equation}
  \calV_h \equiv \{ v \in \calV \ | \ v|_K \circ \calG^K \in \PP^p(\tilde K), \ \forall K \in \calT_h \},
  \label{eq:fe_impl_loc_vh}
\end{equation}
where $\calG^K: \tilde K \to K$ is a $\PP^p$ isoparametric mapping. The number of shape functions per element is denoted by $n_s$.

We first consider the evaluation of the \emph{local load vector}, $\hat f^K \in \RR^{n_s}$ such that
\begin{equation*}
  \hat f^K_\alpha \equiv \ell(\phi^K_\alpha) \quad \forall \alpha = 1,\dots,n_s.
\end{equation*}
For our model linear form~\eqref{eq:fe_impl_model_ell}, the local load vector is given by
\begin{equation*}
  \hat f^K_\alpha
  \equiv \ell(\phi^K_\alpha) \equiv \int_K \phi^K_\alpha(x) f(x) dx , \quad \alpha = 1,\dots, n_s.
\end{equation*}
We now change the integration domain from the physical domain $K$ to the reference domain $\tilde K$; this is in preparation for the application of the numerical quadrature on the reference domain.  We employ (i) the isoparametric mapping for the argument of $f$, $x = \calG^K(\tilde x)$, (ii) the relationship between the reference and physical shape functions, $\phi^K_\alpha(x \equiv \calG^K(\tilde x)) = \tilde \phi_\alpha(\tilde x)$, and (iii) the area transformation $dx = |J(\tilde x)| d \tilde x$, where $|J|$ is the determinant of the Jacobian of the isoparametric mapping, to obtain 
\begin{equation}
  \hat f^K_\alpha
  = \int_{\tilde K} \phi_\alpha(\tilde x) f(\calG^K(\tilde x)) |J(\tilde x)| d \tilde x .
  \label{eq:fe_impl_loc_vec_2}
\end{equation}
We finally apply a $n_q$-point numerical quadrature to ``evaluate'' the integral:
\begin{equation*}
    \hat f^K_\alpha \
    \text{``}\hspace{-0.2em}=\hspace{-0.2em}\text{''}
    \ \sum_{q=1}^{n_q} \tilde \rho_q \tilde \phi_\alpha(\tilde \xi_q) f(\calG^K(\tilde \xi_q)) |J(\tilde \xi_q)| .
\end{equation*}
Here, we put the equality in quotes because the evaluation is exact only if the integrand is a polynomial that can be integrated exactly using the quadrature rule.  Otherwise, the use of the numerical quadrature results in an \emph{approximation} rather than an \emph{evaluation}. In any event, we can readily evaluate the quantities in the summand using the techniques discussed in the previous sections; we here make a listing for a reference
\begin{itemize}
\item Quadrature rule $\{\tilde \xi_q, \tilde \rho_q\}_{q=1}^{n_q}$ was discussed in Sections~\ref{sec:fe_quad}--\ref{sec:fe_quad_nd}.
\item The evaluation of quantities associated with isoparametric mapping, $\calG^K$ and $J$, was discussed in Sections~\ref{sec:fe_iso_map}.
\item The evaluation of reference shape functions $\{\tilde \phi_\alpha \}_{\alpha=1}^{n_s}$ was discussed in Sections~\ref{sec:fe_lin_line}--\ref{sec:fe_gen_shape}.
\end{itemize}
We hence can evaluate (or approximate) the local load vector $\hat f^K \in \RR^{n_s}$.

We now consider the evaluation of the \emph{local stiffness matrix}, $\hat A^K \in \RR^{n_s \times n_s}$ such that
\begin{equation*}
  \hat A^K_{\alpha,\beta} \equiv a(\phi^K_\beta,\phi^K_\alpha) \quad \forall \alpha,\beta = 1,\dots,n_s,
\end{equation*}
For our model bilinear form~\eqref{eq:fe_impl_model_a}, the local stiffness matrix is given by
\begin{equation*}
  \hat A^K_{\alpha,\beta}
  \equiv
  a(\phi^K_\beta,\phi^K_\alpha) = \int_K \nabla \phi^K_\alpha \cdot a \nabla \phi^K_\beta dx , \quad \alpha,\beta = 1,\dots,n_s.
\end{equation*}
Following the same procedure we used for the evaluation (or approximation) of the local load vector, we now change the integration domain from the physical domain $K$ to the reference domain $\tilde K$.  We employ (i) the isoparametric mapping for the argument of $a$, $x = \calG^K(\tilde x)$, (ii) the relationship between the derivatives of the reference and physical shape functions, $\nabla \phi^K(x = \calG(\tilde x)) = J^K(\tilde x)^{-T} \tilde \nabla \tilde \phi(\tilde x)$, and (iii) the area transformation $dx = |J^K(\tilde x)| d \tilde x$, where $|J^K|$ is the determinant of the Jacobian of the isoparametric mapping, to obtain
\begin{equation}
  \hat A^K_{\alpha,\beta}
  =
  \int_{\tilde K} ((J^K)^{-T}(\tilde x) \tilde \nabla \tilde \phi(\tilde x))
  \cdot a(\calG^K(\tilde x)) (J^K)^{-T}(\tilde x) \tilde \nabla \tilde \phi(\tilde x)
  |J(\tilde x)| d\tilde x
  \label{eq:fe_impl_loc_mat_2}
\end{equation}
We finally apply a $n_q$-point numerical quadrature to ``evaluate'' the integral:
\begin{equation*}
  \hat A^K_{\alpha,\beta}
  \text{``}\hspace{-0.2em}=\hspace{-0.2em}\text{''}
  \sum_{q=1}^{n_q} \tilde \rho_q
   (J^K(\tilde \xi)^{-T} \tilde \nabla \tilde \phi(\tilde \xi))
  \cdot a(\calG^K(\tilde \xi)) J^K(\tilde \xi)^{-T} \tilde \nabla \tilde \phi(\tilde \xi)
  |J^K(\tilde \xi)|.
\end{equation*}
Again, we put the equality in quotes because the evaluation is exact only if the integrand is a polynomial that can be integrated exactly using the quadrature rule.  In particular, in the presence of a curved element, the integrand is almost always non-polynomial because the inverse Jacobian is non-polynomial. The techniques used to evaluate the element stiffness matrix are almost identical to those used to evaluate the element load vector, except this time we also employ the technique to evaluate the derivative of the physical shape functions discussed in Section~\ref{sec:fe_phy_shape}. If the bilinear form contains other terms, such as a convection term of the form $\int_{\Omega} v b \cdot \nabla w dx$ or a reaction term of the form $\int_\Omega c v w dx$, they can also be evaluated in a similar manner.


We make one note about the expressions for the local load vector and stiffness matrix.  In literature, the expressions~\eqref{eq:fe_impl_loc_vec_2} and \eqref{eq:fe_impl_loc_mat_2} are sometimes restated as
\begin{align*}
  \hat f^K_\alpha
  &=
  \int_{\tilde K} \phi_\alpha(\tilde x) \underbrace{ f(\calG^K(\tilde x)) |J^K(\tilde x)| }_{\equiv f^{\rm trans}(\tilde x)}d \tilde x, \\
  \hat A^K_{\alpha,\beta}
  &=
  \int_{\tilde K} \tilde \nabla \tilde \phi(\tilde x)
  \cdot  \underbrace{ (J^K(\tilde x)^{-1} a(\calG^K(\tilde x)) J^K(\tilde x)^{-T} |J^K(\tilde x)| }_{\equiv  a^{\rm trans}(\tilde x)}\tilde \nabla \tilde \phi(\tilde x)  d\tilde x,
\end{align*}
where $f^{\rm trans}: \tilde K \to \RR$  and $a^{\rm trans}:\tilde K \to \RR^{d \times d}$ are the transformed coefficients for the reference domain $\tilde K$. Note that the difference lies only in the interpretation; the local load vector and stiffness matrix for both formulations are mathematically the same.

\subsection{Local stiffness matrix and vectors: facet terms}
We now consider evaluation of a load vector that requires integration on a boundary and, in turn, on a facet.  Again to provide a concrete example, in this section we consider $\calV \equiv H^1(\Omega)$, a linear form $\ell: \calV \to \RR$ such that
\begin{equation}
  \ell(v) \equiv \int_{\Gamma_N} v g dx \quad \forall v \in \calV,
  \label{eq:fe_impl_model_g}
\end{equation}
for $g: \Gamma_N \to \RR$ the Neumann source function. Our approximation space $\calV_h$ is as defined in~\eqref{eq:fe_impl_loc_vh}.  The number of shape fucntions per element is denoted by $n_s$, and the number of shape functions on a facet of the element is denoted by $n_s^F$.  (We only consider the load vector, and not the stiffness matrix, as the evaluation procedures are essentially the same.)

To evaluate the local load vector $\tilde f^K \in \RR^{n_s}$, we appeal to the fact that for the shape functions $\{\phi^K_{\alpha}\}_{\alpha=1}^{n_s}$ in $\RR^d$ their trace is given by the shape functions $\{\chi^F_\gamma\}_{\gamma=1}^{n_s^F}$ in $\RR^{d-1}$.  Specifically, we appeal to the map
\begin{equation*}
  \ell(\phi^{K(F)}_{\alpha = \theta_{\rm e-f}(i,\gamma)}|_{F}) = \ell(\chi_\gamma^{F}), \quad \forall \gamma = 1,\dots,n_s^F,
\end{equation*}
where $F(K,i)$ is the $i$-th facet of the element $K$, and $\alpha = \theta_{\rm e-f}(F,\gamma)$ is the Lagrange node of the element $K$ that corresponds to the $\gamma$-th Lagrange node of the facet $F(K,i)$.  We now evaluate our model linear form~\eqref{eq:fe_impl_model_g} for $\{\chi_\gamma^F\}_{\gamma=1}^{n_s^F}$, 
\begin{equation*}
  \hat g^F_\gamma \equiv \ell(\chi_\gamma^F) = \int_{F} \chi^F_\gamma(s) g(s) ds, \quad \gamma = 1,\dots,n_s^F.
\end{equation*}
We then change the integration domain from the physical facet $F$ to the reference line segment $\tilde I$.  We employ (i) the facet isoparametric mapping for the argument of $g$, $s \equiv \calG^F(\tilde s)$, (ii) the relationship between the reference and physical shape functions, $\tilde \chi^F_\gamma(s \equiv \calG^F(\tilde s)) = \tilde \chi(\tilde s)$, and (iii) the length transformation $ds = |J^F(\tilde s)|d \tilde s$, to obtain
\begin{equation*}
  \hat g^F_\gamma = \int_{\tilde I} \tilde \chi_\gamma(\tilde s) g(\calG^F(\tilde s)) |J^F(\tilde s)| d \tilde s.
\end{equation*}
We then apply a $n_q$-point numerical quadrature defined by $\{ \tilde \xi_q, \tilde \rho_q \}_{q=1}^{n_q}$ to ``evaluate'' the integral:
\begin{equation*}
  \hat g^F_\gamma
  \text{``}\hspace{-0.2em}=\hspace{-0.2em}\text{''}
  \sum_{q=1}^{n_q} \tilde \rho_q \tilde \chi_\gamma(\tilde \xi_q) g(\calG^F(\tilde \xi_q)) |J^F(\tilde \xi_q)| .
\end{equation*}
We finally map the vector $\hat g^F \in \RR^{n_s^F}$ to $\hat f^K \in \RR^{n_s}$ according to
\begin{equation*}
  \hat f^{K}_{\alpha \equiv \theta_{\rm e-f}(F(K,i),\gamma)} = \hat g^F_\gamma, \quad \forall \gamma = 1,\dots,n_s^F,
\end{equation*}
where $F(K,i)$ is the $i$-th facet of the element $K$.

\subsection{Matrix and vector assembly}

We have so far introduced technical ingredients required to assemble, for any $K \in \calT_h$, 
and the local load vector $\hat f^K \in \RR^{n_s}$ such that
\begin{equation*}
  \hat f^K_\alpha = \ell(\phi^K_\alpha) \quad \forall \alpha = 1,\dots,n_s.
\end{equation*}
We now wish to assemble the local stiffness matrices and vectors construct the (global) stiffness matrix and vector.

To this end, we employ the element-to-node connectivity map,
\begin{equation*}
  \theta : \{ K_1, \dots, K_{n_e} \} \times \{ 1, \dots, n_s \} \to \{ 1,\dots,n \},
\end{equation*}
which takes in the element and local node number as the input and returns the associated global node number.  We recall that the maps is typically stored as a table of the size $n_e \times n_s$.  Then, to form the (global) stiffness matrix $\hat A_h \in \RR^{n \times n}$, we successively insert the local stiffness matrices $\hat A^K \in \RR^{n_s \times n_s}$, $K \in \calT_h$, 
\begin{equation*}
  \hat A_{h,ij} \leftarrow \hat A_{h,ij} + \hat A^K_{\alpha\beta}, \quad \forall \alpha,\beta = 1,\dots,n_s,
\end{equation*}
where $i = \theta(K,\alpha)$ and $j = \theta(K,\beta)$ are the global node indices associated with the local nodes $\alpha$ and $\beta$, respectively, of the element $K$. Similarly, to form the (global) load vector $\hat f_h \in \RR^n$, we successively insert the local load vectors $\hat f^K \in \RR^{n_s}$, $K \in \calT_h$,
\begin{equation*}
  \hat f_{h,i} \leftarrow \hat f_{h,i} + \hat f^K_{\alpha}, \quad \forall \alpha = 1,\dots,n_s,
\end{equation*}
where $i = \theta(K,\alpha)$ is the global node index associated with the local node $\alpha$ of the element $K$.

%\section{Efficient implementation by BLAS}

%$\Phi \in \RR^{n_q \times n_s}$
%\begin{equation*}
%  \tilde \Phi_{qi} = \tilde \phi_i(\tilde \xi_q)
%\end{equation*}

%\begin{equation*}
%  \widetilde {\nabla \Phi}_{qij} = \left. \pp{\tilde \phi_i}{\tilde x_j} \right|_{\tilde \xi_q}
%\end{equation*}

%\begin{equation*}
%  \nabla \Phi_{qij} = \left. 
%\end{equation*}




%% \begin{figure}
%%   \centering
%%   \includegraphics[width=0.48\textwidth]{shape_global_p1}
%% \end{figure}



%% \begin{figure}
%%   \centering
%%   \subfigure[vertex shape function]{
%%     \includegraphics[width=0.48\textwidth]{shape_global_p2_1}
%%   }
%%   \subfigure[edge shape function]{
%%     \includegraphics[width=0.48\textwidth]{shape_global_p2_2}
%%   }
%% \end{figure}


%% \begin{figure}
%%  \centering
%%  \includegraphics[width=0.4\textwidth]{fe_mesh_p2}
%%  \caption{$p=2$ mesh.}
%%  \label{fig:fe_mesh_p2}
%% \end{figure}
%% \begin{table}
%%   \centering
%%   \subfigure[coordinates]{
%%     \begin{tabular}{c|cc}
%%       node & $x_1$ & $x_2$ \\
%%       \hline
%% $1$ & $-0.89$ & $\hphantom{-}0.45$ \\ 
%% $2$ & $-0.89$ & $-0.46$ \\ 
%% $3$ & $-0.29$ & $\hphantom{-}0.04$ \\ 
%% $4$ & $-0.21$ & $-0.98$ \\ 
%% $5$ & $-0.21$ & $\hphantom{-}0.98$ \\ 
%% $6$ & $\hphantom{-}0.28$ & $-0.07$ \\ 
%% $7$ & $\hphantom{-}0.60$ & $\hphantom{-}0.80$ \\ 
%% $8$ & $\hphantom{-}0.61$ & $-0.79$ \\ 
%% $9$ & $1.00$ & $\hphantom{-}0.02$ \\ 
%% $10$ & $-1.00$ & $-0.01$ \\ 
%% $11$ & $-0.59$ & $\hphantom{-}0.25$ \\ 
%% $12$ & $-0.61$ & $\hphantom{-}0.79$ \\ 
%% $13$ & $-0.59$ & $-0.21$ \\ 
%% $14$ & $-0.61$ & $-0.79$ \\ 
%% $15$ & $-0.25$ & $-0.47$ \\ 
%% $16$ & $-0.25$ & $\hphantom{-}0.51$ \\ 
%% $17$ & $-0.01$ & $-0.01$ \\ 
%% $18$ & $\hphantom{-}0.03$ & $-0.52$ \\ 
%% $19$ & $\hphantom{-}0.22$ & $-0.98$ \\ 
%% $20$ & $\hphantom{-}0.03$ & $\hphantom{-}0.46$ \\ 
%% $21$ & $\hphantom{-}0.22$ & $\hphantom{-}0.98$ \\ 
%% $22$ & $\hphantom{-}0.44$ & $\hphantom{-}0.36$ \\ 
%% $23$ & $\hphantom{-}0.44$ & $-0.43$ \\ 
%% $24$ & $\hphantom{-}0.64$ & $-0.02$ \\ 
%% $25$ & $\hphantom{-}0.89$ & $\hphantom{-}0.46$ \\ 
%% $26$ & $\hphantom{-}0.90$ & $-0.43$ \\   
%%     \end{tabular}
%%   }
%%   \subfigure[connectivity]{
%%     \begin{tabular}{c|cccccc}
%%       element & node 1 & node 2 & node 3 & node 4 & node 5 & node 6\\
%%       \hline
%% $1$ & $9$ & $6$ & $8$ & $23$ & $26$ & $24$ \\ 
%% $2$ & $8$ & $6$ & $4$ & $18$ & $19$ & $23$ \\ 
%% $3$ & $4$ & $6$ & $3$ & $17$ & $15$ & $18$ \\ 
%% $4$ & $3$ & $5$ & $1$ & $12$ & $11$ & $16$ \\ 
%% $5$ & $6$ & $5$ & $3$ & $16$ & $17$ & $20$ \\ 
%% $6$ & $7$ & $6$ & $9$ & $24$ & $25$ & $22$ \\ 
%% $7$ & $7$ & $5$ & $6$ & $20$ & $22$ & $21$ \\ 
%% $8$ & $2$ & $3$ & $1$ & $11$ & $10$ & $13$ \\ 
%% $9$ & $4$ & $3$ & $2$ & $13$ & $14$ & $15$ \\ 
%%     \end{tabular}
%%   }
%%   \caption{Node coordinate and connectivity table for $p=2$ mesh shown in Figure~\ref{fig:fe_mesh_p2}}
%% \end{table}



%i.e., for a piecewise polynomial space $\calV_h$,
%\begin{equation*}
%  \calV_h \subset \calV \quad \Leftrightarrow \quad \calV_h \subset C^0(\overline \Omega) ,
%\end{equation*}
%where  $C^0(\overline \Omega)$ is the space of continuous functions over $\overline \Omega$.

%Given the continuity requirement, we will construct finite element spaces of the form
%\begin{equation}
%  \calV_h \equiv \{ v \in C^0(\overline \Omega) \ | v |_{K_i} \in \PP^p(K_i), \ i = 1,\dots, n_e \};
%  \label{eq:fe_space}
%\end{equation}
%we recall that $\PP^p(K_i)$ is the space of degree $p$ polynomials over $K_i$.





%% \section{Linear Lagrange element on a line segment}
%% \label{sec:fe_lin_line}
%% We first introduce arguably the simplest finite element: linear Lagrange element on a unit line segment $\tilde K$.  Our unit line segment $\tilde K \equiv (\tilde x^1, \tilde x^2)$ is delineated by two endpoints
%% \begin{equation*}
%%   \tilde x^1 = 0 \quad \text{and} \quad \tilde x^2 = 1.
%% \end{equation*}
%% For the linear polynomial space $\PP^1(\tilde K)$ and the interpolation points $\{\tilde x^1, \tilde x^2\}$, a unique set of \emph{Lagrange basis functions} (or \emph{Lagrange shape functions}) is given by
%% \begin{equation*}
%%   \tilde \phi_1(\tilde x) = 1 - \tilde x \quad \text{and} \quad \tilde \phi_2(\tilde x) = \tilde x.
%% \end{equation*}
%% Note that these basis functions satisfy the interpolation condition
%% \begin{equation*}
%%   \phi_i(\tilde x^j) = \delta_{ij}.
%% \end{equation*}
%% Here $\delta_{ij}$ is the \emph{Kronecker delta}: $\delta_{ij} = 1$ for $i = j$ and $\delta_{ij} = 0$ for $i \neq j$.

%% With these basis functions, we can describe any function $v \in \PP^1(\tilde K)$ as
%% \begin{equation*}
%%   v = \sum_{i=1}^{n_s} \tilde v_i \tilde \phi_i
%% \end{equation*}
%% for $\tilde v_i \equiv v(\tilde x^i)$, $i = 1,2$; the values of the function at the end points are the degree of freedom of the finite element.  Similarly, the derivative of the function is given by
%% \begin{equation*}
%%   \pp{v}{\tilde x} = \sum_{i=1}^{n_s} \tilde v_i \pp{\tilde \phi_i}{\tilde x},
%% \end{equation*}
%% where the direct differentiation of the basis functions yields $\pp{\tilde \phi_1}{\tilde x} = -1$ and $\pp{\tilde \phi_2}{\tilde x} = 1$.


%To see the equivalence, we observe that .  Conversely, if a polynomial space is no
%We hence choose
%\begin{equation*}
%  \calV_h \equiv \{ v \in C^0(\overline \Omega) \ | v |_{K_i} \in \PP^p(K_i), \ i = 1,\dots, n_e \};
%\end{equation*}
%





%% Using the linear Lagrange basis functions, we can now represent any function $v$ that is in $\PP^1(\tilde K)$.  Specifically, we may represent $v \in \PP^1(\tilde K)$ in terms of a coefficient vector $\hat v \in \RR^3$ as
%% \begin{equation*}
%%   v(\tilde x) = \sum_{j=1}^{3} \hat v_j  \tilde \phi_j(\tilde x) \quad \forall \tilde x \in \tilde K
%% \end{equation*}
%% for $\hat v_j \equiv v(\tilde x^j)$, $j = 1,2,3$.  We hence have a one-to-one mapping between \emph{any} element in $\PP^1(\tilde K)$ and the associated coefficient vector in $\RR^3$.  For the linear Lagrange basis functions, the coefficients $\hat v \in \RR^3$ is associated with the values of the function at the vertices of the triangle.


%% For a quadratic triangle, the mapping is quadratic and vertices and mid-edge nodes of the reference triangle are mapped to the respective vertices and mid-edge nodes of the physical triangle.

%% Formally, a finite element space is parametrized by the following three properties:
%% \begin{enumerate}
%% \item the triangulation $\calT_h$ of $\Omega$;
%% \item the type of functions that constitutes the space (e.g., piecewise linear polynomial);
%% \item the degrees of freedom used to describe functions in the space.
%% \end{enumerate}
%% The first two are apparent from the definition of the finite element space~\eqref{eq:fe_space}.  The last property determines how a function $v \in \calV_h$ is represented on a computer.  Specifically, given a $N$-dimensional function space $\calV_h$, we assign $N$ degrees of --- by choosing $N$ basis functions --- such that the a function $v \in \calV_h$ can be uniquely described by $N$ real numbers.  We will clarify this third property in Section~\ref{sec:fe_map}.


%% \section{Bilinear Lagrange element on a quadrilateral}
%% We now consider arguably the simplest basis function on quadrilaterals: bilinear Lagrange basis on a reference quadrilateral.  Our reference quadrilateral is a unit square that is delineated by vertices
%% \begin{equation*}
%%   x^1 = (0,0), \quad x^2 = (1,0), \quad x^3 = (0,1), \quad \text{and} \quad x^4 = (1,1).
%% \end{equation*}
%% In two dimensions, any bilinear function can be expressed as a linear combination of monomial basis $\{ 1, x_1, x_2, x_1 x_2 \}$, which, unlike the triangular case, includes the cross term. Our interpolation points are the four vertices of the quadrilateral $\{ x^1, x^2, x^3, x^4 \}$.  Our shape functions are given by 
%% \begin{equation}
%%   \phi_i(x) = a_1^i + a_2^i x_1 + a_3^i x_2 + a_4^i x_1 x_2, \quad i = 1,\dots,4,
%%   \label{eq:fe_lin_quad_rep}
%% \end{equation}
%% where the coefficients satisfy
%% \begin{equation*}
%%   \bmat{cccc}
%%   1 & x_1^1 & x_2^1 & x_1^1 x_2^1 \\
%%   1 & x_1^2 & x_2^2 & x_1^2 x_2^2 \\
%%   1 & x_1^3 & x_2^3 & x_1^3 x_2^3 \\
%%   1 & x_1^4 & x_2^4 & x_1^4 x_2^4 \\
%%   \emat
%%   \bmat{cccc}
%%   a_1^1 & a_1^2 & a_1^3 & a_1^4 \\
%%   a_2^1 & a_2^2 & a_2^3 & a_2^4 \\
%%   a_3^1 & a_3^2 & a_3^3 & a_3^4 \\
%%   a_4^1 & a_4^2 & a_4^3 & a_4^4 \\
%%   \emat
%%   =
%%   \bmat{cccc}
%%   1 & 0 & 0 & 0 \\
%%   0 & 1 & 0 & 0 \\
%%   0 & 0 & 1 & 0 \\
%%   0 & 0 & 0 & 1
%%   \emat.
%% \end{equation*}
%% Once we find the coefficients, we can evaluate the value of the shape function at any point in the quadrilateral by evaluating~\eqref{eq:fe_lin_quad_rep}. We can also differentiate~\eqref{eq:fe_lin_quad_rep} to obtain gradient of the shape functions:
%% \begin{equation*}
%%   \pp{\phi_i}{x_1}(x) = a_2^i + a_4^ix_2
%%   \quad \text{and} \quad
%%   \pp{\phi_i}{x_2}(x) = a_3^i + a_4^ix_1, \quad i = 1,\dots,4.
%% \end{equation*}
%% Unlike the linear shape functions for triangles, the gradient of the \emph{bi}linear shape functions for quadrilateral depends on the evaluation point.


%% Formally, a finite element is defined by a triplet $(K,\calP,\Sigma)$ where
%% \begin{itemize}
%% \item[(i)] $K$ defines the domain
%% \item[(ii)] $\calP$ defines the (finite-dimensional) linear space of functions over $K$
%% \item[(iii)] $\Sigma$ defines the degrees of freedom such that a function $v \in \calP$ is uniquely determined.
%% \end{itemize}
%% For instance, for the linear Lagrange element in Section~\ref{sec:fe_lin_tri} chooses (i) the triangle as the domain $K$, (ii) space of linear functions $\PP^1(K)$ as the function space $\calP$, and (iii) the values of the function at the vertices of the triangle as the degree of freedom $\Sigma$. 

%% In general, a Lagrange basis is uniquely determined by (i) the degree of polyno
