\chapter{Finite element method: theory}

\disclaimer

\section{Motivation}
In the previous lecture, we introduced a variational formulation and the associated finite element method (FEM) for a concrete one-dimensional model equation.  However, the true strength of the variational formulation and FEM lies in the abstraction, both in the context of formulation and analysis.  In this lecture we introduce an abstract framework to the FEM.





\section{Interpolation error: linear element in one dimension}
\label{sec:fe_interp_1d}
In this section, we analyze the error associated with the piecewise linear interpolation of functions in one dimension. By way of preliminary, we first provide the definition of \emph{interpolant}.
\begin{definition}[interpolant]
Given $w \in \calV$, an interpolant $\calI_h w$ is an element of $\calV_h$ that satisfies the interpolation condition
\begin{equation*}
  (\calI_h w)(x_i) = w(x_i) \quad i = 1,\dots, N,
\end{equation*}
where $\{x_i \}_{i=1}^N$ is the set of interpolation points.
\end{definition}

We now focus on the piecewise linear space in one dimension.  To this end, given $\Omega \subset \RR$, we introduce an approximation space
\begin{equation*}
  \calV_h = \{ v \in \calV \ | \ v|_K \in \PP^1(K), \ \forall K \in \calT_h \}.
\end{equation*}

\begin{lemma}[One-dimensional linear interpolation error bound for $K$]
  Let $K \equiv [a,b]$ be the domain of length $h \equiv b - a$, $w \in C^2(K)$ be a function we wish to interpolate, and $\calI_h w \in \PP^1(K)$ be the linear interpolant based on the interpolation points $\{a,b\}$. Then, the interpolation error satisfies
  \begin{align}
    \| w - \calI_h w \|_{L^2(K)} &\leq \frac{1}{2} h^{5/2} \| w'' \|_{L^\infty(K)} \label{eq:fe_interp_lin_l2_elem} \\
    | w - \calI_h w |_{H^1(K)} &\leq h^{3/2} \| w '' \|_{L^\infty(K)}. \label{eq:fe_interp_lin_h1_elem}
  \end{align}
  \begin{proof}
    We first introduce an auxiliary function
    \begin{equation*}
      g(s) \equiv (w - \calI_hw)(s) - \left(
      \frac{(w - \calI_h w)(x)}{(x - a)(x-b)}
      \right)(s - a)(s-b).
    \end{equation*}
  We note that $g(x) = g(a) = g(b) = 0$ by construction. Hence $g$ has at least three roots in $K \equiv[a,b]$.  By Rolle's theorem, $g'$ has at least two roots in $K$.  Invoking Rolle's theorem one more time, we conclude that $g''$ has at least one root in $K$; let $\xi \in K$ be one of the roots of $g''$: i.e., $g''(\xi) = 0$.  We now compute the second derivative of $g$:
  \begin{equation*}
    g''(s) = w''(s) - \left(
      \frac{(w - \calI_h w)(x)}{(x - a)(x-b)}
      \right) \cdot 2;
  \end{equation*}
  note that $(\calI_h w)'' = 0$ since $\calI_h w$ is a linear function.  We now evaluate the expression at $\xi$ to obtain
  \begin{equation*}
    0 = w''(\xi) - \left(
      \frac{(w - \calI_h w)(x)}{(x - a)(x-b)}
      \right) \cdot 2, \quad \forall x \in K
  \end{equation*}
  or, equivalently,
  \begin{equation*}
    (w - \calI_h w)(x) = \frac{1}{2} w''(\xi) (x - a)(x - b).
  \end{equation*}
  The $L^2$ error bound follows from
  \begin{align*}
    \| w - \calI_h w \|^2_{L^2(K)}
    &= \int_K (w - \calI_h w)^2 dx
    = \frac{1}{4} \int_K w''(\xi)^2 (x-a)^2 (x-b)^2 dx
    \\
    &\leq\frac{1}{4} \| w'' \|_{L^\infty(K)}^2 \int_K (x-a)^2(x-b)^2 dx
    \leq \frac{1}{4} h^5 \| w'' \|_{L^\infty(K)}^2,
  \end{align*}
  where the inequality follows from $|x-a| < h$ and $|x-b| < h$.  To obtain the $H^1$ error bound, we first note
  \begin{equation*}
    (w - \calI_hw)'(x) = \frac{1}{2} w''(\xi) ((x-a) + (x-b));
  \end{equation*}
  it thus follows
  \begin{align*}
    | w - \calI_h w |^2_{H^1(K)}
    &= \int_K ((w - \calI_h w)')^2 dx
    = \int_K w''(\xi)^2 \frac{1}{4} ((x-a) + (x-b))^2 dx
    \\
    &\leq \| w'' \|_{L^\infty(K)}^2 \int_K \frac{1}{4} ((x-a) + (x-b))^2 dx
    \leq h^3\| w'' \|_{L^2(K)}^2,
  \end{align*}
  where the inequality again follows from $|x - a| < h$ and $|x - b| < h$.
    \end{proof}
\end{lemma}

\begin{proposition}[One-dimensional linear interpolation error bound for $\Omega$]
  Let $\Omega \subset \RR^1$ be the domain, $\calT_h$ be a uniform triangulation over $\Omega$ of characteristic length $h$, $w \in \oplus_{K \in \calT_h}  C^2(K)$ be a function we wish to interpolate, and $\calI_h w \in \calV_h$ be the linear interpolant associated with $\calV_h \equiv \{ v \in C^0(\Omega) \ | \ v|_K \in \PP^1(K), \ \forall K \in \calT_h \}$. Then, the interpolation error satisfies
  \begin{align}
    \| w - \calI_h w \|_{L^2(\Omega)} &\leq \frac{1}{2} h^2 \| w'' \|_{L^\infty(\Omega)} \label{eq:fe_interp_lin_l2} \\
    | w - \calI_h w |_{H^1(\Omega)} &\leq h \| w'' \|_{L^\infty(\Omega)} \label{eq:fe_interp_lin_h1}
  \end{align}
  \begin{proof}
    The $L^2$ error bound follows from the application of~\eqref{eq:fe_interp_lin_l2_elem} to each element:
    \begin{equation*}
      \| w - \calI_h w \|^2_{L^2(\Omega)}
      =
      \sum_{K \in \calT_h} \| w - \calI_h w \|^2_{L^2(K)}
      \leq
      \frac{1}{h} \frac{1}{4} h^5 \| w'' \|_{L^\infty(\Omega)}^2
      = \frac{1}{4} h^4 \| w'' \|_{L^\infty(\Omega)}^2.
    \end{equation*}
    The $H^1$ error bound similarly follows from the application of~\eqref{eq:fe_interp_lin_h1_elem} to each element:
        \begin{equation*}
      \| w - \calI_h w \|^2_{L^2(\Omega)}
      =
      \sum_{K \in \calT_h} \| w - \calI_h w \|^2_{L^2(K)}
      \leq
      \frac{1}{h} h^3 \| w'' \|_{L^\infty(\Omega)}^2
      = h^2 \| w'' \|_{L^\infty(\Omega)}^2.
    \end{equation*}
  \end{proof}
\end{proposition}
The proposition shows that the $L^2$ interpolation error (i) depends on the maximum value of the second derivative $\| w '' \|_{L^\infty(\Omega)}$ and (ii) decreases as $h^2$.  The $H^1$ interpolation error similarly depends on $\| w'' \|_{L^\infty(\Omega)}$ but decreases as $h^1$. 

\section{Interpolation error: general polynomial interpolant}
The interpolation error bound obtained in Section~\ref{sec:fe_interp_1d} can be generalized to (i) higher dimensions, (ii) higher degree polynomials, and (iii) $H^k(\Omega)$ norm for $k \geq 0$.  However, the associated proof, which builds on the Bramble-Hilbert lemma, is beyond the scope of this lecture.  We here simply state the result.
\begin{proposition}
Let $w \in H^s(\Omega)$ be a function we wish to interpolate, $\calT_h$ be a triangulation of a characteristic diameter $h$, $\calI_h w \in \calV_h$ be the piecewise polynomial interpolant of degree $p$ associated with $\calT_h$. Then, the $L^2(\Omega)$ interpolation error satisfies
\begin{align*}
  \| w - \calI_h w \|_{L^2(\Omega)} \leq C h^{r+1} | w |_{H^{r+1}(\Omega)}
\end{align*}
for $r = \min\{ s,p \}$ and some $C$ independent of $h$. Similarly, for $k \geq 0$, the $H^k(\Omega)$ interpolation error satisfies 
\begin{align*}
  \| w - \calI_h w \|_{H^k(\Omega)} \leq C h^{r+1-k} | w |_{H^{r+1}(\Omega)}
\end{align*}
for $r = \min\{ s,p \}$ and some $C$ independent of $h$.
\end{proposition}


\section{Finite element error: energy norm}
In this section and the following sections, we analyze the error in the finite element solution in various norms. By way of preliminaries, we define conditions that are referenced throughout our analysis:
\begin{assumption}[Assumptions on forms]
  We make the following assumptions:
\begin{enumerate}
  \item $\calV \subset H^1(\Omega)$
  \item $a(\cdot,\cdot)$ is a bilinear form on $\calV$ (but not necessarily symmetric)
  \item $a(\cdot,\cdot)$ is $\calV$-continuous with the continuity constant $\gamma$
  \item $a(\cdot,\cdot)$ is $\calV$-coercive with the coercivity constant $\alpha$
  \item $\ell(\cdot)$ is a linear form on $\calV$
  \item $\ell(\cdot)$ is continuous on $\calV$
\end{enumerate}
\end{assumption}


we define the exact variational problem: find $u \in \calV$ such that
\begin{equation}
  a(u,v) = \ell(v) \quad \forall v \in \calV. \label{eq:th_pde}
\end{equation}
We also define the finite element approximation problem: find $u_h \in \calV_h$ such that
\begin{equation}
  a(u_h,v) = \ell(v) \quad \forall v \in \calV_h. \label{eq:th_fe}
\end{equation}
We first analyze the error $e \equiv u - u_h$ in the \emph{energy norm}. 
\begin{definition}[energy norm]
  Given a symmetric, coercive, and continuous bilinear form $a: \calV \times \calV \to \RR$, the energy norm $\enorm{\cdot} : \calV \to \RR_{\geq 0}$ is defined by
  \begin{equation*}
    \enorm{v} \equiv \sqrt{a(v,v)} \quad \forall v \in \calV.
  \end{equation*}
\end{definition}
The energy norm is equivalent to $\| \cdot \|_{H^1(\Omega)}$ in the sense there exists $c > 0$ and $C < \infty$ such that
\begin{equation*}
  c \| v \|_{H^1(\Omega)} \leq \enorm{v} \leq C \| v \|_{H^1(\Omega)} \quad \forall v \in \calV;
\end{equation*}
we in fact recognize that we may set $c$ and $C$ equal to the coercivity constant $\alpha$ and the continuity constant $\gamma$, respectively.

\begin{lemma}[Galerkin orthogonality]
  Let $u \in \calV$ and $u_h \in \calV_h$ be the solutions to~\eqref{eq:th_pde} and \eqref{eq:th_fe}, respectively.  Then, the error $u - u_h \in \calV$ satisfies 
  \begin{equation*}
    a(u - u_h, v) = 0 \quad \forall v \in \calV_h.
  \end{equation*}
  \begin{proof}
    Condition~\eqref{eq:th_pde} implies $a(u,v) = \ell(v)$, $\forall v \in \calV_h \subset \calV$.  The subtraction of \eqref{eq:th_fe} from the relationship yields
    \begin{equation*}
      a(u - u_h, v) = a(u,v) - a(u_h,v) = \ell(v) - \ell(v) = 0 \quad \forall v \in \calV_h,
    \end{equation*}
    which is the desired relationship.
  \end{proof}
\end{lemma}

The following theorem states that the finite element solution is optimal in the energy norm.
\begin{theorem}[energy-norm error optimality]
  Let $a: \calV \times \calV \to \RR$ be a symmetric, coercive, and continuous bilinear form, $\enorm{\cdot}$ be the associated energy norm, and $u \in \calV$ be $u_h \in \calV_h$ be the solutions to~\eqref{eq:th_pde} and \eqref{eq:th_fe}, respectively. Then, the finite element solution is optimal in the energy norm in the sense that
  \begin{equation*}
   \enorm{ u - u_h } = \inf_{w_h \in \calV_h} \enorm{ u - w_h }.
  \end{equation*}
  \begin{proof}
    Let $w_h$ be an arbitrary element in $\calV_h$ and express it as $w_h = u_h + v_h$ for $v_h \in \calV_h$.  Then,
\begin{align*}
  \enorm{u - w_h}^2 &= \enorm{u - u_h - v_h}^2
  = a(u - u_h - v_h, u - u_h - v_h) \\
  &= a(u - u_h, u - u_h) - 2 \underbrace{ a(u - u_h, v_h) }_{= 0 \text{ by Galerkin orthogonality}} + \underbrace{ a(v_h,v_h) }_{> 0 \text{ for $v_h \neq 0$ by coercivity}}
  \\
  &> \enorm{u - u_h}^2 \quad \forall v_h \neq 0 \quad \text{ or, equivalently, $\forall w_h \neq u_h$}.
\end{align*}
  \end{proof}
\end{theorem}
In words, the optimality of the finite element error in the energy norm implies the following: even \emph{if} we knew the exact solution $u \in \calV$ to~\eqref{eq:th_pde}, we could not find a $w_h \in \calV_h$ that is more accurate than $u_h \in \calV_h$. This optimality result is a direct consequence of Galerkin orthogonality, which states that the error $u - u_h \in \calV$ is orthogonal to the space $\calV_h$ in the inner product associated with the bilinear form $a$.

\section{Finite element error: $H^1$ norm}

\begin{lemma}[Cea's]
  Let $a: \calV \times \calV \to \RR$ be a symmetric, coercive, and continuous bilinear form, and $u \in \calV$ be $u_h \in \calV_h$ be the solutions to~\eqref{eq:th_pde} and \eqref{eq:th_fe}, respectively. Then, the error $u - u_h$ satisfies
  \begin{equation*}
    \| u - u_h \|_\calV \leq \frac{\gamma}{\alpha} \inf_{w_h \in \calV_h} \| u - w_h \|_\calV,
  \end{equation*}
  for the coercivity constant $\alpha$ and continuity constant $\gamma$ defined in Definitions~\ref{def:th_coercivity} and \ref{def:th_continuity}, respectively.
  \begin{proof}
    We first observe
    \begin{align*}
      \alpha \| u - u_h \|^2_\calV
      &\leq a(u - u_h, u - u_h) & \text{(coercivity)} \\
      &= a(u - u_h, u - w_h) + a(u - u_h, w_h - u_h) &\text{(bilinearity)} \\
      &= a(u - u_h, u - w_h) &\text{(Galerkin orthogonality)} \\
      &\leq \gamma \| u - u_h \|_\calV \| u - w_h \|_\calV &\text{(continuity)}.
    \end{align*}
    The division by $\| u - u_h \|_\calV$ yields the desired result.
  \end{proof}
\end{lemma}


\section{Finite element error: $L^2$ norm}

\begin{lemma}[elliptic regularity estimate]
  Let $a: \calV \times \calV \to \RR$ be a continuous, coercive bilinear form and $u$ be the solution to 
\end{lemma}

\begin{theorem}[$L^2$ error bound]
  
\begin{align*}
  \| u - u_h \|_{L^2(\Omega)} \leq C h \| u-u_h \|_\calV
\end{align*}

\begin{proof}
  The proof by so-called \emph{Aubin-Nitsche trick}. we first pose a dual problem: find $\psi \in \calV$ such that
\begin{equation*}
  a(w,\psi) = (w,e)_{L^2(\Omega)} \quad \forall w \in \calV
\end{equation*}
for $e \equiv u - u_h$. We then observe that
\begin{equation*}
  \| e \|_{L^2(\Omega)}^2 = a(e, \psi)
  = a(e, \psi - \Pi_{H^1(\Omega)} \psi)
  \leq C \| e \|_\calV \| \psi - \calI_h \psi \|_\calV
\end{equation*}
We note that, since $u \in H^1(\Omega)$ and $u_h \in H^1(\Omega)$,  $e \equiv u - u_h \in \calV \subset H^1(\Omega) \subset L^2(\Omega)$; by the elliptic regularity estimate, $\| \psi \|_{H^2(\Omega)} \leq C \| e \|_{L^2(\Omega)}$. It hence follows that
\begin{align*}
  \| \psi - \calI_h \psi \|_\calV
  &=\| \psi - \calI_h \psi \|_{H^1(\Omega)} & \text{(definition of $\calV$ norm)} \\
  &\leq C h \| \psi \|_{H^2(\Omega)} & \text{(interpolation error bound)}\\
  &\leq C' h  \| e \|_{L^2(\Omega)} & \text{(elliptic regularity estimate)}
\end{align*}
\end{proof}
\end{theorem}

\begin{corollary}
  \begin{equation*}
    \| u - u_h \|_{L^2(\Omega)} \leq C h^{s+1} \| u \|_{H^{s+1}(\Omega)}
  \end{equation*}
\end{corollary}

\section{Finite element error: functional output}
