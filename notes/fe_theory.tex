\chapter{Finite element method: theory}

\disclaimer

\section{Motivation}
We have so far discussed formulation and implementation of the finite element method.  We now study error associated with finite element methods.



%In this section and the following sections, we analyze the error in the finite element solution in various norms. By way of preliminaries, we define conditions that are referenced throughout our analysis:
%\begin{assumption}[Assumptions on forms]
%  We make the following assumptions:
%\begin{enumerate}
%  \item $\calV \subset H^1(\Omega)$
%  \item $a(\cdot,\cdot)$ is a bilinear form on $\calV$ (but not necessarily symmetric)
%  \item $a(\cdot,\cdot)$ is $\calV$-continuous with the continuity constant $\gamma$
%  \item $a(\cdot,\cdot)$ is $\calV$-coercive with the coercivity constant $\alpha$
%  \item $\ell(\cdot)$ is a linear form on $\calV$
%  \item $\ell(\cdot)$ is continuous on $\calV$
%\end{enumerate}
%\end{assumption}
\section{Finite element error analysis: preliminary}
In this section and the following sections, we analyze the error in the finite element approximations in various norms. By way of preliminaries, we introduce a Hilbert space $\calV$ on $\Omega \subset \RR^d$ such that $H^1_0(\Omega) \subset \calV \subset H^1(\Omega)$; the space $\calV$ is endowed with an inner product $(\cdot,\cdot)_\calV$ and the associated induced norm $\| \cdot \|_\calV$. We next introduce the variational problem: find $u \in \calV$ such that
\begin{equation}
  a(u,v) = \ell(v) \quad \forall v \in \calV. \label{eq:th_pde}
\end{equation}
We then introduce the finite element approximation problem: find $u_h \in \calV_h$ such that
\begin{equation}
  a(u_h,v) = \ell(v) \quad \forall v \in \calV_h. \label{eq:th_fe}
\end{equation}
For all error estimates, we assume that the linear form $\ell:\calV \to \RR$ is continuous in $\calV$.  The requirement for the bilinear form $a:\calV \times \calV \to \RR$ varies with the particular form of error estimate; we will clearly state the assumptions for each estimate.

We now introduce \emph{Galerkin orthogonality}, a relationship that will be used throughout our analysis of error in finite element approximations.
\begin{lemma}[Galerkin orthogonality]
  Let $u \in \calV$ and $u_h \in \calV_h$ be the solutions to~\eqref{eq:th_pde} and \eqref{eq:th_fe}, respectively.  Then, the error $u - u_h \in \calV$ satisfies 
  \begin{equation*}
    a(u - u_h, v) = 0 \quad \forall v \in \calV_h.
  \end{equation*}
  \begin{proof}
    Condition~\eqref{eq:th_pde} implies $a(u,v) = \ell(v)$, $\forall v \in \calV_h \subset \calV$.  The subtraction of \eqref{eq:th_fe} from the relationship yields
    \begin{equation*}
      a(u - u_h, v) = a(u,v) - a(u_h,v) = \ell(v) - \ell(v) = 0 \quad \forall v \in \calV_h,
    \end{equation*}
    which is the desired relationship.
  \end{proof}
\end{lemma}


\section{Finite element error: energy norm}
In this section we consider a symmetric, coercive bilinear form and assess our error in \emph{energy norm}.
\begin{definition}[energy norm]
  Given a symmetric, coercive, and continuous bilinear form $a: \calV \times \calV \to \RR$, the energy norm $\enorm{\cdot} : \calV \to \RR_{\geq 0}$ is defined by
  \begin{equation*}
    \enorm{v} \equiv \sqrt{a(v,v)} \quad \forall v \in \calV.
  \end{equation*}
\end{definition}
Because the bilinear form is symmetric and coercive, the bilinear form $a(\cdot,\cdot)$ is in fact an inner product that satisfies the requirement on the (i) linearity, (ii) symmetry, and (iii) Cauchy-Shwarz inequality.  The energy norm is the induced norm associated with this inner product; the norm hence satisfies the requirement on the (i) linearity, (ii) positivity, and (iii) triangle inequality.

The energy norm is \emph{equivalent} to $\| \cdot \|_{H^1(\Omega)}$ in the following sense.
\begin{definition}[equivalence of norms]
  Given a Hilbert space $\calV$, a norm $\| \cdot \|_A$ is said to be equivalent to a norm $\| \cdot \|_B$ if there exists $c > 0$ and $C < \infty$ such that
  \begin{equation*}
    c \| v \|_{B} \leq \| v \|_A \leq C \| v \|_{B} \quad \forall v \in \calV.
  \end{equation*}
\end{definition}
\begin{lemma}[equivalence of energy and $H^1$ norm]
  Given a symmetric, coercive, continuous bilinear form $a:\calV \times \calV \to \RR$, the energy norm $\enorm{\cdot}$ is equivalent to the $H^1$ norm $\| \cdot \|_{H^1(\Omega)}$.
  \begin{proof}
    From coercivity and continuity of $a(\cdot,cdot)$ in $\calV$, we immediately obtain
    \begin{equation*}
      \alpha \| v \|_\calV^2 \leq a(v,v) \equiv \enorm{v}^2 \leq \gamma \| v \|_\calV^2 \quad \forall v \in \calV,
    \end{equation*}
    where $\alpha > 0$ and $\gamma < \infty$ are the coercivity and continuity constants, respectively. Taking the square root of the inequality yields the desired relationship.
  \end{proof}
\end{lemma}

We now show that the finite element approximation is optimal in energy norm.
\begin{proposition}[energy-error bound]
  \label{prop:th_energy_bound}
  Let $a: \calV \times \calV \to \RR$ be a symmetric, coercive, and continuous bilinear form, $\enorm{\cdot}$ be the associated energy norm, and $u \in \calV$ and $u_h \in \calV_h$ be the solutions to~\eqref{eq:th_pde} and \eqref{eq:th_fe}, respectively. Then, the finite element approximation is optimal in the energy norm in the sense that
  \begin{equation}
    \enorm{ u - u_h } = \inf_{w_h \in \calV_h} \enorm{ u - w_h }.
    \label{eq:th_energy_bound}
  \end{equation}
  \begin{proof}
    Let $w_h$ be an arbitrary element in $\calV_h$ and express it as $w_h = u_h + v_h$ for $v_h \in \calV_h$.  Then,
\begin{align*}
  \enorm{u - w_h}^2 &= \enorm{u - u_h - v_h}^2
  = a(u - u_h - v_h, u - u_h - v_h) \\
  &= a(u - u_h, u - u_h) - 2 \underbrace{ a(u - u_h, v_h) }_{= 0 \text{ by Galerkin orthogonality}} + \underbrace{ a(v_h,v_h) }_{> 0 \text{ for $v_h \neq 0$ by coercivity}}
  \\
  &> \enorm{u - u_h}^2 \quad \forall v_h \neq 0, % \quad \text{ or, equivalently, $\forall w_h \neq u_h$}.
\end{align*}
or, equivalently, $\forall w_h \neq u_h$.
  \end{proof}
\end{proposition}
In words, the optimality of the finite element error in the energy norm implies the following: even \emph{if} we knew the exact solution $u \in \calV$ to~\eqref{eq:th_pde}, we could not find a $w_h \in \calV_h$ that is more accurate than $u_h \in \calV_h$. This optimality result is a direct consequence of Galerkin orthogonality, which states that the error $u - u_h \in \calV$ is orthogonal to the space $\calV_h$ in the inner product associated with the bilinear form $a: \calV \times \calV \to \RR$.

We may obtain a particular result for finite element approximations based on piecewise polynomial spaces.
\begin{proposition}(energy-error $h$-convergence)
  \label{prop:th_energy_bound_poly}
  Consider a finite element approximation associated with $\PP^p$ approximation space, and suppose $u \in H^{p+1}(\Omega)$.  Then, the energy norm of the error is bounded by
  \begin{equation*}
    \enorm{u - u_h} \leq C h^p | u |_{H^{p+1}(\Omega)},
  \end{equation*}
  for some constant $C$ independent of $h$.
  \begin{proof}
    We invoke the energy-error bound in Proposition~\eqref{prop:th_energy_bound}, set $w_h$ to be the polynomial interpolant of $u$, $\calI_h u \in \calV_h$, and invoke the polynomial interpolation error bound in Proposition \ref{prop:th_interp_gen}, to obtain
    \begin{align*}
      \enorm{ u - u_h } &= \inf_{w_h \in \calV_h} \enorm{ u - w_h }
      & \text{(energy-error bound \eqref{eq:th_energy_bound})}  \\
      &\leq \enorm{ u - \calI_h u } &\text{($w_h = \calI_h u$)} \\
      &\leq \gamma \| u - \calI_h u \|_{H^1(\Omega)} &\text{(continuity of $a(\cdot,\cdot)$)} \\
      &\leq \gamma C_\calI | u |_{H^{p+1}(\Omega)}. &\text{(interpolation error bound)}
    \end{align*}
    We set $C \equiv \gamma C_\calI$ to obtain the desired relationship.
  \end{proof}
\end{proposition}

\section{Finite element error: $\calV$ norm}
In this section we obtain error bound in the $\calV$ norm.  We assume $H^1_0(\Omega) \subset \calV \subset H^1(\Omega)$, and $\| \cdot \|_\calV$ is equivalent to $\| \cdot \|_{H^1(\Omega)}$.  The first bound is for a variational problem with a symmetric, coercive bilinear form.
\begin{lemma}[Cea's lemma (symmetric)]
  \label{lemma:th_cea_sym}
  Let $a: \calV \times \calV \to \RR$ be a symmetric, coercive, and continuous bilinear form and $u \in \calV$ be $u_h \in \calV_h$ be the solutions to~\eqref{eq:th_pde} and \eqref{eq:th_fe}, respectively. Then, the error $u - u_h$ satisfies
  \begin{equation}
    \| u - u_h \|_\calV \leq \sqrt{\frac{\gamma}{\alpha}} \inf_{w_h \in \calV_h} \| u - w_h \|_\calV,
    \label{eq:th_cea_sym}
  \end{equation}
  for the coercivity constant $\alpha$ and the continuity constant $\gamma$
  \begin{proof}
    The bound is a consequence of the energy-error optimality:
    \begin{align*}
      \alpha \| u - u_h \|^2_\calV
      &\leq a(u - u_h, u - u_h) & \text{(coercivity)} \\
      &\leq \enorm{ u - u_h }^2 & \text{(energy norm)} \\
      & = \inf_{w_h \in \calV_h} \enorm{ u - u_h }^2 & \text{(energy-error optimality)} \\
      & = \gamma \| u - u_h \|^2_\calV & \text{(continuity)}
    \end{align*}
    Taking the square root yields the desired inequality.
  \end{proof}
\end{lemma}

The second bound is for a variational problem with a bilinear form that is coercive but not necessarily symmetric.
\begin{lemma}[Cea's lemma (nonsymmetric)]
  \label{lemma:th_cea_nonsym}
  Let $a: \calV \times \calV \to \RR$ be a coercive and continuous (but not necessarily symmetric) bilinear form, and $u \in \calV$ and $u_h \in \calV_h$ be the solutions to~\eqref{eq:th_pde} and \eqref{eq:th_fe}, respectively. Then, the error $u - u_h$ satisfies
  \begin{equation}
    \| u - u_h \|_\calV \leq \frac{\gamma}{\alpha} \inf_{w_h \in \calV_h} \| u - w_h \|_\calV,
    \label{eq:th_cea}
  \end{equation}
  for the coercivity constant $\alpha$ and the continuity constant $\gamma$.
  \begin{proof}
    We observe
    \begin{align*}
      \alpha \| u - u_h \|^2_\calV
      &\leq a(u - u_h, u - u_h) & \text{(coercivity)} \\
      &= a(u - u_h, u - w_h) + a(u - u_h, w_h - u_h) &\text{(bilinearity)} \\
      &= a(u - u_h, u - w_h) &\text{(Galerkin orthogonality)} \\
      &\leq \gamma \| u - u_h \|_\calV \| u - w_h \|_\calV &\text{(continuity)}.
    \end{align*}
    The division by $\| u - u_h \|_\calV$ yields the desired result.
  \end{proof}
\end{lemma}
Because $\gamma/\alpha \geq 1$ by the definition of the continuity and coercivity constants, the bound~\eqref{eq:th_cea} for nonsymmetric bilinear forms is looser than the bound~\eqref{eq:th_cea_sym} for symmetric bilinear forms.  In both cases, we observe that the finite element approximation is quasi-optimal in the sense that $\| u - u_h \|_\calV$ is at most a constant multiple of the best-fit error $\inf_{w_h \in \calV_h} \| u - w_h \|_\calV$, where the constant is independent of $h$.

Given these quasi-optimality results, we can readily obtain particular results for finite element approximations based on piecewise polynomial spaces.
\begin{proposition}[$\calV$-norm convergence]
  Consider the same setting as Lemma~\ref{lemma:th_cea_nonsym} and suppose $u \in H^{p+1}(\Omega)$.  Then, 
  \begin{equation}
    \| u - u_h \|_\calV \leq C h^{p} | u |_{H^{p+1}(\Omega)}
  \end{equation}
  for some $C < \infty$ independent of $u$ and $h$. Moreover,
  \begin{equation}
    \| u - u_h \|_{H^1(\Omega)} \leq C h^{p} | u |_{H^{p+1}(\Omega)}
  \end{equation}
  for some $C < \infty$ independent of $u$ and $h$.
  \begin{proof}
    We invoke Cea's lemma~\ref{lemma:th_cea_sym}, the equivalence of $\| \cdot \|_{\calV}$ and $\| \cdot \|_{H^1(\Omega)}$, an the polynomial interpolation error bound in Proposition~\ref{prop:th_interp_gen}.
    The second result follows from the equivalence of $\| \cdot \|_{\calV}$ and $\| \cdot \|_{H^1(\Omega)}$.
  \end{proof}
\end{proposition}



\section{Finite element error: $L^2$ norm}
We now analyze the convergence of finite element approximations in $L^2(\Omega)$ norm.  Unfortunately, the $L^2(\Omega)$ error analysis relies on an equation-specific result called the \emph{elliptic regularity estimate}.  Hence, in this section, unlike in the previous sections, we restrict ourselves to (variable coefficients) advection-reaction-diffusion equation.

\begin{lemma}[elliptic regularity estimate]
  \label{lemma:th_elliptic_reg}
  Let $\Omega \subset \RR^d$ be a Lipschitz domain, $\calV$ be a Hilbert space such that $H_0^1(\Omega) \subset \calV \subset H^1(\Omega)$, and let $a: \calV \times \calV \to \RR$ be
  \begin{equation*}
    a(w,v) = \int_\Omega (\nabla v \cdot a \nabla w + v b \cdot \nabla w + c v w) dx,  \quad \forall w, v \in \calV,
  \end{equation*}
  for $a \in C^{1}(\bar \Omega)^{d \times d}$ and elliptic, $b \in C^0(\bar \Omega)^d$, $c \in C^0(\bar \Omega)$.  Then, the solution to the weak problem, find $u \in \calV$ such that
  \begin{equation*}
    a(u,v) = (f,v)_{L^2(\Omega)} \quad \forall v \in \calV,
  \end{equation*}
  satisfies
  \begin{equation*}
    \| u \|_{H^2(\Omega)} \leq C_{\rm reg} \| f \|_{L^2(\Omega)}
  \end{equation*}
  for some $C_{\rm reg} < \infty$.
  \begin{proof}
    Proof is beyond the scope of this course.  See, e.g., Ern and Guermond, \emph{Theory and practice of finite element method}, Theorem 3.10.
  \end{proof}
\end{lemma}
\begin{theorem}[$L^2$ error bound (Aubin-Nitsche)]
  Suppose the conditions of the elliptic regularity estimate, Lemma~\ref{lemma:th_elliptic_reg}, are satisfied, and $u \in \calV$ and $u_h \in \calV_h$ are the solutions to~\eqref{eq:th_pde} and \eqref{eq:th_fe}, respectively. Then, 
  \begin{equation*}
    \| u - u_h \|_{L^2(\Omega)} \leq C h \| u-u_h \|_\calV,
  \end{equation*}
  for $C < \infty$ independent of $h$.
  \begin{proof}
    The proof by so-called \emph{Aubin-Nitsche trick}. We first pose a dual problem: find $\psi \in \calV$ such that
    \begin{equation*}
      a(w,\psi) = (w,e)_{L^2(\Omega)} \quad \forall w \in \calV
    \end{equation*}
    for $e \equiv u - u_h$. We then observe that
    \begin{equation*}
      \| e \|_{L^2(\Omega)}^2 = a(e, \psi)
      = a(e, \psi - \Pi_{H^1(\Omega)} \psi)
      \leq \gamma \| e \|_\calV \| \psi - \calI_h \psi \|_\calV
    \end{equation*}
    We note that, since $u \in H^1(\Omega)$ and $u_h \in H^1(\Omega)$,  $e \equiv u - u_h \in \calV \subset H^1(\Omega) \subset L^2(\Omega)$; by the elliptic regularity estimate, $\| \psi \|_{H^2(\Omega)} \leq C \| e \|_{L^2(\Omega)}$. It hence follows that
    \begin{align*}
      \| \psi - \calI_h \psi \|_\calV
      &=\| \psi - \calI_h \psi \|_{H^1(\Omega)} & \text{(definition of $\calV$ norm)} \\
      &\leq C_\calI h \| \psi \|_{H^2(\Omega)} & \text{(interpolation error bound)}\\
      &\leq C_\calI C_{\rm reg} h  \| e \|_{L^2(\Omega)} & \text{(elliptic regularity estimate)}
    \end{align*}
  \end{proof}
\end{theorem}

\begin{corollary}
  
  \begin{equation*}
    \| u - u_h \|_{L^2(\Omega)} \leq C h^{p+1} \| u \|_{H^{p+1}(\Omega)}
  \end{equation*}
\end{corollary}

\section{Finite element error: functional output}
In this section we consider the error in an \emph{output} or \emph{quantity of interest}.  We assume that the output is defined by a linear functional $\ell^o \in \calV'$ and that the functional is continuous in $\calV$: $\exists c < \infty$ such that $|\ell^o(w)| \leq c \| w \|_\calV$ $\forall w \in \calV$. 
\begin{proposition}[output error bound (symmetric)]
  \label{prop:th_output_sym}
  Let $a: \calV \times \calV \to \RR$ be a symmetric, coercive, continuous bilinear form, $u \in \calV$ and $u_h \in \calV_h$ be the solutions to~\eqref{eq:th_pde} and \eqref{eq:th_fe}, respectively, and $\ell^o: \calV \to \RR$ be a linear functional. Then, the error in the functional output is bounded by 
\begin{equation*}
  |\ell^o(u) - \ell^o(u_h)| \leq
  \inf_{w_h \in \calV_h} \enorm{u - w_h} \inf_{v_h \in \calV_h} \enorm{\psi - v_h}
\end{equation*}
\begin{proof}
  We observe that, $\forall v_h \in \calV_h$, 
\begin{align*}
  |\ell^o(u) - \ell^o(u_h)|
  &= |\ell^o(u - u_h)| & \text{(linearity of $\ell^o$)} \\
  &= |a(u-u_h,\psi)| & \text{(definition of adjoint $\psi$)} \\
  &= |a(u-u_h,\psi-v_h)| & \text{(Galerkin orthogonality)} \\
  &\leq \enorm{ u - u_h }\enorm{ \psi - v_h } & \text{(Cauchy-Schwarz)} \\
  &\leq \inf_{w_h \in \calV_h} \enorm{ u - w_h }\enorm{ \psi - v_h }. & \text{(energy-error optimality of $u_h$)}
\end{align*}
We then take $v_h \in \calV_h$ to be the minimizer of $\enorm{\psi - v_h}$.
\end{proof}
\end{proposition}

\begin{proposition}[output error bound (nonsymmetric)]
  \label{prop:th_output_nonsym}
  Let $a: \calV \times \calV \to \RR$ be a coercive, continuous (but not necessarily symmetric) bilinear form, $u \in \calV$ and $u_h \in \calV_h$ be the solutions to~\eqref{eq:th_pde} and \eqref{eq:th_fe}, respectively, and $\ell^o: \calV \to \RR$ be a linear functional. Then, the error in the functional output is bounded by 
\begin{equation*}
  |\ell^o(u) - \ell^o(u_h)|
  \leq \frac{\gamma^2}{\alpha} \inf_{w_h \in \calV_h} \| u - w_h \|_\calV \inf_{v_h \in \calV_h} \| \psi - v_h \|_\calV,
\end{equation*}
where $\alpha$ and $\gamma$ are the coercivity and continuity constants of $a : \calV \times \calV \to \RR$, respectively.
\begin{proof}
  We observe that, $\forall v_h \in \calV_h$, 
  \begin{align*}
    |\ell^o(u) - \ell^o(u_h)|
    &= |\ell^o(u - u_h)| & \text{(linearity of $\ell^o$)} \\
    &= |a(u-u_h,\psi)| & \text{(definition of adjoint $\psi$)} \\
    &= |a(u-u_h,\psi-v_h)| & \text{(Galerkin orthogonality)} \\
    &\leq \gamma \| u - u_h \|_\calV \| \psi - v_h \|_\calV & \text{(continuity of $a(\cdot,\cdot)$)} \\
    &\leq \frac{\gamma^2}{\alpha} \inf_{w_h \in \calV_h} \| u - w_h \|_\calV \| \psi - v_h \|_\calV. & \text{($\| \cdot \|_\calV$ error bound of $u_h$)}
  \end{align*}
  We then take $v_h \in \calV_h$ to be the minimizer of $\enorm{\psi - v_h}$.
\end{proof}
\end{proposition}

\begin{remark}[output superconvergence]
  We observe in both Propositions~\ref{prop:th_output_sym} and \ref{prop:th_output_nonsym} that the output converges faster than the 
\end{remark}
%% We now focus on the piecewise linear space in one dimension.  To this end, given $\Omega \subset \RR$, we introduce an approximation space
%% \begin{equation*}
%%   \calV_h = \{ v \in \calV \ | \ v|_K \in \PP^1(K), \ \forall K \in \calT_h \}.
%% \end{equation*}

%% \begin{lemma}[One-dimensional linear interpolation error bound for $K$]
%%   Let $K \equiv [a,b]$ be the domain of length $h \equiv b - a$, $w \in C^2(K)$ be a function we wish to interpolate, and $\calI_h w \in \PP^1(K)$ be the linear interpolant based on the interpolation points $\{a,b\}$. Then, the interpolation error satisfies
%%   \begin{align}
%%     \| w - \calI_h w \|_{L^2(K)} &\leq \frac{1}{2} h^{5/2} \| w'' \|_{L^\infty(K)} \label{eq:fe_interp_lin_l2_elem} \\
%%     | w - \calI_h w |_{H^1(K)} &\leq h^{3/2} \| w '' \|_{L^\infty(K)}. \label{eq:fe_interp_lin_h1_elem}
%%   \end{align}
%%   \begin{proof}
%%     We first introduce an auxiliary function
%%     \begin{equation*}
%%       g(s) \equiv (w - \calI_hw)(s) - \left(
%%       \frac{(w - \calI_h w)(x)}{(x - a)(x-b)}
%%       \right)(s - a)(s-b).
%%     \end{equation*}
%%   We note that $g(x) = g(a) = g(b) = 0$ by construction. Hence $g$ has at least three roots in $K \equiv[a,b]$.  By Rolle's theorem, $g'$ has at least two roots in $K$.  Invoking Rolle's theorem one more time, we conclude that $g''$ has at least one root in $K$; let $\xi \in K$ be one of the roots of $g''$: i.e., $g''(\xi) = 0$.  We now compute the second derivative of $g$:
%%   \begin{equation*}
%%     g''(s) = w''(s) - \left(
%%       \frac{(w - \calI_h w)(x)}{(x - a)(x-b)}
%%       \right) \cdot 2;
%%   \end{equation*}
%%   note that $(\calI_h w)'' = 0$ since $\calI_h w$ is a linear function.  We now evaluate the expression at $\xi$ to obtain
%%   \begin{equation*}
%%     0 = w''(\xi) - \left(
%%       \frac{(w - \calI_h w)(x)}{(x - a)(x-b)}
%%       \right) \cdot 2, \quad \forall x \in K
%%   \end{equation*}
%%   or, equivalently,
%%   \begin{equation*}
%%     (w - \calI_h w)(x) = \frac{1}{2} w''(\xi) (x - a)(x - b).
%%   \end{equation*}
%%   The $L^2$ error bound follows from
%%   \begin{align*}
%%     \| w - \calI_h w \|^2_{L^2(K)}
%%     &= \int_K (w - \calI_h w)^2 dx
%%     = \frac{1}{4} \int_K w''(\xi)^2 (x-a)^2 (x-b)^2 dx
%%     \\
%%     &\leq\frac{1}{4} \| w'' \|_{L^\infty(K)}^2 \int_K (x-a)^2(x-b)^2 dx
%%     \leq \frac{1}{4} h^5 \| w'' \|_{L^\infty(K)}^2,
%%   \end{align*}
%%   where the inequality follows from $|x-a| < h$ and $|x-b| < h$.  To obtain the $H^1$ error bound, we first note
%%   \begin{equation*}
%%     (w - \calI_hw)'(x) = \frac{1}{2} w''(\xi) ((x-a) + (x-b));
%%   \end{equation*}
%%   it thus follows
%%   \begin{align*}
%%     | w - \calI_h w |^2_{H^1(K)}
%%     &= \int_K ((w - \calI_h w)')^2 dx
%%     = \int_K w''(\xi)^2 \frac{1}{4} ((x-a) + (x-b))^2 dx
%%     \\
%%     &\leq \| w'' \|_{L^\infty(K)}^2 \int_K \frac{1}{4} ((x-a) + (x-b))^2 dx
%%     \leq h^3\| w'' \|_{L^2(K)}^2,
%%   \end{align*}
%%   where the inequality again follows from $|x - a| < h$ and $|x - b| < h$.
%%     \end{proof}
%% \end{lemma}

%% \begin{proposition}[One-dimensional linear interpolation error bound for $\Omega$]
%%   Let $\Omega \subset \RR^1$ be the domain, $\calT_h$ be a uniform triangulation over $\Omega$ of characteristic length $h$, $w \in \oplus_{K \in \calT_h}  C^2(K)$ be a function we wish to interpolate, and $\calI_h w \in \calV_h$ be the linear interpolant associated with $\calV_h \equiv \{ v \in C^0(\Omega) \ | \ v|_K \in \PP^1(K), \ \forall K \in \calT_h \}$. Then, the interpolation error satisfies
%%   \begin{align}
%%     \| w - \calI_h w \|_{L^2(\Omega)} &\leq \frac{1}{2} h^2 \| w'' \|_{L^\infty(\Omega)} \label{eq:fe_interp_lin_l2} \\
%%     | w - \calI_h w |_{H^1(\Omega)} &\leq h \| w'' \|_{L^\infty(\Omega)} \label{eq:fe_interp_lin_h1}
%%   \end{align}
%%   \begin{proof}
%%     The $L^2$ error bound follows from the application of~\eqref{eq:fe_interp_lin_l2_elem} to each element:
%%     \begin{equation*}
%%       \| w - \calI_h w \|^2_{L^2(\Omega)}
%%       =
%%       \sum_{K \in \calT_h} \| w - \calI_h w \|^2_{L^2(K)}
%%       \leq
%%       \frac{1}{h} \frac{1}{4} h^5 \| w'' \|_{L^\infty(\Omega)}^2
%%       = \frac{1}{4} h^4 \| w'' \|_{L^\infty(\Omega)}^2.
%%     \end{equation*}
%%     The $H^1$ error bound similarly follows from the application of~\eqref{eq:fe_interp_lin_h1_elem} to each element:
%%         \begin{equation*}
%%       \| w - \calI_h w \|^2_{L^2(\Omega)}
%%       =
%%       \sum_{K \in \calT_h} \| w - \calI_h w \|^2_{L^2(K)}
%%       \leq
%%       \frac{1}{h} h^3 \| w'' \|_{L^\infty(\Omega)}^2
%%       = h^2 \| w'' \|_{L^\infty(\Omega)}^2.
%%     \end{equation*}
%%   \end{proof}
%% \end{proposition}
%% The proposition shows that the $L^2$ interpolation error (i) depends on the maximum value of the second derivative $\| w '' \|_{L^\infty(\Omega)}$ and (ii) decreases as $h^2$.  The $H^1$ interpolation error similarly depends on $\| w'' \|_{L^\infty(\Omega)}$ but decreases as $h^1$. 
