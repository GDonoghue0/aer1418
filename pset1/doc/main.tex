\documentclass[preprint,11pt]{article}
%\documentclass{elsart3-1}

\usepackage{amsthm,amsmath,amssymb,amsfonts,amscd,amsbsy}%,latexsym,epsfig,subfig}
%\usepackage{graphicx,psfrag,verbatim,wasysym}
\usepackage{graphicx}
\usepackage[ruled,noline,linesnumbered]{algorithm2e}
%\usepackage{bm}
\usepackage[tight]{subfigure}
\usepackage{fullpage}
\usepackage{color}
%\usepackage{setspace}
%\usepackage[tight]{subfigure}
%\usepackage{listings}
\usepackage{enumerate}
\newcommand{\doi}[1]{\textsc{doi:} \textsf{#1}\xspace}

\DeclareGraphicsExtensions{.jpg,.png,.pdf}
\DeclareGraphicsRule{*}{mps}{*}{}

%\setlength{\parindent}{0em}
\setlength{\parskip}{0em}



% bold letters
\newcommand{\bola}{\mathbf{a}}
\newcommand{\bolc}{\mathbf{c}}
\newcommand{\bolf}{\mathbf{f}}
\newcommand{\bolg}{\mathbf{g}}
\newcommand{\boll}{\mathbf{l}}
\newcommand{\bolq}{\mathbf{q}}
\newcommand{\bolp}{\mathbf{p}}
\newcommand{\bolu}{\mathbf{u}}
\newcommand{\bolv}{\mathbf{v}}
\newcommand{\bolw}{\mathbf{w}}
\newcommand{\bolz}{\mathbf{z}}

\newcommand{\bolA}{\mathbf{A}}
\newcommand{\bolB}{\mathbf{B}}
\newcommand{\bolC}{\mathbf{C}}
\newcommand{\bolD}{\mathbf{D}}
\newcommand{\bolE}{\mathbf{E}}
\newcommand{\bolF}{\mathbf{F}}
\newcommand{\bolG}{\mathbf{G}}
\newcommand{\bolH}{\mathbf{H}}
\newcommand{\bolI}{\mathbf{I}}
\newcommand{\bolJ}{\mathbf{J}}
\newcommand{\bolK}{\mathbf{K}}
\newcommand{\bolL}{\mathbf{L}}
\newcommand{\bolM}{\mathbf{M}}
\newcommand{\bolN}{\mathbf{N}}
\newcommand{\bolO}{\mathbf{O}}
\newcommand{\bolP}{\mathbf{P}}
\newcommand{\bolQ}{\mathbf{Q}}
\newcommand{\bolR}{\mathbf{R}}
\newcommand{\bolS}{\mathbf{S}}
\newcommand{\bolT}{\mathbf{T}}
\newcommand{\bolU}{\mathbf{U}}
\newcommand{\bolV}{\mathbf{V}}
\newcommand{\bolW}{\mathbf{W}}
\newcommand{\bolX}{\mathbf{X}}
\newcommand{\bolY}{\mathbf{Y}}
\newcommand{\bolZ}{\mathbf{Z}}

% bold symbols
\newcommand{\bolalpha}{\boldsymbol{\alpha}}
\newcommand{\bolbeta}{\boldsymbol{\beta}}
\newcommand{\boleta}{\boldsymbol{\eta}}
\newcommand{\bolpsi}{\boldsymbol{\psi}}

% shadowed letters
\newcommand{\PP}{\mathbb{P}}
\newcommand{\RR}{\mathbb{R}}
\newcommand{\CC}{\mathbb{C}}
\newcommand{\ZZ}{\mathbb{Z}}

% mathcal letters
\newcommand{\calA}{\mathcal{A}}
\newcommand{\calB}{\mathcal{B}}
\newcommand{\calC}{\mathcal{C}}
\newcommand{\calD}{\mathcal{D}}
\newcommand{\calE}{\mathcal{E}}
\newcommand{\calF}{\mathcal{F}}
\newcommand{\calG}{\mathcal{G}}
\newcommand{\calH}{\mathcal{H}}
\newcommand{\calI}{\mathcal{I}}
\newcommand{\calJ}{\mathcal{J}}
\newcommand{\calK}{\mathcal{K}}
\newcommand{\calL}{\mathcal{L}}
\newcommand{\calM}{\mathcal{M}}
\newcommand{\calN}{\mathcal{N}}
\newcommand{\calO}{\mathcal{O}}
\newcommand{\calP}{\mathcal{P}}
\newcommand{\calQ}{\mathcal{Q}}
\newcommand{\calR}{\mathcal{R}}
\newcommand{\calS}{\mathcal{S}}
\newcommand{\calT}{\mathcal{T}}
\newcommand{\calU}{\mathcal{U}}
\newcommand{\calV}{\mathcal{V}}
\newcommand{\calW}{\mathcal{W}}
\newcommand{\calX}{\mathcal{X}}
\newcommand{\calY}{\mathcal{Y}}
\newcommand{\calZ}{\mathcal{Z}}


% derivatives
\newcommand{\pp}[2]{\frac{\partial #1}{\partial #2}}
\newcommand{\dd}[2]{\frac{d #1}{d #2}}


% fraction shortcut
\newcommand{\f}[2]{\frac{#1}{#2}}
\newcommand{\slfrac}[2]{\left.#1\middle/#2\right.}

% common operators
\newcommand{\vvvert}{|\kern-1pt|\kern-1pt|}
\newcommand{\enorm}[1]{\vvvert #1 \vvvert}


% matrices
\newcommand{\bmat}[1]{\left(\begin{array}{#1}}
\newcommand{\emat}{\end{array}\right)} 



\newcommand{\blist}{\begin{list}{\ballrefb}{\leftmargin=2.0em}
  \setlength{\itemsep}{2pt}
  \setlength{\parskip}{0pt}}
\newcommand{\elist}{\end{list}}



% common format strings
\def\etal{{\it et al.}}
\def\ie{{\it i.e.}}
\def\eg{{\it e.g.}}


\DeclareMathOperator*{\arginf}{arg\,inf}
\DeclareMathOperator*{\argsup}{arg\,sup}
\DeclareMathOperator*{\argmax}{arg\,max}
\DeclareMathOperator*{\argmin}{arg\,min}


%\renewcommand{\theenumi}{(\alph{enumi})}
% 
%
%

\begin{document}

\noindent\begin{Large}\textbf{AER1418 Problem Set 1}\end{Large} \hfill Due Tuesday, 6 February 2018, at 14:00.

\vspace{1em}
\noindent A hard copy of the problem set is due in class at the specified time.  %For problems that require coding, please turn in a hard copy of the program listings and also upload the code to the Portal to facilitate the grading process.
Please adhere to the collaboration policy: the final write up must be prepared individually without consulting others. (See the syllabus for details.)

\section*{Part 1. Formulation and analysis}
We wish to understand the insulation characteristics of a jacket worn by a person walking outside in a cold, windy day.  In our simplified analysis, the person is a circle that acts as a heat source, and the jacket is a ring that surrounds the person. We wish to model the temperature distribution in the ring-shaped jacket and evaluate the temperature on the inner surface, which contacts the person.

We now provide a mathematical description of the problem.  All variables are nondimensionalized.  We introduce a two-dimensional annular domain
\begin{equation*}
  \Omega_{\rm annulus} \equiv \{ x \in \RR^2 \ | \ r_{\rm in} < \sqrt{x_1^2 + x_2^2} < r_{\rm out} \},
\end{equation*}
where $r_{\rm in}$ and $r_{\rm out}$ are the inner and outer radius, respectively, so that $0 < r_{\rm in} < r_{\rm out}$. The associated inner and outer boundaries are
\begin{align*}
  \Gamma_{\rm in} &\equiv \{ x \in \RR^2 \ | \ \sqrt{x_1^2 + x_2^2} = r_{\rm in} \}, \\
  \Gamma_{\rm out} &\equiv \{ x \in \RR^2 \ | \ \sqrt{x_1^2 + x_2^2} = r_{\rm out} \}.
\end{align*}
The steady-state temperature distribution inside the annulus is modeled by the (steady-state) heat equation
\begin{equation*}
  - \nabla \cdot (k \nabla u) = 0 \quad \text{in } \Omega_{\rm annulus} ,
\end{equation*}
where $k > 0$ is the thermal conductivity. The heat transfer at the inner wall $\Gamma_{\rm in}$ from the heat source is modeled by
\begin{equation*}
  - n \cdot (k \nabla u) = -g \quad \text{on } \Gamma_{\rm in},
\end{equation*}
where $n$ is the outward-pointing normal with respect to $\Omega$, and $g > 0$ is the heat flux. %; here, the negative flux indicates that the energy is transferred from the heat source to the annulus.
The heat transfer at the outer wall $\Gamma_{\rm out}$ to the air is modeled by 
\begin{equation*}
  - n \cdot (k \nabla u) = B_{\rm out} (u - u_{\infty})  \quad \text{on } \Gamma_{\rm out},
\end{equation*}
where $B_{\rm out} > 0$ is the Biot number that characterizes the heat transfer to the surrounding air.  Without loss of generality, we set $u_{\infty} = 0$; i.e., the temperature $u$ is defined with respect to the ambient air temperature.

Answer the following questions:
\begin{enumerate}[(a)]
\item Derive a variational form of the two-dimensional heat conduction problem.  Specifically, identify (i) the trial and test spaces, (ii) bilinear form, and (iii) linear form.
\item The two-dimensional problem is axisymmetric. Reformulate the variational problem as a one-dimensional variational problem on the domain $\Omega \equiv (r_{\rm in}, r_{\rm out}) \subset \RR^1$.  Specifically, identify (i) the trial and test spaces, (ii) bilinear form, and (iii) linear form.

  \emph{Note:} the integrand of the bilinear form should depend on the coordinate $x \in \Omega \subset \RR^1$.
\end{enumerate}
For the following questions, the variational problem, bilinear form, and linear form refer to those obtained in (b). 
\begin{enumerate}[(a)]
  \setcounter{enumi}{2}
\item Show that the bilinear form is coercive.
\item Show that the bilinear form is continuous.
\item Show that the linear form is continuous.
\item Prove or provide a counter example to the following statement: the variational problem has a unique solution.
\item Suppose the Biot number is zero: $B_{\rm out} = 0$.  Prove that the bilinear form is not coercive.  Prove or provide a counter example to the following statement:  the variational problem has a unique solution.

  \emph{Note:} to prove that the bilinear from is not coercive, you need to find only one function for which the coercivity condition is violated.
\item State a minimization form of the variational problem.  Specifically, identify (i) the function space and (ii) energy functional.
\end{enumerate}

\section*{Part 2. Implementation}
We now wish to develop a finite-element code that approximates the solution to the one-dimensional variational problem obtained in Part 1. The code should use a piecewise linear finite element space with a uniform (but arbitrary) spacing $h > 0$.%and accept an arbitrary discretization parameter $h > 0$.  
\begin{enumerate}[(a)]
\item Write a finite-element code that solves the heat conduction problem.

  \emph{Note.} We need to evaluate the entries of the stiffness matrix $\hat A_h$ and load vector $\hat f_h$.  For example, to evaluate the diagonal entries of $\hat A_h$, consider a hat-shaped piecewise-linear basis function delineated by points $x_{i-1}$, $x_i$, and $x_{i+1}$ with the support $(x_{i-1},x_{i+1})$ and the peak at $x_i$; then, express the diagonal entry as a function of the coordinates $x_{i-1}$, $x_{i}$, and $x_{i+1}$.  Evaluate all other entries of the stiffness matrix and load vector in a similar manner. 
%\item Evaluate all other entries of the stiffness matrix.  Introduce an appropriate set of points that delineates basis functions, and express entries as the coordinates of those points.
%\item Write a finite-element code that solves the heat conduction problem.
\item Plot the finite-element solution $u_h$ for $r_{\rm in} = 1/2$, $r_{\rm out} = 2$, $k = 1$, $g = 1$, $B_{\rm out} = 1$, and a uniform mesh spacing of $h = 1/4$.  
\item Consider the output $\ell^{o}(u) \equiv g r_{\rm in} u(x = r_{\rm in})$. Show that
  \begin{equation*}
    \ell^o(u) - \ell^o(u_h) = a(e,e),
  \end{equation*}
  where $e \equiv u - u_h$ is the error in the finite element approximation $u_h$.  
\item Solve the finite element problem for $r_{\rm in} = 1/2$, $r_{\rm out} = 2$, $k = 1$, $g = 1$, $B_{\rm out} = 1$, and uniform grid spacings of $h = 1/2, 1/4, 1/8, \dots, 1/64$. Report the reference value of the output for $h_{\rm ref} \equiv 1/512$, $\ell^o(u_{h_{\rm ref}})$.  Plot the output error (with respect to $\ell^o(u_{h_{\rm ref}})$) against the grid spacing $h$ in log-log scale, and report the observed convergence rate.  Does the observed convergence rate match your expectation based on the theory?
%\item Execute the finite-element code for $B_{\rm out} = 0$.  %How does the code fail?
%  Does the observed behavior match your expectation?
\end{enumerate}


\end{document}

