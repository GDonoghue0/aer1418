\documentclass[preprint,11pt]{article}
%\documentclass{elsart3-1}

\usepackage{amsthm,amsmath,amssymb,amsfonts,amscd,amsbsy}%,latexsym,epsfig,subfig}
%\usepackage{graphicx,psfrag,verbatim,wasysym}
\usepackage{graphicx}
\usepackage[ruled,noline,linesnumbered]{algorithm2e}
%\usepackage{bm}
\usepackage[tight]{subfigure}
\usepackage{fullpage}
\usepackage{color}
%\usepackage{setspace}
%\usepackage[tight]{subfigure}
%\usepackage{listings}
\usepackage{enumerate}
\newcommand{\doi}[1]{\textsc{doi:} \textsf{#1}\xspace}

\DeclareGraphicsExtensions{.jpg,.png,.pdf}
\DeclareGraphicsRule{*}{mps}{*}{}

\setlength{\parindent}{0em}
\setlength{\parskip}{0em}

\DeclareGraphicsExtensions{.jpg,.png,.pdf}
\DeclareGraphicsRule{*}{mps}{*}{}
\graphicspath{{fig/}}

\include{commands}

%\renewcommand{\theenumi}{(\alph{enumi})}
% 
%
%

\begin{document}

\begin{Large}\textbf{AER1418 Problem Set 1 Solution}\end{Large} \hfill %Due Thursday, January 21st, at 12:10pm.

\vspace{1em}

\section*{Part 1. Formulation and analysis}
\begin{enumerate}[(a)]
\item Since both the Neumann and Robin boundary conditions are natural boundary conditions, we choose for our function space
  \begin{equation*}
    \calV \equiv H^1(\Omega_{\rm annulus}).
  \end{equation*}
  We then invoke the weighted residual method to obtain
  \begin{align*}
    \int_{\Omega_{\rm annulus}} v (- \nabla \cdot (k \nabla u)) dx
    &=
    \int_{\Omega_{\rm annulus}} \nabla v \cdot k \nabla u dx
    - \int_{\Gamma_{\rm in}} v n \cdot k \nabla u ds
    - \int_{\Gamma_{\rm out}} v n \cdot k \nabla u ds
    \\
    &=
    \int_{\Omega_{\rm annulus}} \nabla v \cdot k \nabla u dx
    - \int_{\Gamma_{\rm in}} v g ds
    + \int_{\Gamma_{\rm out}} v (B_{\rm air} (u  -u_\infty)) ds;
  \end{align*}
  here we have applied the boundary conditions to obtain the second equality.
  We hence obtain the following weak formulation: find $u \in \calV$ such that
  \begin{equation*}
    a(u,v) = \ell(v) \quad \forall v \in \calV,
  \end{equation*}
  where
  \begin{align*}
    a(w,v) &\equiv \int_{\Omega_{\rm annulus}} \nabla v \cdot k \nabla u dx + \int_{\Gamma_{\rm out}} B_{\rm air} v w ds, \\
    \ell(v) &\equiv \int_{\Gamma_{\rm in}} v g ds .
  \end{align*}
\item To recast the axisymmetric 2d problem as a 1d problem, we first introduce the function space
  \begin{equation*}
    \calV \equiv H^1(\Omega \equiv (r_{\rm in},r_{\rm out})).
  \end{equation*}
  We then make the following substitutions of the integrals: for $f \in \calV$,
  \begin{align*}
    \int_{\Omega_{\rm annulus}} f dx
    &= \int_{\Omega \equiv (r_{\rm in},r_{\rm out})} f(r) 2\pi r dr \\
    \int_{\Gamma_{\rm in}} f ds
    &= 2\pi r_{\rm in} f(r_{\rm in}) \\
    \int_{\Gamma_{\rm out}} f ds
    &= 2\pi r_{\rm out} f(r_{\rm out}).
  \end{align*}
  We have hence obtain the following weak formulation: find $u \in \calV$ such that
  \begin{equation*}
    a(u,v) = \ell(v) \quad \forall v \in \calV,
  \end{equation*}
  where
  \begin{align*}
    a(w,v) &\equiv \int_{\Omega} k x \dd{v}{x} \dd{u}{x} dx + B_{\rm air} r_{\rm out} v(r_{\rm out})w(r_{\rm out}) \\
    \ell(v) &\equiv  g r_{\rm in} v(r_{\rm in})   .
  \end{align*}
  (Note that we have divided through by $2\pi$ and are using $x$ as the coordinate variable.)
\item We first invoke the Poincar\'e-Friedrichs inequality for $\Omega \equiv (r_{\rm in}, r_{\rm out})$ and $\Gamma \equiv \{ r_{\rm out} \} \subset \partial \Omega$: $\forall v \in \calV$,
  \begin{align*}
    \| v \|_{L^2(\Omega)}^2 \leq C_{\rm PF} (|v|^2_{H^1(\Omega)} + v(r_{\rm out})^2). 
  \end{align*}
  It follows that
  \begin{align*}
    \| v \|_{H^1(\Omega)}^2
    &\equiv
    \| v \|_{L^2(\Omega)}^2  + | v |_{H^1(\Omega)}^2
    \leq
    C_{\rm PF} (|v|^2_{H^1(\Omega)} + v(r_{\rm out})^2) + | v |_{H^1(\Omega)}^2
    \\
    &= (1 + C_{\rm PF}) | v |^2_{H^1(\Omega)} +  C_{\rm PF} v(r_{\rm out})^2
    \\
    &= (1 + C_{\rm PF}) \int_{x = r_{\rm in}}^{r_{\rm out}} (\dd{v}{x})^2 dx +  C_{\rm PF} v(r_{\rm out})^2
    \\
    &\leq
    (1 + C_{\rm PF}) \frac{1}{k r_{\rm in}} \int_{x = r_{\rm in}}^{r_{\rm out}} kx (\dd{v}{x})^2 dx +  C_{\rm PF} \frac{1}{B_{\rm out} r_{\rm out}}B_{\rm out} r_{\rm out} v(r_{\rm out})^2
    \\
    &\leq
    \max\{   (1 + C_{\rm PF}) \frac{1}{k r_{\rm in}}, C_{\rm PF} \frac{1}{B_{\rm out} r_{\rm out}} \} ( \int_{x = r_{\rm in}}^{r_{\rm out}} kx (\dd{v}{x})^2 dx + B_{\rm out} r_{\rm out} v(r_{\rm out})^2) \\
 &\leq
         \max\{   (1 + C_{\rm PF}) \frac{1}{k r_{\rm in}}, C_{\rm PF} \frac{1}{B_{\rm out} r_{\rm out}} \} a(v,v).
  \end{align*}
  Hence, the bilinear form is coercive with a coercivity constant
  \begin{equation*}
    \alpha = \left( \max\left\{   (1 + C_{\rm PF}) \frac{1}{k r_{\rm in}}, C_{\rm PF} \frac{1}{B_{\rm out} r_{\rm out}} \right\} \right)^{-1}.
  \end{equation*}
\item The bilinear form is continuous because
  \begin{align*}
    a(w,v) &\equiv \int_{\Omega} kx \dd{v}{x} \dd{w}{x} dx + B_{\rm out} r_{\rm out} v(r_{\rm out}) w(r_{\rm out})
    \\
    &\leq
    k r_{\rm out} | v |_{H^1(\Omega)} | w |_{H^1(\Omega)} + B_{\rm out} r_{\rm out} C_{\rm tr}^2 \| v \|_{H^1(\Omega)} \| w \|_{H^1(\Omega)}
    \\
    &\leq
    \max\{ k r_{\rm out}, B_{\rm out} r_{\rm out} C_{\rm tr}^2 \}  \| v \|_{H^1(\Omega)}\| w \|_{H^1(\Omega)},
  \end{align*}
  where the first inequality follows from the trace inequality, $|v(r_{\rm out})| \leq C_{\rm tr}\|v\|_{H^1(\Omega)}$, $\forall v \in \calV$.
\item The linear form is continuous because
  \begin{equation*}
    | \ell(v) | = |gr_{\rm in} v(r_{\rm in})| \leq gr_{\rm in} C_{\rm tr} \|v\|_{H^1(\Omega)}, 
  \end{equation*}
  where the inequality follows from the trace inequality.  We note that $\| \ell(\cdot) \|_{\calV'} \leq  gr_{\rm in} C_{\Gamma}$.
\item We note that all conditions of the Lax-Milgram theorem are satisfied: $\Omega$ is a Lipschitz domain; $a(\cdot,\cdot)$ is coercive and continuous; and $\ell(\cdot)$ is continuous.  Hence, a unique solution to the variational problem exists.
\item Let $B_{\rm out} = 0$.  Then, for $w = v = 1 \in H^1(\Omega)$,
  \begin{equation*}
    a(v,v) = \int_{\Omega} k x \dd{v}{x} \dd{w}{x} dx = \int_{\Omega} k x 0 dx = 0.
  \end{equation*}
  Since $\| v \|_{H^1(\Omega)} = 1$, we conclude that there is no $\alpha > 0$ for which $a(v,v) \geq \alpha \| v \|_{H^1(\Omega)}^2$; hence, the bilinear form is not coercive.

  The problem does not have a unique solution.  In particular, suppose $u \in \calV$ is a solution to the variational form.  Then, $u + c$, where $c$ is any constant function, is also a solution to the variational form.
\item The minimization form of the variational problem is as follows: find $u \in \calV$ such that
  \begin{align*}
    u = \argmin_{w \in \calV} J(w),
  \end{align*}
  where
  \begin{align*}
    J(w) \equiv \frac{1}{2} a(w,w) - \ell(w)
    =
    \frac{1}{2}  (\int_{\Omega_{\rm annulus}} \nabla v \cdot k \nabla u dx + \int_{\Gamma_{\rm out}} B_{\rm air} v w ds) - \int_{\Gamma_{\rm in}} v g ds.
  \end{align*}
\end{enumerate}


\section*{Part 1. Implementation}
\begin{enumerate}
\item Implementation.
\item The finite element solution for for $r_{\rm in} = 1/2$, $r_{\rm out} = 2$, $k = 1$, $g = 1$, $B_{\rm out} = 1$, and $h = 1/4$ is shown in Figure~\ref{fig:state}.
  \begin{figure}[!h]
    \centering
    \includegraphics[width=0.45\textwidth]{state}
    \caption{Solution field. \label{fig:state}}
  \end{figure}
\item We note that $\ell^o(w) \equiv gr_{\rm in} w(r_{\rm in}) = \ell(w)$, $\forall w \in \calV$. It hence follows
  \begin{align*}
    \ell^o(u) - \ell^o(u_h)
    &= \ell(u) - \ell(u_h) & \text{(equivalence of $\ell$ and $\ell^o$)} \\
    &= \ell(u - u_h) & \text{(linearity of $\ell$)} \\
    &= a(u, u - u_h) & \text{($a(u,v) = \ell(v)$, $\forall v \in \calV$)} \\
    &= a(u - u_h, u - u_h), & \text{(Galerkin orthogonality)}
  \end{align*}
  which is the desired equality.
\item
  The convergence plot for the output is shown in Figure~\ref{fig:conv}.  The error varies as $e \sim h^r$ and hence produces a line in log-log scale.
  \begin{figure}[!h]
    \centering
    \includegraphics[width=0.45\textwidth]{conv}
    \caption{Output convergence. \label{fig:conv}}
  \end{figure}
  The observed convergence rate for the two finest meshes is
  \begin{equation*}
    r_{\rm obs} = \frac{\log(\ell^o(u_{h = 1/64}) / \ell^o(u_{h = 1/32}))}{ \log( (1/64) / (1/32))} = 2.07.
  \end{equation*}
  This value is close to the expected convergence rate of 2 for $a(e,e) \equiv \enorm{e}^2$, the \emph{square} of the energy norm of the error, for linear finite element.
\end{enumerate}

\end{document}

